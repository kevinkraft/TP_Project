\documentclass[10pt,a4paper]{beamer}

\usepackage{xcolor}
\usepackage{graphicx}
\usepackage{float}
\usepackage{listings}
\usepackage{enumerate}

%for beamer style
\usetheme{Madrid}
\author{Kevin Maguire}
\defbeamertemplate*{footline}{shadow theme}
{\leavevmode
 \hbox{\begin{beamercolorbox}[wd=.5\paperwidth,ht=2.5ex,dp=1.125ex,leftskip=.3cm plus1fil,rightskip=.3cm]{author in head/foot}
\usebeamerfont{author in head/foot}\insertframenumber\,/\,\inserttotalframenumber\hfill\insertshortauthor
\end{beamercolorbox}
\begin{beamercolorbox}[wd=.5\paperwidth,ht=2.5ex,dp=1.125ex,leftskip=.3cm,rightskip=.3cm plus1fil]{title in head/foot}
    \usebeamerfont{title in head/foot}\insertshorttitle
\end{beamercolorbox}}
  \vskip0pt
}

\begin{document}

\begin{frame}
\title{New Stripping line for $D^{*}$ tagged $D^{0} \rightarrow h h \pi^{0}$}
\author{\textbb{Kevin Maguire}, \newline Chirs Parkes, \newline  Mika Vesterinen \newline \newline \newline kevin.maguire@cern.ch}
\date{\today}
\maketitle
\pagenumbering{arabic}
\end{frame}

\begin{frame}

\begin{minipage}{5.25cm}
\vspace{1.5em}
\begin{itemize}
\item{We consider candiates that are TOS on the inclusive $D^{*}$ HLT2 line}
\item{HLT2 has tight cuts on the daughter particles. We want to resurrect these candidates in the fastest way possible in the stripping}
\end{itemize}
\begin{center}
\includegraphics[scale=0.4]{figs/decay.png}
\end{center}
\end{minipage}
\hspace{2.0em}
\begin{minipage}{5.25cm}
\begin{enumerate}
\item{Filter on HLT2 decision}
\item{Construct a 2 track vertex from the h h daughters with PIDK and flight distance cuts}
\item{One of these daughters is required to have a large IP AND PT by the HLT1TrackAllL0 line}
\item{Combine with a $\pi^{0}$ with loose PT cut$ > 500~$MeV to form a $D^{0}$}
\item{Add slow pion with PT $> 300~$MeV to form a $D^{*}$}

\item{TOS requirement on the inclusive D* line}
\end{enumerate}
\end{minipage}
\end{frame}

\begin{frame}
\frametitle{Timing and Retention}
\begin{center}
\begin{tabular}{l|ccc}
\hline
\hline
\vspace{0.5em}
Stripping Line & Retention (\%)  & $t$/$(ms)$ & Signal Purity (\%) \\
\hline
Kpipi0 Line       & 0.2833         & 0.239      & 6.2                \\
Kpipi0WS  Line    & 0.1318         & 0.132      & 3                  \\
pipipi0 Line      & 0.3252         & 0.193      & 2.4 (as PIDK$<$7) \\
KKpi0 Line        & 0.0264         & 0.087      & 2                  \\
Kpipi0\_M Line    & 0.0264         & 0.100      & 9.1*                \\
pipipi0\_M Line   & 0.0183         & 0.119      & 1.5* (as PIDK$<$7)   \\
KKpi0\_M Line     & 0.0014         & 0.080      & 8.7*                \\
MassConstraint    & 0.2826         & 0.660      & 5.1 ???            \\
\hline
\hline
\end{tabular}
\newline
\vspace{1.0em}
* Low statistics
\end{center}
\vspace{1.0em}
\begin{itemize}
\item{``MassConstraint'' line indicates a pi0 refit will not be possible within time constraints}
\end{itemize}
\end{frame}

%\begin{frame}
%\frametitle{Resulting ntuple Analysis: D0\_M}
%\begin{minipage}{5.25cm}
%\vspace{1.0em}
%\begin{itemize}
%\item{with no cuts}
%\item{with cuts}
%\begin{itemize}
%\item{pi0\_MM, (125,146)} 
%\item{delta mass, (137,153)
%\end{itemize}
%\item{with fit}
%\begin{itemize}
%\item{\#signal $=2113.69 \slash 10447$} 
%\item{$\Rightarrow 5.2$ Million signal events in 2012 data}
%\end{itemize}
%\end{itemize}
%\vspace{1.5em}
%\includegraphics[scale=0.3]{//afs/cern.ch/user/k/kmaguire/cmtuser/DaVinci_v33r5/Phys/StrippingSelections/Analysis/1.7/Kpipi0RS/pdf/fit_D0M.pdf}
%\end{minipage}
%\hspace{2.0em}
%\begin{minipage}{5.25cm}
%\includegraphics[scale=0.3]{//afs/cern.ch/user/k/kmaguire/cmtuser/DaVinci_v33r5/Phys/StrippingSelections/Analysis/1.7/Kpipi0RS/pdf/D0_M_plot.pdf}
%\newline
%\includegraphics[scale=0.3]{//afs/cern.ch/user/k/kmaguire/cmtuser/DaVinci_v33r5/Phys/StrippingSelections/Analysis/1.7/Kpipi0RS/pdf/D0_M_cut_plot.pdf}
%\end{minipage}
%\end{frame}

\begin{frame}
\frametitle{Resulting ntuple Analysis: Raw Signal}
\begin{itemize}
\item{Kpipi0}
\end{itemize}
\begin{minipage}{5.25cm}
\begin{center}
\includegraphics[scale=0.16]{//afs/cern.ch/user/k/kmaguire/cmtuser/DaVinci_v33r5/Phys/StrippingSelections/Analysis/2.0/Kpipi0RS/pdf/D0_M_plot.pdf}
\end{center}
\end{minipage}
\begin{minipage}{5.25cm}
\begin{center}
\includegraphics[scale=0.16]{//afs/cern.ch/user/k/kmaguire/cmtuser/DaVinci_v33r5/Phys/StrippingSelections/Analysis/2.0/Kpipi0RS/pdf/Dstar_M-D0_M_plot.pdf}
\end{center}
\end{minipage}
\begin{itemize}
\item{pipipi0}
\end{itemize}
\begin{minipage}{5.25cm}
\begin{center}
\includegraphics[scale=0.16]{//afs/cern.ch/user/k/kmaguire/cmtuser/DaVinci_v33r5/Phys/StrippingSelections/Analysis/2.0/pipipi0/pdf/D0_M_plot.pdf}
\end{center}
\end{minipage}
\begin{minipage}{5.25cm}
\begin{center}
\includegraphics[scale=0.16]{//afs/cern.ch/user/k/kmaguire/cmtuser/DaVinci_v33r5/Phys/StrippingSelections/Analysis/2.0/pipipi0/pdf/Dstar_M-D0_M_plot.pdf}
\end{center}
\end{minipage}
\begin{itemize}
\item{KKpi0}
\end{itemize}
\begin{minipage}{5.25cm}
\begin{center}
\includegraphics[scale=0.16]{//afs/cern.ch/user/k/kmaguire/cmtuser/DaVinci_v33r5/Phys/StrippingSelections/Analysis/2.0/KKpi0/pdf/D0_M_plot.pdf}
\end{center}
\end{minipage}
\begin{minipage}{5.25cm}
\begin{center}
\includegraphics[scale=0.16]{//afs/cern.ch/user/k/kmaguire/cmtuser/DaVinci_v33r5/Phys/StrippingSelections/Analysis/2.0/KKpi0/pdf/Dstar_M-D0_M_plot.pdf}
\end{center}
\end{minipage}

\end{frame}

\begin{frame}
\frametitle{Resulting ntuple Analysis: Kpipi0 Fits}
\begin{minipage}{5.25cm}
\begin{itemize}
\item{with cuts}
\begin{itemize}
\item{pi0\_MM, (125,146)}  
\item{delta mass, (137,153)}
\item{Purity 6.2 \%}
\item{$\Rightarrow 6.7$ Milion signal events in 2012 data}
\end{itemize}
\end{itemize}
\includegraphics[scale=0.3]{//afs/cern.ch/user/k/kmaguire/cmtuser/DaVinci_v33r5/Phys/StrippingSelections/Analysis/2.0/Kpipi0RS/pdf/fit_D0M.pdf}
\end{minipage}
\hspace{2.0em}
\begin{minipage}{5.25cm}
\begin{itemize}
\item{with cuts}
\begin{itemize}
\item{pi0\_MM, (125,146)}  
\item{D0\_M, (1810,1930)}
\item{Purity 6.4 \%}
\item{$\Rightarrow 6.9$ Million signal events in 2012 data}
\end{itemize}
\end{itemize}
\includegraphics[scale=0.3]{//afs/cern.ch/user/k/kmaguire/cmtuser/DaVinci_v33r5/Phys/StrippingSelections/Analysis/2.0/Kpipi0RS/pdf/fit_DeltaM.pdf}
\end{minipage}
\end{frame}

%\begin{frame}
%\frametitle{Resulting ntuple Analysis: delta mass}
%\begin{minipage}{5.25cm}
%\vspace{1.0em}
%\begin{itemize}
%\item{with no cuts}
%\item{with cuts}
%\begin{itemize}
%\item{pi0\_MM, (125,146)}  
%\item{D0\_M, (1810,1930)}
%\end{itemize}
%\item{with fit}
%\begin{itemize}
%\item{\#signal $=  2922.34 \slash  14'116 $}
%\item{$\Rightarrow 7.2$ Million signal events in 2012 data}
%\end{itemize}
%\end{itemize}
%\vspace{1.5em}
%\includegraphics[scale=0.3]{//afs/cern.ch/user/k/kmaguire/cmtuser/DaVinci_v33r5/Phys/StrippingSelections/Analysis/2.0/Kpipi0RS/pdf/fit_DeltaM.pdf}
%\end{minipage}
%\hspace{2.0em}
%\begin{minipage}{5.25cm}
%\includegraphics[scale=0.3]{//afs/cern.ch/user/k/kmaguire/cmtuser/DaVinci_v33r5/Phys/StrippingSelections/Analysis/2.0/Kpipi0RS/pdf/Dstar_M-D0_M_plot.pdf}
%\newline
%\includegraphics[scale=0.3]{//afs/cern.ch/user/k/kmaguire/cmtuser/DaVinci_v33r5/Phys/StrippingSelections/Analysis/2.0/Kpipi0RS/pdf/Dstar_M-D0_M_cut_plot.pdf}
%\end{minipage}
%\end{frame}

\begin{frame}
\frametitle{Resulting ntuple Analysis: pi0\_M refit effect}
\begin{minipage}{5.25cm}
\begin{itemize}
\item{With $\pi^{0}$ mass refit}
\item{\# retained events $-$}
\end{itemize}
\includegraphics[scale=0.3]{//afs/cern.ch/user/k/kmaguire/cmtuser/DaVinci_v33r5/Phys/StrippingSelections/Analysis/2.0/Kpipi0RS_MassConstraint/pdf/D0_M_plot.pdf}
\end{minipage}
\hspace{2.0em}
\begin{minipage}{5.25cm}
\begin{itemize}
\item{Without $\pi^{0}$ mass refit}
\item{\# retained events $-$}
\end{itemize}
\includegraphics[scale=0.3]{//afs/cern.ch/user/k/kmaguire/cmtuser/DaVinci_v33r5/Phys/StrippingSelections/Analysis/2.0/Kpipi0RS/pdf/D0_M_plot.pdf}
\end{minipage}
\begin{itemize}
\item{Scaled up to full 2012 data set the difference is - events}
\item{not much difference, but mass refit takes much longer}
\end{itemize}
\end{frame}

\begin{frame}
\frametitle{Conclusion:}
\begin{itemize}
\item{We found that:}
\begin{itemize}
\item{ - cuts imporve the stripping line, and will be added}
\item{ - cuts do not imporve the stripping line}
\item{Pi0 Mass Refit takes too long, no existing LHCb tool to make this fast enough}
\item{Fast and efficient stripping line allows us to have a low pi0\_PT and high signal retention}
\end{itemize}
\end{itemize}
\hspace{3.0em}
%\begin{itemize}
\item{$\Rightarrow$ Well tested stripping line ready for next stripping at the end of August}
%\end{itemize}
\vspace{3.0em}
%\begin{itemize}
\hspace{1.0em} \item{\underline{Next:} Trigger for tagged $D^{0} \rightarrow h h \pi^{0}$}
%\end{itemize}
\end{frame}

\begin{frame}
\frametitle{Back up Slides}
\end{frame}

\begin{frame}
\frametitle{Order of Cuts}
\begin{enumerate}
\item{GECs}
\item{Pion (PT $>$ 500 MeV) \& (TRGHOSTPROB $<$ 0.35) \& \textcolor{red}{(PIDK $<$ 7)}}
\item{Kaon (PT $>$ 500 MeV) \& (TRGHOSTPROB $<$ 0.35) \& \textcolor{red}{(PIDK $>$ 7)}}		
\item{Slowpi (PT $>$ 300 MeV) \& (TRGHOSTPROB $<$ 0.35) \& (PIDe $<$ 5) \& (MIPCHI2DV(PRIMARY)$<$ 9.0)}
\item{Pi0R (PT $>$ \textcolor{red}{500 MeV}) \& \textcolor{red}{(M $>$ 135-15 MeV) \& (M $<$ 135+15 MeV)}}
\item{Pi0M (PT $>$ 1000 MeV) (not used)}
\item{\textcolor{red}{Kst (((ACHILD(PT,1) $>$ 1.7*GeV) \& (ACHILD(BPVIPCHI2(),1) $>$ 36)) $|$ ((ACHILD(PT,2) $>$ 1.7*GeV) \& (ACHILD(BPVIPCHI2(),2) $>$ 36))) \& (AM $<$ 1850*MeV) \& (ADOCACHI2CUT(15,''))}}
\item{\textcolor{red}{Kst (VFASPF(VCHI2/VDOF) $<$ 3 \& (BPVVDCHI2 $>$ 100))}}
\item{D0 (ADAMASS('D0') $<$ \textcolor{red}{150 $+$ 10 MeV}) \& \textcolor{red}{(APT $>$ 1400 MeV)}}
\item{D0 \textcolor{red}{(DMASS('D0') $<$ 150 MeV)} }
\item{Dst (AM - ACHILD(M,1) $<$ 180 MeV) \& (ADOCACHI2CUT(20,''))}
\item{Dst (VFASPF(VCHI2/VDOF) $<$ 9.0)}			
\item{\textcolor{red}{HLT2 TOS}}		
\item[]{\textcolor{red}{\hspace{9.0em} RED$=$ADDED TO DRAFT }}		
\end{enumerate}
\end{frame}

\begin{frame}
\frametitle{Timing and Retention, Full}
\begin{center}
\begin{tabular}{l|ccc}
\hline
\hline
\vspace{0.5em}
Stripping Line    & Retention (\%) & $t$/$(ms)$ & Signal Purity (\%) \\
\hline
Kpipi0 Line       & 0.2833         & 0.239      & 6.2                \\
Kpipi0WS  Line    & 0.1318         & 0.132      & 3                  \\
pipipi0 Line      & 0.3252         & 0.193      & 2.4 (as PIDK$<$7) \\
KKpi0 Line        & 0.0264         & 0.087      & 2                  \\
Kpipi0\_M Line    & 0.0264         & 0.100      & 9.1*                \\
pipipi0\_M Line   & 0.0183         & 0.119      & 1.5* (as PIDK$<$7)    \\
KKpi0\_M Line     & 0.0014         & 0.080      & 8.7*                \\
MassConstraint    & 0.2826         & 0.660      & 5.1 ???            \\
pi0M tight        & 0.2623         & 0.437      & 6.5 ~~~            \\
D0 APT $>$ 2000   & 0.2373         & 0.406      & 6.5 ~~~            \\
Pion PIDK $<$ 0   & 0.2373         & 0.404      & 6.5 ~~~            \\
no pi0PT cut      & 0.3212         & 0.542      & 4.8 ~~~            \\
\hline
\hline
\end{tabular}
\end{center}
\newline
\vspace{1.0em}
* Low statistics
\end{frame}

\begin{frame}
\frametitle{Resulting ntuple Analysis: pi0\_PT cut effect}
\begin{minipage}{5.25cm}
\begin{itemize}
\item{With no pi0\_PT cut}
\item{$-$/$-$ signal events}
\item{$0 \rightarrow 500$ PT cut}
\item{Loss of $-~$\% of ``signal'' events}
\end{itemize}
\includegraphics[scale=0.3]{//afs/cern.ch/user/k/kmaguire/cmtuser/DaVinci_v33r5/Phys/StrippingSelections/Analysis/2.0/Kpipi0RS_pi0PTConstraint/pdf/fit_D0M.pdf}
\end{minipage}
\hspace{2.0em}
\begin{minipage}{5.25cm}
\begin{itemize}
\item{pi0\_PT $> 1000~$MeV}
\item{$-$/$-$ signal events}
\item{$500 \rightarrow 1000$ PT cut}
\item{Loss of $-$\% of signal events}
\end{itemize}
\includegraphics[scale=0.3]{//afs/cern.ch/user/k/kmaguire/cmtuser/DaVinci_v33r5/Phys/StrippingSelections/Analysis/2.0/Kpipi0RS/pdf/fit_D0M_pi0PTcut.pdf}
\end{minipage}
\end{frame}

\begin{frame}[fragile]
\frametitle{Code for Pi0 Mass Refit}
\begin{flushleft}
\begin{flushleft}
\begin{lstlisting}
DstComb.addTool( OfflineVertexFitter )                                                                                                  
DstComb.ParticleCombiners.update({"":"OfflineVertexFitter"})                                                                                    
DstComb.OfflineVertexFitter.useResonanceVertex = False
DstComb.OfflineVertexFitter.applyDauMassConstraint = True                                                                                      
DstComb.ReFitPVs = True
\end{lstlisting}
\end{flushleft}
\end{flushleft}
\end{frame}


\begin{frame}
\frametitle{Discarded cuts}
\begin{itemize}
\item{HLT1 TOS requirment}
\begin{itemize}
\item{Doesn't Change the retension as the HLT2 does the same job, and adds to the time/event as spends time looking through the Kst candidates, and only removes a very small amount of these candidates}
\end{itemize}
\item{D0 (VFASPF(VCHI2/VDOF)}
\begin{itemize}
\item{A cut is made on daughter vertecies of the D0 so this extra cut after the creation of a D0 doesn't do anything}
\end{itemize}
\end{itemize}
\end{frame}

%templates

%\begin{frame}
%\frametitle{Signal Yeild}
%\begin{minipage}{3cm}
%\item $L=0.798~{pb}^{-1}$ with $N=4546$ 
%\item[]
%\Rightarrow If   $L=2000~{pb}^{-1}$ then $N=11.3$ Million events
%\end{minipage}
%\begin{minipage}{7.5cm}
%\includegraphics[scale=0.5]{figs/lumi.png}
%\end{minipage}
%\end{frame}

\end{document}
