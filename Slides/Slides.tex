\documentclass[10pt,a4paper]{beamer}

\usepackage{xcolor}
\usepackage{graphicx}
\usepackage{float}
\usepackage{listings}
\usepackage{enumerate}
\usepackage{pgfpages}
\setbeameroption{show notes}
\setbeameroption{show notes on second screen=right}

\usetheme{Madrid}
\author{Kevin Maguire}
\defbeamertemplate*{footline}{shadow theme}
{\leavevmode
 \hbox{\begin{beamercolorbox}[wd=.5\paperwidth,ht=2.5ex,dp=1.125ex,leftskip=.3cm plus1fil,rightskip=.3cm]{author in head/foot}
\usebeamerfont{author in head/foot}\insertframenumber\,/\,\inserttotalframenumber\hfill\insertshortauthor
\end{beamercolorbox}
\begin{beamercolorbox}[wd=.5\paperwidth,ht=2.5ex,dp=1.125ex,leftskip=.3cm,rightskip=.3cm plus1fil]{title in head/foot}
    \usebeamerfont{title in head/foot}\insertshorttitle
\end{beamercolorbox}}
  \vskip0pt
}

\usetheme{Madrid}
\author{Kevin Maguire}
\title{Singular Lorentz Transformations and Pure Radiation Fields}
\date{\today}
\pagenumbering{arabic}

\begin{document}

\setbeamercovered{transparent}

\begin{frame}
\maketitle
\end{frame}

\begin{frame}
\frametitle{Layout}
\begin{enumerate}
\item<1->{Introduction: Lorentz Transformations}
\item<2->{Strange Minkowskian Line Element}
\item<3->{Singular Lorentz transformation}
\end{enumerate}
\end{frame}

%INTRODUCTION

\begin{frame}
\begin{minipage}{6cm}
\frametitle{Introduction: Lorentz Transformations}
\begin{itemize}
\item<1->{A Lorentz transformation is defined by the preservation of the quadratic form $${x'}^2 + {y'}^2 + {z'}^2 - {t'}^2 = x^2 + y^2 + z^2 - t^2,$$ in the transformation $(x,y,z,t) \rightarrow (x',y',z',t')$}
\item<2->{Take the \textcolor{red}{Proper Orthochronous Lorentz Transformations(POLTs)} which form the \textcolor{red}{restricted Lorentz group} $SO^{+}(1,3)$}
\item<3->{In general lorentz transformations have two invariant null directions}
\end{itemize}
\end{minipage}
\begin{minipage}{4.5cm}
\includegraphics<4->[scale=0.4]{../Tex/figs/1_1.jpg}
\end{minipage}
\note{
\begin{itemize}
\item<3->{--Proper is det 1 . preserves the orientation of spacial axes, preserves handedness}
\item<3->{--orthochronous means time is always positive and the direction of time is preserved}
\item<5->{--Think of the standard Lornetz transformation, always two null directions at $x \pm t$}
\end{itemize}
}
\end{frame}

\begin{frame}
\frametitle{Layout}
add in the contents

\note{
\begin{itemize}
\item<1->{derive a strange minkowskian line element}
\item<2->{making a complicated transformation that keeps a single null geodesic fixed look trivial}
\end{itemize}
}
\end{frame}

%STRANGE MINKOWSKIAN LINE ELEMENT

\begin{frame}
\frametitle{Strange Minkowskian Line Element}
\begin{itemize}
\item<1->{Start with the Schwarzschild solution $$\epsilon {\mathrm{d}s}^2 = {\left(1 - \frac{2m}{r}\right)}^{-1} {\mathrm{d}r}^{2} + r^2 ({\mathrm{d}\theta}^2 + {{\sin}^2 \theta}{\mathrm{d} \phi}^2) - \left(1 - \frac{2m}{r}\right) {\mathrm{d}t}^2.$$}
\item<2->{Make the Eddington-Finkelstein coordinate transformation [2] $$u = t-r - 2m \ln(r-2m).$$}
\item<3->{Make further coordinate transformations to obtain $$\epsilon {\mathrm{d}s}^2 = \frac{r^2}{\cosh^{2}{\mu \xi}} ({\mathrm{d}\xi}^2 + {\mathrm{d}\eta}^2) - 2 {\mathrm{d}u}{\mathrm{d}r} - \left( \mu^{2} - \frac{2k}{r} \right) {\mathrm{d}u}^2.$$}
\item<4->{Taking the limit as the energy, $\mu \rightarrow 0$ gives The \textcolor{red}{Kasner Solution} $$\epsilon {\mathrm{d}s}^2 = r^2 ({\mathrm{d}\xi}^2 + {\mathrm{d}\eta}^2) - 2 {\mathrm{d}u}{\mathrm{d}r} - \frac{2k}{r} {\mathrm{d}u}^2.$$}
\end{itemize}
\note{
\begin{itemize}
\item<2->{First we are going to derive a strange form of the Minkowskian line element.. of the vacuum field equations, which will be familiar to most of us}
\item<3->{to remove the coordinate singularity in the Schwarzchild solution}
\item<4->{These transformations put the line element in a form where we can take the limit as the energy goes to 0}
\item<5->{It is easily shown with further coord transforms that this is Kasner, but it wont be done here}
\end{itemize}
}
\end{frame}

\begin{frame}
\frametitle{Strange Minkowskian Line Element}
\begin{itemize}
\item<1->{Then with $m = 0$ the strange Minkowskian line element is obtained $$\epsilon {\mathrm{d}s}^2 = r^2 ({\mathrm{d}\xi}^2 + {\mathrm{d}\eta}^2) - 2 {\mathrm{d}u}{\mathrm{d}r}.$$}
\item<2->{It is easily shown that $r = 0$ gives $$\epsilon {\mathrm{d}s}^2 = 0,$$ and thus is a single null geodesic.}
\end{itemize}
\note{
\begin{itemize}
\item<2->{Its easily shown with suitable coordinate transforms that this is minkowskian line element}
\item<3->{This is best shown by calculating the geodesic equations after the Eddington-Finkelstein coord transforms, all zero if $u$ is proper time along the geodesic}
\end{itemize}
}
\end{frame}

\begin{frame}
\frametitle{Layout}
add in the contents
\note{LTs that leave one null invariant direction are constructed}
\end{frame}

%SINGULAR LORENTZ TRANSFORMATION

\begin{frame}
\frametitle{Singular Lorentz Transformation}
\begin{itemize}
\item<1->{Define an arbitrary complex parameter $\zeta \vcentcolon \xi + i \eta,$ to get the new line element $$\epsilon {ds^2} = r^2 {d\zeta}{d\bar{\zeta}} - 2 {du}{dr}.$$}
\item<2->{The transformation $\zeta \rightarrow \zeta + w$, where $w \in \mathbb{C}$ is then trivial and leave the single null geodesic $r = 0$ invariant.}
\item<3->{Now see what this looks like in Cartesian coordinates}
\end{itemize}


\note{
\begin{itemize}
\item<3->{This is what we want, An LT which leaves one null invariant.}
\item<3->{The use in the previous coord transforms was to make this transformation look trivial}
\end{itemize}
}
\end{frame}









\begin{frame}
\frametitle{temp}

\note{
\begin{itemize}
\item<1->{temp}
\end{itemize}
}
\end{frame}
















\begin{frame}
\frametitle{References}
\begin{itemize}
\item[1]{ J.L. Synge - ``Relativity: The Special Theory'' - North Holland Publishing Company (1965)}
\item[2]{ D. Finkelstein - ``Past-Future Asymmetry of the Gravitational Field of a Point Particle'' - Phys.Rev.Vol 110, (1958) - \url{http://journals.aps.org/pr/pdf/10.1103/PhysRev.110.965}}
\item[3]{ P.A. Hogan, C.Barrab\`es - ``Advanced General Relativity: Gravity Waves, Spinning Particles and Black Holes'' - Oxford University Press (May 2013)}
\item[4]{ I. Robinson ``Spherical Gravitational Waves'' - Phys.Rev.Lett. 4 (1960) 431-432 - \url{http://journals.aps.org/prl/pdf/10.1103/PhysRevLett.4.431}}
\item[5]{ P.A Hogan, C. Barrab\`es - ``Singular Null Hypersurfaces'' - World Scientific Pub Co Inc (April 2004)}
\item[6]{ R. Penrose, W. Rindler - ``Spinors and Space-Time: Volume 1, Two-Spinor Calculus and Relativistic Fields'' - Cambridge University Press, (Feb 1987)}
\item[7]{ Tristan Needham - ``Visual Complex Analysis'' - Clarendon Press, Oxford (1997)}
\item[8]{ Various Authors - ``Space-Time and Geometry: The Alfred Schild Lectures'' - University of Texas Press (March 21, 2012)}
\end{itemize}
\end{frame}

\end{document}
