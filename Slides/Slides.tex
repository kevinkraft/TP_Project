\documentclass[10pt,a4paper]{beamer}

\usepackage{xcolor}
\usepackage{graphicx}
\usepackage{float}
\usepackage{listings}
\usepackage{enumerate}
\usepackage{pgfpages}
\setbeameroption{show notes}
\setbeameroption{show notes on second screen=right}

\usetheme{Madrid}
\author{Kevin Maguire}
\defbeamertemplate*{footline}{shadow theme}
{\leavevmode
 \hbox{\begin{beamercolorbox}[wd=.5\paperwidth,ht=2.5ex,dp=1.125ex,leftskip=.3cm plus1fil,rightskip=.3cm]{author in head/foot}
\usebeamerfont{author in head/foot}\insertframenumber\,/\,\inserttotalframenumber\hfill\insertshortauthor
\end{beamercolorbox}
\begin{beamercolorbox}[wd=.5\paperwidth,ht=2.5ex,dp=1.125ex,leftskip=.3cm,rightskip=.3cm plus1fil]{title in head/foot}
    \usebeamerfont{title in head/foot}\insertshorttitle
\end{beamercolorbox}}
  \vskip0pt
}

\usetheme{Madrid}
\author{Kevin Maguire}
\title{Singular Lorentz Transformations and Pure Radiation Fields}
\date{\today}
\pagenumbering{arabic}

\begin{document}

\setbeamercovered{transparent}

\begin{frame}
\maketitle
\end{frame}

\begin{frame}
\frametitle{Layout}
\begin{enumerate}
\item<1->{Introduction}
\item<2->{Line Element in Minkowskian Space-Time}
\item<0>{Test}
\item<3->{Singular Lorentz transformation}
\end{enumerate}
\end{frame}

\begin{frame}
\begin{minipage}{6cm}
\frametitle{Introduction: Lorentz Transformations}
\begin{itemize}
\item<1->{A Lorentz transformation is defined by the preservation of the quadratic form $${x'}^2 + {y'}^2 + {z'}^2 - {t'}^2 = x^2 + y^2 + z^2 - t^2$$ in the transformation $(x,y,z,t) \rightarrow (x',y',z',t')$}
\item<2->{Take the \textcolor{red}{Proper Orthochronous Lorentz transformations} which form the \textcolor{red}{restricted Lorentz group} $SO^{+}(1,3)$}
\item<3->{In general lorentz transformations have two invarient null directions}
\end{itemize}
\end{minipage}
\begin{minipage}{4.5cm}
\includegraphics<4->[scale=0.4]{../Tex/figs/1_1.jpg}
\note<2->{--Proper is det 1 . preserves the orientation of spacial axes, preserves handedness}
\note<2->{--orthochronous means time is always positive and the direction of time is preserved}
\note<4->{--Think of the standard Lornetz transformation, always two null directions at $x \pm t$}
\end{minipage}
\end{frame}






\end{document}
