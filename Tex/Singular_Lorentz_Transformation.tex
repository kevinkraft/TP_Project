\section{The Singular Lorentz Transformation}

In this section a Lorentz transformation that leaves our line element Eqn.(\ref{Kasner_after_limit_no_k}) invariant is constructed. This transformation is then expressed in terms of $(x,y,z,t)$ and examined to see what form it has. The subgroup of the Lorentz group that it makes is then examined. First we define an arbitrary complex parameter by $\zeta = \xi + i \eta$ so that the differentials are given by

\begin{eqnarray*}
d\zeta = {d\xi} + i {d\eta}, \\
d\bar{\zeta} = {d\bar{\xi}} - i {d\bar{\eta}}, \\
\end{eqnarray*}

\noindent and the line element can be rewritten as

\begin{equation*}
\epsilon {ds^2} = r^2 {d\zeta}{d\bar{\zeta}} - 2 {du}{dr}.
\end{equation*}

\noindent In this form the transformation $\zeta \rightarrow \zeta + w$, where $w \in \mathbb{C}$, is trivial. It leaves the line element unchanged and the null geodesic $r = 0$ trivially invariant. This is a Lorentz transformation which leaves one null direction invariant. Therefore it is a two real parameter, singular Lorentz transformation, where the two parameters come from the complex variable $w$. With this form of the line element the transformation is obviously trivially invariant, we now want to see what this transformation looks like in terms of the usual coordinates $(x,y,z,t)$.

First invert the transformation (\ref{trans_x_to_xi}) and use the new variable $\zeta$

\begin{align} \nonumber
x + iy  & = r (\xi + i \eta) = r \zeta \\\nonumber
z  & = u + \frac{r}{2}(-1 + \zeta \bar{\zeta}) \\\label{Singular_Trans_x,y,z,t_xi,eta,r,u_first}
t &  = u + \frac{r}{2}(1 + \zeta \bar{\zeta})
\end{align}

\noindent From this it is clear that

\begin{align*}
t - z & = r, \\
t + z & = 2 u + r \zeta \bar{\zeta}. 
\end{align*}

\noindent So finally

\begin{align}\nonumber
r & = t - z, \\\nonumber
\zeta & = \frac{x + i y}{t-z}, \\\label{Singular_Trans_x,y,z,t_xi,eta,r,u_second}
u & = \frac{1}{2} (t + z) - \frac{(x^2 + y^2)}{2(t - z)}.
\end{align}

\noindent Now make the desired transformation $(\zeta', \bar{\zeta}', r', u') \rightarrow (\zeta + w, \bar{\zeta} + \bar{w}, r, u)$ by first replacing these new quantities into transformation (\ref{Singular_Trans_x,y,z,t_xi,eta,r,u_first})

\begin{align*}
x' + iy' & = r'\zeta' = r(\zeta + w), \\
z' & = u + \frac{r}{2}(-1 + \zeta \bar{\zeta} + \zeta \bar{w} + \bar{\zeta} w + w \bar{w}), \\
t' & = u + \frac{r}{2} (1 + \zeta \bar{\zeta} + \zeta \bar{w} + \bar{\zeta} w + w \bar{w}). 
\end{align*}

\noindent So the transformed Cartesian coordinates have been written in terms of the untransformed particular coordinates, $(\zeta, r, u)$. Next, using the relations (\ref{Singular_Trans_x,y,z,t_xi,eta,r,u_second}), write the transformed Cartesian coordinates in terms of the untransformed Cartesian coordinates.

\begin{align}
x' + i y' & = x + iy + w(t-z) \label{sing_final_no_prime_1}, \\
z' - t' & = -r = z - t \label{sing_final_no_prime_2}, \\
z' + t' & = z+t + w(x - i y) + w(x + iy) + w\bar{w} (t-z) \label{sing_final_no_prime_3}.
\end{align}

It is necessary to show that this is indeed a Lorentz transformation by verifying the usual Lorentz invariant quadratic form. First Eqn.(\ref{sing_final_no_prime_1}) implies

\begin{align*}
{x'}^2 + {y'}^2 &  = (x + iy + w(t-z))(x - iy + \bar{w}(t-z)) \\
& = x^2 + y^2 + \bar{w}(t - z)(x+iy) + w(t-z)(x-iy) + w\bar{w}{(t-z)}^2.
\end{align*}

\noindent Then Eqn.(\ref{sing_final_no_prime_2}) and Eqn.(\ref{sing_final_no_prime_3}) imply

\begin{align*}
{z'}^2 - {t'}^2 & = (z' + t')(z' - t') \\
& = z^2 - t^2 + (z - t)w(x-iy) + (z-t)\bar{w}(x+iy) + (z -t)w\bar{w}(t-z)
\end{align*}

\noindent Thus the extra terms cancel and Lorentz invariant quadratic form in the primed frame is the same as that of the unprimed frame,

\begin{equation*} 
{x'}^2 + {y'}^2 + {z'}^2 - {t'}^2 = {x}^2 + {y}^2 + {z}^2 - {t}^2.
\end{equation*} 

\noindent It is also clear from Eqn.(\ref{sing_final_no_prime_2}) that the null direction $z = t$ is invariant under this Lorentz transformation.

In conclusion, this transformation involves one complex parameter and thus two real parameters. In the usual Cartesian coordinates it is described by Eqns.(\ref{sing_final_no_prime_1}) - (\ref{sing_final_no_prime_3}) and in the coordinates $(\xi, \eta, r, u)$, also denoted by $(\zeta, r,u)$ derived in previous sections, it is expressed simply as

\begin{align*}
\zeta' & = \zeta + w, \\
r' & = r, \\
u' & = u.
\end{align*}

Note that the operation of addition of complex numbers is commutative so that if $\zeta' = \zeta + w_1$ and $\zeta'' = \zeta' + w_2$ then

\begin{equation*} 
\zeta'' = \zeta + w_1 + w_2 = \zeta + w_3.
\end{equation*} 
 
\noindent Thus these transformations form a 2-parameter abelian subgroup of the Lorentz group with the binary operation of addition of complex numbers. (ERROR. SHOW ALL 2 PARAM ABELIAN SUBGROUPS OF THE LORENTZ GROUP ARE SINGULAR TRANSFORMATIONS???)

So a Lorentz transformation that preserves the line element Eqn.(\ref{Kasner_after_limit_no_k}) has been constructed. It is found that this transformation is a singular Lorentz transformation as it keeps the null direction $r=0$ fixed. Thus it is shown that all two parameter abelian subgroups of the Lorentz group are singular Lorentz transformations. (ERROR. HAVE WE DONE ENOUGH TO SAY THIS)






