\section{The Singular Lorentz Transformation}

In this section a Lorentz transformation that leaves our line element Eqn.(\ref{Kasner_after_limit_no_k}) invariant is constructed. This transformation is then expressed in terms of $(x,y,z,t)$ and examined to see what for it has. First we define an arbitrary complex parameter by $\zeta = \xi + i \eta$ so that the differentials are given by:

\begin{eqnarray*}
d\zeta = {d\xi} + i {d\eta} \\
d\bar{\zeta} = {d\bar{\xi}} + i {d\bar{\eta}} \\
\end{eqnarray*}

\noindent and the line element can be rewritten as:

\begin{equation*}
\epsilon {ds^2} = r^2 {d\zeta}{d\bar{\zeta}} - 2 {du}{dr}
\end{equation*}

\noindent In this form the transformation $\zeta \rightarrow \zeta + w$, where $w \in \mathbb{C}$, is trivial. It leaves the line element unchanged and the null geodesic $r = 0$ trivially invariant. This is a Lorentz transformation which leaves one null direction invariant. Therefore it is a two real parameter, singular Lorentz transformation, where the two parameters come from the complex variable $w$. With this form of the line element the transformation is obviously trivial, we now want to see what this transformation looks like in terms of the usual variables $(x,y,z,t)$.

First we invert the transformation (\ref{trans_x_to_xi_4}) and use the new variable $\zeta$:

\begin{eqnarray*} 
x + iy = r (\xi + i \eta) = r \zeta \\
z = u + \frac{r}{2}(-1 + \zeta \bar{\zeta}) \\
t = u + \frac{r}{2}(1 + \zeta \bar{\zeta})
\end{eqnarray*}

\noindent From this it is clear that:

\begin{eqnarray*}
t - z = r \text{, and } \\
t + z = 2 u + r \zeta \bar{\zeta} 
\end{eqnarray*}

\noindent So finally:

\begin{eqnarray*}
\zeta = \frac{x + i y}{t-z} \\
r = t - z \\
u = \frac{1}{2} (t + z) - \frac{(x^2 + y^2)}{2(t - z)}
\end{eqnarray*}

Now make the desired transformation:

\begin{eqnarray*}
\zeta' \rightarrow \zeta + w \\
\bar{\zeta}' \rightarrow \bar{\zeta} + \bar{w} \\
r' = r \\
u' = u
\end{eqnarray*}

(SEE NOTES PG 2:6 FOR CALC)

\noindent To obtain:

\begin{eqnarray}
x' + i y' = x + iy + w(t-z) \label{sing_final_no_prime_1} \\
z' - t' = -r = z - t \label{sing_final_no_prime_2} \\
z' + t' = z+t + w(x - i y) + w(x + iy) + w\bar{w} (t-z) \label{sing_final_no_prime_3}
\end{eqnarray}

Next, it is necessary to show that this is indeed a Lorentz transformation by verifying the usual Lorentz invariant quantity. First from Eqn.(\ref{sing_final_no_prime_1}) implies:

\begin{eqnarray*}
{x'}^2 + {y'}^2 = (x + iy + w(t-z))(x - iy + \bar{w}(t-z)) \\
= x^2 + y^2 + \bar{w}(t - z)(x+iy) + w(t-z)(x-iy) + w\bar{w}{(t-z)}^2
\end{eqnarray*}

Then Eqn.(\ref{sing_final_no_prime_2}) and Eqn.(\ref{sing_final_no_prime_3}) imply:

\begin{eqnarray*}
(z' + t')(z' - t') = {z'}^2 - {t'}^2 \\
= z^2 - t^2 + (z - t)w(x-iy) + (z-t)\bar{w}(x+iy) + (z -t)w\bar{w}(t-z)
\end{eqnarray*}

\noindent Thus the Lorentz invariant quantity in the primed frame is the same as that of the unprimed frame:

\begin{equation*} 
{x'}^2 + {y'}^2 + {z'}^2 - {t'}^2 = {x}^2 + {y}^2 + {z}^2 - {t}^2
\end{equation*} 

\noindent It is also clear from Eqn.(\ref{sing_final_no_prime_2}) that the null direction $z = t$ is invariant under this Lorentz transformation.

In conclusion, this transformation involves one complex parameter and thus two real parameters. In the usual Cartesian coordinates it is described by Eqns.(\ref{sing_final_no_prime_1}) - (\ref{sing_final_no_prime_1}) and in the coordinates $(\xi, \eta, r, u)$ (SHOULD THIS NOT BE WITH A $\zeta$ ???) derived in previous sections, it is expressed simply as:

\begin{eqnarray*}
\zeta' = \zeta + w \\
r' = r \\
u' = u
\end{eqnarray*}

Note that the operation of addition of complex numbers is commutative so that:

\begin{eqnarray*} 
\zeta' = \zeta + w_1 \text{, and } \zeta'' = \zeta' + w_2 \\
\Rightarrow \zeta'' = \zeta + w_1 + w_2
\end{eqnarray*} 
 
\noindent and thus these transformations form a 2-parameter abelian subgroup of the Lorentz group with the binary operation of addition of complex numbers. (SHOW ALL 2 PARAM ABELIAN SUBGROUPS OF THE LORENTZ GROUP ARE SINGULAR TRANSFORMATIONS???)






