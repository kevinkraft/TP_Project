\documentclass[floatfix,aps,prd,amsmath,amssymb]{revtex4}

\usepackage{epsfig}
\usepackage{color}
\usepackage{graphicx}
\usepackage{float}
\usepackage{listing} %listings
\usepackage{braket}
\usepackage{hyperref}
\usepackage[all]{hypcap}
\usepackage{mathtools}
\usepackage[noabbrev,capitalise]{cleveref} %for \cref in CKM-Mechanism.tex 
\providecommand{\e}[1]{\ensuremath{\times 10^{#1}}} %because my scientific notation wouldn't work: Kevin
\usepackage{tensor}
\usepackage{amsmath}

\begin{document}
\title{Singular Lorentz Transformations and Pure Radiation Fields}
\author{Kevin Maguire (10318135)}
\date{\today}

\begin{abstract}
\textit{This report deals with those proper orthochronous Lorentz transformations (POLTs) that have only one fixed null direction, known as singular Lorentz transformations. It will be shown that in general, Lorentz transformations have two fixed null directions, the special case where these coincide is considered. Starting with the Schwarzschild solution of the vacuum field equations, a novel form of Minkowskian space-time will be derived. It will be shown that a seemingly trivial transformation of this strange Minkowskian space-time is a singular Lorentz transformation, which with further inspection will imply that all singular Lorentz transformations are $2$-parameter abelian subgroups of the Lorentz group. A connection between $2 \times 2$ Hermitian matrices and Minkowskian space-time will be established which gives a two to one correspondence between $SL(2,\mathbb{C})$ matrices and POLTs, with numerous examples. Then a coordinate transformation between Minkowskian space-time and the extended complex plane will be constructed through stereographic projection. It will be shown that a POLT in these coordinates gives the so called fractional linear, or Mobius transformation and thus there is a one to one correspondence between POLTs and fractional linear transformations. Certain conditions will then be applied to this transformation to select only those which give singular Lorentz transformations. These results will then be applied to infinitesimal Lorentz transformations to find that they have the same form as the Lorentz force. By considering the singular infinitesimal Lorentz transformations it will be shown that if the world line of a charged particle is generated by such a transformation then the particle is moving in a pure radiation electromagnetic field. The form of the pure radiation conditions will be written in terms of Minkowskian space-time coordinates to find that the null direction in Minkowskian space-time is the propagation direction of the radiation and that the electric and magnetic fields and the propagation direction in $\mathbb{R}^2$ form an orthonormal triad. The importance of the Lorentz transformation in physics is most evident in special relativity and particle physics. As will be discussed here the irreducible pieces of the Lorentz group are representations of physical particles and thus are essential for any understanding of the properties of particles and how they interact with each other.}
\end{abstract}

\maketitle 
\pagenumbering{roman}

%\begin{figure}[h!]
%\begin{center}
\includegraphics[scale=0.8]{figs/Cover.jpg}
%\end{center}
%\caption{\textit{}}
%\label{}
%\end{figure}

\newpage

\tableofcontents

\newpage

\pagenumbering{arabic}

\section{Introduction}

All Lorentz transformations have two inherent null directions, except the singular Lorentz transformations \cite[p. 85]{Relativity_Synge}. The Lorentz Transform is defined by $(x,y,z,t) \rightarrow (x',y',z',t')$ such that

\begin{equation*}
{x'}^2 + {y'}^2 + {z'}^2 - {t'}^2 = x^2 + y^2 + z^2 - t^2.
\end{equation*}

\noindent If the transformation preserves the orientation of the spatial axes then is it called a \textit{proper} Lorentz transformation, this is equivalent to taking the subgroup of the Lorentz group of transformations with determinant $1$. This is also understood as the preservation of the handedness of the axes. If determinant $-1$ was chosen a reflection in some spacial axis, $x^j = -{x'}^j$ would also give a proper Lorentz transformation. If $t \geq 0$ implies that time is always positive then it is called an \textit{orthochronous} Lorentz transformation, which ensures that the time direction is preserved. In this project the ``Lorentz transformation'' will refer to the proper, orthochronous Lorentz transformation.

\begin{figure}[h!]
\begin{center}
\caption{\textit{Space-Time Diagram for a photon moving at the speed of light, $c=1$ and starting from $x = 0$. Two null directions can be seen.}}
\label{figure_Photon_Space_Time}
\includegraphics[scale=0.8]{figs/1_1.jpg}
\end{center}
\end{figure}

Consider a photon moving in the $x$ direction at the speed of light, $c = 1$, and starting at $x = 0$. The space-time for such a photon can be illustrated as in Fig.(\ref{figure_Photon_Space_Time}). It is clear that there are two null directions in this space-time, $x = \pm t$. To see this use the standard Lorentz transformation

\begin{align*}
x'  = \gamma (x - vt),  \\
t'  = \gamma (t - vx),
\end{align*}

\noindent where $\gamma = {(1 - v^2)}^{-1/2}$. Rearrange to obtain

\begin{eqnarray*}
x' - t' = \gamma (1 + v) (x - t), \\
x' + t' = \gamma (1 - v) (x + t).
\end{eqnarray*}

\noindent It is clear that $x = \pm t$ implies $x' = \pm t'$. Thus there are two null directions in this space-time at $x = \pm t$, as null directions are by definition invariant under a Lorentz transformation. It can be shown that all Lorentz transformations have two invariant null directions except the singular Lorentz transformation which has only one fixed null direction.  


\section{Reparameterisation of the Schwarzschild Solution}

In this section the Kasner solution of the vacuum field equations is derived from the Schwarzschild solution by taking the limit as the mass goes to infinity. It is then shown that the special case of the Kasner solution with no mass is equivalent to a novel form of Minkowskian space-time. (what do we want the Kasner vacuum solution for???). Start with the Schwarzschild Solution of the vacuum field equations given by

\begin{equation}\label{Schwarz_Sol} 
\epsilon {\mathrm{d}s}^2 = {\left(1 - \frac{2m}{r}\right)}^{-1} {\mathrm{d}r}^{2} + r^2 ({\mathrm{d}\theta}^2 + {{\sin}^2 \theta}{\mathrm{d} \phi}^2) - \left(1 - \frac{2m}{r}\right) {\mathrm{d}t}^2.
\end{equation}

\subsection{Eddington-Finkelstein Coordinate Transformation}

\noindent First, make the Eddington-Finkelstein coordinate transformation

\begin{equation}\label{Ed-Fin_trans}
u = t - r - 2m \ln(r - 2m).
\end{equation}

\noindent Calculate the differentials

\begin{align*}
\mathrm{d}u & = \mathrm{d}t - \mathrm{d}r - \frac{2m \mathrm{d}r}{r - 2m},\\
            & = \mathrm{d}t - \mathrm{d}r{\left( 1-\frac{2m}{r}  \right)}^{-1},\\
\mathrm{d}t & = \mathrm{d}u + {\left( 1-\frac{2m}{r}  \right)}^{-1} \mathrm{d}r, 
\end{align*}

\noindent and sub them into Eqn.(\ref{Schwarz_Sol}).

\begin{equation*}
\epsilon {\mathrm{d}s}^{2} = {\left( 1-\frac{2m}{r}  \right)}^{-1} \mathrm{d}r^2 + r^2 ({\mathrm{d}\theta}^2 + {{\sin}^2 \theta}{\mathrm{d} \phi}^2) - {\left( 1-\frac{2m}{r}  \right)} {\left( \mathrm{d}u + {\left( 1-\frac{2m}{r}  \right)}^{-1} \mathrm{d}r \right)}^{2},
\end{equation*}

\noindent which gives the result,

\begin{equation}\label{Reparameterisation_Schwarzschild_Before_Limit}
\epsilon {\mathrm{d}s}^{2} = r^2 ({\mathrm{d}\theta}^2 + {{\sin}^2 \theta}{\mathrm{d} \phi}^2) - 2 \mathrm{d}u \mathrm{d}r - {\mathrm{d}u}^{2} + \frac{2m}{r} {\mathrm{d}u}^{2}. 
\end{equation}

Note that if $m = 0$ the space-time becomes Minkowskian, as expected. If $r = 0$ in this Minkowskian space-time the line element becomes

$$ \epsilon {\mathrm{d}s}^2 = - {\mathrm{d}u}^{2},$$

\noindent which implies that $\epsilon = -1$ and the first integral of this trajectory is also equal to $-1$. Thus $r = 0$ is a time-like world-line in Minkowskian space-time, with proper time u.(CHECK ITS A GEODESIC???NEED TO CHECK ANSWER WITH HOGAN)

The limit of the Schwarzschild solution as $m \rightarrow \infty$ must be calculated to find the Kasner solution. In its current form the limit cannot be taken, so two suitable coordinate transformations must be made to get it in a more useful form. First Set 

\begin{gather*} 
u = \mu u', \\
r = {\mu}^{-1} r',   
\end{gather*} 

\noindent where $\mu = \text{ const}$ so that
  
\begin{gather*} 
\mathrm{d}u = \mu \mathrm{d}u', \\
\mathrm{d}r = {\mu}^{-1} \mathrm{d}r'. \\
\end{gather*} 

\noindent The product of the differentials is then invariant

\begin{equation*}
{\mathrm{d}u}{\mathrm{d}r} = {\mathrm{d}u'}{\mathrm{d}r'}. 
\end{equation*}

\noindent When the new coordinates are subbed into Eqn.(\ref{Reparameterisation_Schwarzschild_Before_Limit}) the Schwarzschild solution becomes

\begin{equation}\label{Reparameterise_Schwarzschild_Next_Before_Limit}
\epsilon {\mathrm{d}s}^2 = {r'}^2 \sin^2 \theta \left\{ \frac{{\mathrm{d}\theta}^2}{\mu^2 \sin^2 \theta} +  \mu^{-2} {\mathrm{d} \phi}^2 \right\} - 2 {\mathrm{d}u'} {\mathrm{d}r'} - \left( \mu^2 - \frac{2 m \mu^3}{r'}\right) {\mathrm{d}u'}^2 . 
\end{equation}

\noindent Now set $m \mu^{3} = k$ or $ m = k \mu^{-3}$ and make another transformation first done by Ivor Robinson(check this???) given by

\begin{align*} 
\sin{\theta} = \frac{1}{\cosh{(\mu \xi)}}, & \mu^{-1} \phi = \eta.  
\end{align*} 

\noindent The second of these transformations gives simply $\mu^{-2} \mathrm{d}\phi^2 = \mathrm{d} \eta^2$. To rewrite the first coordinate transformation, first differentiate.

\begin{align*} 
\cos{\theta} \mathrm{d} \theta & = \frac{-1}{(\cosh(\mu \xi))^{2}} \sinh(\mu \xi) \mu \mathrm{d} \xi, \\
                      & = -{\sin}^{2}\theta \sinh(\mu \xi) \mu \mathrm{d} \xi.
\end{align*}

\noindent Use the forumla $\cosh^2 A - \sinh^2 A = 1$, divide by $\mathrm{d} \xi$ and simplfy using trigonometric identities

\begin{align*}
\cos{\theta} \frac{\mathrm{d} \theta}{\mathrm{d} \xi} & = -\mu \sin^2 \theta \sqrt{\frac{1}{\sin^2 \theta} - 1}, \\
                                    & = - \mu \sin \theta \cos \theta.
\end{align*}

\noindent Finally rewrite in terms of $\mathrm{d} \xi$

\begin{equation*}
\mathrm{d} \xi^2 = {\left( \frac{\mathrm{d} \theta}{\mu \sin \theta}  \right)}^2.
\end{equation*}

\noindent Subbing these transformations into Eqn.(\ref{Reparameterise_Schwarzschild_Next_Before_Limit}) gives

\begin{equation*}
\epsilon {\mathrm{d}s}^2 = \frac{r^2}{\cosh^{2}{\mu \xi}} ({\mathrm{d}\xi}^2 + {\mathrm{d}\eta}^2) - 2 {\mathrm{d}u}{\mathrm{d}r} - \left( \mu^{2} - \frac{2k}{r} \right) {\mathrm{d}u}^2,
\end{equation*}

\noindent where the primes have been dropped for convenience. 

\subsection{The Kasner Solution}

This is now in an appropriate form to take the limit $m \rightarrow \infty$ which is equivalent to $\mu \rightarrow 0$, to obtain

\begin{equation}\label{Kasner_after_limit}
\epsilon {\mathrm{d}s}^2 = r^2 ({\mathrm{d}\xi}^2 + {\mathrm{d}\eta}^2) - 2 {\mathrm{d}u}{\mathrm{d}r} - \frac{2k}{r} {\mathrm{d}u}^2
\end{equation}

\noindent This is still a solution of the field equations but it is no longer the Schwarzschild solution. In this section it is shown to be the Kasner Solution(READ UP ABOUT THIS), which by definition is given by

\begin{equation*} 
\epsilon {\mathrm{d}s}^2 = T^{2p} {\mathrm{d}X}^2 + T^{2q} \mathrm{d}Y^2 + T^{2r} \mathrm{d}Z^2 - \mathrm{d}T^2,
\end{equation*}

\noindent With:

\begin{eqnarray*}
p + q + r = 1 = p^2 + q^2 + r^2.
\end{eqnarray*}

To write Eqn.(\ref{Kasner_after_limit}) in this form first make the transformation

\begin{align*} 
\xi' = \lambda^{-1} \xi, & \eta' = \lambda^{-1} \eta, \\
r' = \lambda r,          & u' = \lambda^{-1} u,
\end{align*}

\noindent with $\lambda \vcentcolon = k^{-1/3}$. Subbing in these new coordinates gives

\begin{equation*}  
\epsilon \mathrm{d} s^2 = {r'}^2 (\mathrm{d} {\xi'}^2 + \mathrm{d} {\eta'}^2) - 2 \mathrm{d} u' \mathrm{d} r' + \frac{2}{r'}\mathrm{d} {u'}^2.
\end{equation*}

\noindent Now add and subtract $(r'/2) \mathrm{d} {r'}^2$ to complete the square as follows

\begin{equation*}  
\epsilon \mathrm{d} s^2 = r^2 (\mathrm{d} \xi^2 + \mathrm{d} \eta^2) \frac{2}{r}{\left( \mathrm{d} u  - \frac{r}{2} \mathrm{d} r\right)}^2 - \frac{r}{2}\mathrm{d} r^2,
\end{equation*}

\noindent where the primes have again been dropped for convenience. Now set

\begin{align*}
\bar{X} = u - \frac{r^2}{4}, & T = \frac{\sqrt{2}}{3} r^{3/2},
\end{align*}

\noindent so the differential of $\bar{X}$ is

\begin{equation*}
\mathrm{d} \bar{X} = \mathrm{d} u - \frac{r}{2} \mathrm{d}r
\end{equation*}

(ERROR HAVE CHECK TO HERE)

\begin{equation}\label{Our_Kasner} 
\epsilon {\mathrm{d}s}^2 = T^{-2/3} {\mathrm{d}X}^2 + T^{4/3} \left( \mathrm{d}Y^2 + \mathrm{d}Z^2 \right) - \mathrm{d}T^2
\end{equation}

\noindent So it is clear that Eqn.(\ref{Our_Kasner}) is the Kasner solution with $p = -1/3$ and $q = r = 2/3$, and thus Eqn.(\ref{Kasner_after_limit}) is also the Kasner solution

\subsection{Line Element of Minkowskian Space-Time}

Minkowskian space-time reemerges again by setting $k = 0$ in Eqn.(\ref{Kasner_after_limit}), which is equivalent to $m = 0$. 

\begin{equation}\label{Kasner_after_limit_no_k}
\epsilon {\mathrm{d}s}^2 = r^2 ({\mathrm{d}\xi}^2 + {\mathrm{d}\eta}^2) - 2 {\mathrm{d}u}{\mathrm{d}r}
\end{equation}

Setting $r = 0$ it can be shown that $\epsilon {\mathrm{d}s}^2 = 0$ in this case. Then $r = 0$ is a null geodesic with $u$ an affine parameter along it. To demonstrate these properties first let $x^i = (x,y,z,t)$ be rectangular Cartesian coordinates with time in Minkowskian space-time with the usual line element:

\begin{equation*} 
\epsilon {\mathrm{d}s_0}^2 = {\mathrm{d}x}^2 + {\mathrm{d}y}^2 + {\mathrm{d}z}^2 - {\mathrm{d}t}^2
\end{equation*} 

\noindent We note that $C: x = 0, y = 0, z = t$ is a null geodesic as it will lie in the light cone of Minkowskian space-time. it we write it parametrically as $x^i = w^i (u)$ such that $w^i = (0,0,u,u)$ then $u$ is an affine parameter along $C$. The tangent to $C$ is then computed as:

\begin{equation*} 
v^i (u) = \frac{\mathrm{d} w^i}{\mathrm{d}u} = (0,0,1,1)
\end{equation*} 
   
\noindent As $C$ is a null geodesic the first integral will be $v_i v^i = 0$ (IS THIS CALLED THE FIRST INTEGRAL?) and thus $v_i = (0,0,1,-1)$ where we have chosen the convention $(+,+,+,-)$. 

The position vector of a point in Minkowskian space time can be written in the form(DO THE PICTURE FROM 2:1):

\begin{eqnarray*}
x^i = w^i (u) = r k^i \\
\text{or } x^i = w^i(u) + r k^i 
\end{eqnarray*}

\noindent Thus $r$ is a new parameter which tell us the shortest distance between $C$ and some point $x^i$, and $k^i$ is the unit vector in that direction. As $k^i$ is a unit vector it satisfies the relations:

\begin{eqnarray}
k^i k_i = 0 \label{k_rel_1}\\
k^i v_i = -1 \label{k_rel_2}
\end{eqnarray}

\noindent Thus $k^i$ is normalized so that $k^i$ and $v^i$ are both future pointing (HOW DOES THIS MAKE THEM BOTH FUTURE POINTING?). Making the parameterisation:

\begin{eqnarray*}
k^i = (\xi, \eta, A, B) \\
\Rightarrow k_i = (\xi, \eta, A, -B)
\end{eqnarray*}

\noindent We can choose any variable for the first two slots of $k^i$ so we choose $\xi$ and $\eta$ from before for convenience. Using the relation (\ref{k_rel_1}) it is clear that:

\begin{equation*}
\xi^2 + \eta^2 + A^2 - B^2 = 0
\end{equation*}

\noindent and using the relation (\ref{k_rel_2}) it is found that:

\begin{eqnarray}
A - B = -1 \label{sim_rel_1}\\
\Rightarrow A^2 - B^2 = (A + B)(A - B) = - (A + B)
\end{eqnarray}


\noindent Which implies:

\begin{equation}\label{sim_rel_2}
\xi^2 + \eta^2 = A + B 
\end{equation}

\noindent So expressions for $A$ and $B$ are found using Eqn.(\ref{sim_rel_1}) and Eqn.(\ref{sim_rel_2}):

\begin{eqnarray*}
A = \frac{1}{2} (-1 + \xi^2 + \eta^2) \\
B = \frac{1}{2} (1 + \xi^2 + \eta^2) \\
\end{eqnarray*}

In summary so far we have:

\begin{eqnarray}
x^i = w^i (u) + r k^i \label{rel_for_trans_1}\\
w^i = (0,0, u,u) \label{rel_for_trans_2}\\
k^i = (\xi, \eta, \frac{1}{2} (-1 + \xi^2 + \eta^2), \frac{1}{2} (1 + \xi^2 + \eta^2)) \label{rel_for_trans_3}\\
x^i = (x, y, z, t) \label{rel_for_trans_4}  
\end{eqnarray}

Consider Eqn.(\ref{rel_for_trans_1}) as a coordinate transformation from $(x,y,z,t)$ to $(\xi,\eta, r, u )$ such that:

\begin{eqnarray}
x = r \xi \nonumber \\
y = r \eta \nonumber \\
z = u + \frac{r}{2} (-1 + \xi^2 + \eta^2) \nonumber \\
t = u + \frac{r}{2} (1 + \xi^2 + \eta^2)  \label{trans_x_to_xi} 
\end{eqnarray} 

\noindent as is clear from Eqn.(\ref{rel_for_trans_1}) - Eqn.(\ref{rel_for_trans_4}). Now this is applied to the Minkowskian line element Eqn.(\ref{Kasner_after_limit_no_k}). First, the $x$ and $y$ differentials are:

\begin{eqnarray*} 
dx = r \mathrm{d}\xi + \xi \mathrm{d}r \\
\mathrm{d}y = r \mathrm{d}\eta + \eta \mathrm{d}r 
\end{eqnarray*} 

\noindent Which gives:

\begin{equation}\label{differentials_1}
{\mathrm{d}x}^2 + {\mathrm{d}y}^2 = r^2 ({\mathrm{d}\xi}^2 + {\mathrm{d}\eta}^2) + 2 r \xi {\mathrm{d}\xi} {\mathrm{d}r} + 2 r \eta {\mathrm{d}\eta}{\mathrm{d}r} + (\xi^2 + \eta^2) {\mathrm{d}r}^2
\end{equation}

\noindent Next, the $z$ and $t$ differentials:

\begin{eqnarray*}
z + t = 2 u + r (\xi^2 + \eta^2) \\
z - t = - r \\
{\mathrm{d}z} + {\mathrm{d}t} = 2 \mathrm{d}u + (\xi^2 + \eta^2) \mathrm{d}r + 2 r \xi {\mathrm{d}\xi} + 2 r \eta {\mathrm{d}\eta} \\
{\mathrm{d}z} - {\mathrm{d}t} = - \mathrm{d}r 
\end{eqnarray*}

\noindent Using difference of two squares to obtain:

\begin{equation}\label{differentials_2}
{\mathrm{d}z}^2 - {\mathrm{d}t}^2 = -2 {\mathrm{d}u}{\mathrm{d}r} - (\xi^2 + \eta^2) {\mathrm{d}r}^2 - 2 r \xi {\mathrm{d}\xi}{\mathrm{d}r} - 2 r \eta {\mathrm{d}\eta}{\mathrm{d}r}
\end{equation}

\noindent Combining Eqn.(\ref{differentials_1}) and (\ref{differentials_2}) to get:

\begin{equation*}
{\mathrm{d}x}^2 + {\mathrm{d}y}^2 + {\mathrm{d}z}^2 - {\mathrm{d}t}^2 = r^2 ({\mathrm{d}\xi}^2 + {\mathrm{d}\eta}^2) - 2 {\mathrm{d}u}{\mathrm{d}r}
\end{equation*}

\noindent and from this it is clear that Eqn.(\ref{Kasner_after_limit_no_k}) is the line element of Minkowskian space-time with $r = 0$ a null geodesic with affine parameter $u$ along it as before. 


\section{The Singular Lorentz Transformation}

In this section a Lorentz transformation that leaves our line element Eqn.(\ref{Kasner_after_limit_no_k}) invarient is constructed. This transformation is then expressed in terms of $(x,y,z,t)$ and examined to see what for it has. First we define an arbitrary complex parameter by $\zeta = \xi + i \eta$ so that the differentials are given by:

\begin{eqnarray*}
d\zeta = {d\xi} + i {d\eta} \\
d\bar{\zeta} = {d\bar{\xi}} + i {d\bar{\eta}} \\
\end{eqnarray*}

\noindent and the line element can be rewritten as:

\begin{equation*}
\epsilon {ds^2} = r^2 {d\zeta}{d\bar{\zeta}} - 2 {du}{dr}
\end{equation*}

\noindent In this form the transforamtion $\zeta \rightarrow \zeta + w$, where $w \in \mathbb{C}$, is trivial. It leaves the line element unchanged and the null geodesic $r = 0$ trivially invariant. This is a lorentz transformation which leaves one null direction invarient. Therefore it is a two real parameter, singular lorentz transformation, where the two parameters come from the complex variable $w$. With this form of the line element the transformation is obviously trivial, we now want to see what this transformation looks like in terms of the usual variables $(x,y,z,t)$.

First we invert the transformation (\ref{trans_x_to_xi_4}) and use the new variable $\zeta$:

\begin{eqnarray*} 
x + iy = r (\xi + i \eta) = r \zeta \\
z = u + \frac{r}{2}(-1 + \zeta \bar{\zeta}) \\
t = u + \frac{r}{2}(1 + \zeta \bar{\zeta})
\end{eqnarray*}

\noindent From this it is clear that:

\begin{eqnarray*}
t - z = r \text{, and } \\
t + z = 2 u + r \zeta \bar{\zeta} 
\end{eqnarray*}

\noindent So finally:

\begin{eqnarray*}
\zeta = \frac{x + i y}{t-z} \\
r = t - z \\
u = \frac{1}{2} (t + z) - \frac{(x^2 + y^2)}{2(t - z)}
\end{eqnarray*}

Now make the desired transformation:

\begin{eqnarray*}
\zeta' \rightarrow \zeta + w \\
\bar{\zeta}' \rightarrow \bar{\zeta} + \bar{w} \\
r' = r \\
u' = u
\end{eqnarray*}

(SEE NOTES PG 2:6 FOR CALC)

\noindent To obtain:

\begin{eqnarray}
x' + i y' = x + iy + w(t-z) \label{sing_final_no_prime_1} \\
z' - t' = -r = z - t \label{sing_final_no_prime_2} \\
z' + t' = z+t + w(x - i y) + w(x + iy) + w\bar{w} (t-z) \label{sing_final_no_prime_3}
\end{eqnarray}

Next, it is necessary to show that this is indeed a Lorentx transformation by verifying the usual Lorentz invarient quantity. First from Eqn.(\ref{sing_final_no_prime_1}) implies:

\begin{eqnarray*}
{x'}^2 + {y'}^2 = (x + iy + w(t-z))(x - iy + \bar{w}(t-z)) \\
= x^2 + y^2 + \bar{w}(t - z)(x+iy) + w(t-z)(x-iy) + w\bar{w}{(t-z)}^2
\end{eqnarray*}

Then Eqn.(\ref{sing_final_no_prime_2}) and Eqn.(\ref{sing_final_no_prime_3}) imply:

\begin{eqnarray*}
(z' + t')(z' - t') = {z'}^2 - {t'}^2 \\
= z^2 - t^2 + (z - t)w(x-iy) + (z-t)\bar{w}(x+iy) + (z -t)w\bar{w}(t-z)
\end{eqnarray*}

\noindent Thus the Lorentz invarient quantity in the primed frame is the same as that of the unprimed frame:

\begin{equation*} 
{x'}^2 + {y'}^2 + {z'}^2 - {t'}^2 = {x}^2 + {y}^2 + {z}^2 - {t}^2
\end{equation*} 

\noindent It is also clear from Eqn.(\ref{sing_final_no_prime_2}) that the null direction $z = t$ is invarient under this Lorentz transformation.

In conclusion, this transformation involves one complex parameter and thus two real parameters. In the usual Cartesian coordinates it is described by Eqns.(\ref{sing_final_no_prime_1}) - (\ref{sing_final_no_prime_1}) and in the coordinates $(\xi, \eta, r, u)$ (SHOULD THIS NOT BE WITH A $\zeta$ ???) derived in previous sections, it is expressed simply as:

\begin{eqnarray*}
\zeta' = \zeta + w \\
r' = r \\
u' = u
\end{eqnarray*}

Note that the operation of addition of complex numbers is commutative so that:

\begin{eqnarray*} 
\zeta' = \zeta + w_1 \text{, and } \zeta'' = \zeta' + w_2 \\
\Rightarrow \zeta'' = \zeta + w_1 + w_2
\end{eqnarray*} 
 
\noindent and thus these transformations form a 2-parameter abelian subgroup of the Lorentz group with the binary operation of addition of complex numbers. (SHOW ALL 2 PARAM ABELIAN SUBGROUPS OF THE LORENTZ GROUP ARE SINGULAR TRANSFORMATIONS???)








\section{Special Linear Matrices of the Lorentz Transformation}\label{Special_Linear_Matrices_of_Lorentz}

Let $\vec{x} = (x,y,z,t)$ be the position vector of a point in Minkowskian space-time. Knowing $\vec{x}$ we can construct the following $2 \times 2$ Hermitian matrix:

\begin{equation}\label{Special_Matrices_A_first}
A = 
\left( 
\begin{array}{cc}
t-z    & x + i y \\
x - iy & t+z \\
\end{array} 
\right)  
\end{equation}

\noindent with $A^{\dagger}(\vec{x}) = A(\vec{x})$. This is useful as its determinant is the same as the usual Lorentz invariant quantity(WHAT DO WE CALL THIS???):

\begin{equation*}
\det(A(\vec{x})) = t^2 - x^2 - y^2 - z^2
\end{equation*}

Consider any $2 \times 2$ Hermitian matrix H. 

\begin{equation*}
H = \left( \begin{array}{cc}
p & q \\
r & s \\
\end{array} \right) \text{ ,     }
H^{\dagger} = \left( \begin{array}{cc}
\bar{p} & \bar{q} \\
\bar{r} & \bar{s} \\
\end{array} \right)
\end{equation*}

\noindent It is know that $H^{\dagger}(\vec{x}) = H(\vec{x})$ so it is clear that $p = \bar{p}$ and $s = \bar{s}$ and thus $p,s \in \mathbb{R}$. Also $q = \bar{r}$ and then of course $\bar{q} = r$. Hence knowing $p,q,r$ and $s$ is equivalent to knowing $4$ real numbers, two from $p$ and two from $q$. From these parameters the coordinates $(x,y,z,t)$ of a point in Minkowskian space-time can be constructed as:

\begin{equation*}
x + iy = q = \bar{r} \text{ ,   } t - z = p \text{ ,   } t+z = s
\end{equation*}

\noindent by comparing with matrix $A$ above. Hence it is true that there is a one to one correspondence between points in Minkowskian space-time and $2 \times 2$ Hermitian matrices.

Construct the following matrix:

\begin{equation*} 
U = \left( 
\begin{array}{cc}
\alpha & \beta \\
\gamma & \delta \\
\end{array}
\right)
\end{equation*}

\noindent with $\alpha$,$\beta$,$\gamma$,$\delta \in \mathbb{C}$. With the condition that $\det(U) = 1$. Such matrices $U$ form a group called the special linear group, which is denoted by $SL(2, \mathbb{C})$. Given $A(\vec{x})$ consider $U A(\vec{x}) U^{\dagger}$. This is a $2 \times 2$ Hermitian matrix since:

\begin{eqnarray*}
(U A(\vec{x}) U^{\dagger})^{\dagger} & =  (U^{\dagger})^{\dagger} A^{\dagger}(\vec{x}) U^{\dagger} \\
                                     & = U A(\vec(x)) U^{\dagger}
\end{eqnarray*} 

\noindent since $(U^{\dagger})^{\dagger} = U$ and $A^{\dagger} = A$. Hence there exists a point $\vec{x'} = (x', y', z', t')$ in Minkowskian space-time for which:

\begin{equation}\label{SL_trans}
A(\vec{x'}) = U A(\vec{x}) U^{\dagger}
\end{equation}

Any $U$ involves 6 real parameters, 2 each from the four complex components, with the condition $\det(U) = 1$ supplying two constraints. One on the real parts and one on the imaginary parts of the components. Now calculate the determinant of the matrix in the primed frame

\begin{eqnarray*}  
\det(A(\vec{x'})) & = \det(U A(\vec{x}) U^{\dagger}) \\
                  & = (\det(U))(\det(A(\vec{x}))(\det(U^{\dagger})) \\
                  & = (\det(U))(\det(A(\vec{x}))\bar{(\det(U))} \\
                  & = \det(A(\vec{x}))
\end{eqnarray*}

\noindent Thus we have the relation:

\begin{equation*}  
{t'}^2 - {x'}^2 - {y'}^2 - {z'}^2 = {t}^2 - {x}^2 - {y}^2 - {z}^2
\end{equation*}

\noindent Hence the transformation $\vec{x} \rightarrow \vec{x'}$ implicit in Eqn.(\ref{SL_trans}) is a Lorentz transformation. Eqn.(\ref{SL_trans}) describes the most general proper, orthochronous Lorentz transformation.

It is useful to calculate the matrix $U$ for some examples of Lorentz transformations. First, write Eqn.(\ref{SL_trans}) in terms of its components:

\begin{eqnarray*} 
\left(
\begin{array}{cc}
t' - z' & x' + i y' \\
x' - i y' & t' + z' \\
\end{array}
\right)
& =
\left(
\begin{array}{cc}
\alpha & \beta \\
\gamma & \delta \\
\end{array}
\right)
\left(
\begin{array}{cc}
t-z & x + i y \\
x - i y & t + z   \\
\end{array}
\right)
\left(
\begin{array}{cc}
\bar{\alpha} & \bar{\beta} \\
\bar{\gamma} & \bar{\delta} \\
\end{array}
\right) \\
& = \left(
\begin{array}{cc}
\alpha & \beta \\
\gamma & \delta \\
\end{array}
\right)
\left(
\begin{array}{cc}
(t-z)\bar{\alpha} + (x + iy)\bar{\beta} & (t-z)\bar{\gamma} + (x + iy)\bar{\delta} \\
(x - iy)\bar{alpha} + (t+z)\bar{\beta} & (x-iy)\bar{\gamma} + (t+z)\bar{\delta} \\
\end{array}
\right)
\end{eqnarray*}

\noindent Thus we have the relations:

\begin{equation}\label{coeff_equate_a}
\tag{{\theequation}a}
t' - z' = (t-z)\alpha\bar{\alpha} + (x + iy)\alpha\bar{\beta} + (x - iy)\beta\bar{alpha} + (t+z)\beta\bar{\beta}
\end{equation}
\begin{equation}\label{coeff_equate_b}
\tag{{\theequation}b}
x' + iy' = (t-z)\alpha\bar{\gamma} + (x + iy)\alpha\bar{\delta} + (x-iy)\beta\bar{\gamma} + (t+z)\beta\bar{\delta}
\end{equation}
\begin{equation}\label{coeff_equate_c}
\tag{{\theequation}c}
t' + z' = (t-z)\gamma\bar{\gamma} + (x + iy)\gamma\bar{\delta} + (x-iy)\delta\bar{\gamma} + (t+z)\delta\bar{\delta}
\end{equation}

\subsection{Example 1: Rotational Transformation}

\noindent Find $U$ \ref{coeff_equate_a}\ref{coeff_equate_b}\ref{coeff_equate_c} corresponding to the one parameter Lorentz transformation:

\begin{eqnarray*} 
x' = x\cos{\theta} + y\sin{\theta} \\
y' = -x\sin{\theta} + y\cos{\theta} \\
z' = z \\
t' = t
\end{eqnarray*} 

\noindent This implies that:

\begin{eqnarray*}
t'-z' = t-z \\
x'+iy' = (x+iy) e^{-i \theta} \\
t'+z' = t+z
\end{eqnarray*}

Equating coefficients of $x$, $y$, $z$, $t$ on both sides of Eqn.(\ref{coeff_equate_a}) to obtain: (SEE CALS)

\subsection{Example 2: Special Relativity Lorentz Transformation}\label{Special_Linear_Matrices_Example_2}

(CALCS)

\subsection{Example 3: Singular Lorentz Transformation}\label{Special_Linear_Matrices_Example_3}

(CALCS)

It is clear that there will always be two matrices $\pm U$ corresponding to every Lorentz transformation, since if $U$ satisfies $A(\vec{x'}) = U A(\vec{x}) U^{\dagger}$ then so does $-U$. Hence there is a $2$ to $1$ correspondence between the elements of $SL(2,\mathbb{C})$ and the proper orthochronous Lorentz transformation.

 
\section{Stereographic Projection and the Extended Complex Plane}\label{Section_Stereographic_Extended_Complex}

(ERROR. NEED MORE INTRO)

In this section a one to one mapping between the $2$-sphere, $\mathbb{S}^2$ and the extended complex plane $\hat{\mathbb{C}}$ is constructed.

\textit{Stereographic projection} is the mapping of points on a sphere to points on a plane. In $\mathbb{R}^3$ with rectangular Cartesian coordinates $(x, y, z)$, consider the unit sphere with centre $(0,0,0)$, defined by

\begin{equation*}
\mathbb{S}^2 \subset \mathbb{R}^3 \text{ : } x^2 + y^2 + x^2 = 1.
\end{equation*}

\noindent The projection $P \rightarrow Q$ is a stereographic projection, see Fig.(\ref{Stereographic_Projecttion_Fig}). A relationship between $(X,Y,0)$ and $(x,y,z)$ is constructed as follows. $P$ is subdivided into the line segment $NQ$ in some ratio, $l:m$ say. 

\begin{figure}[h!]
\begin{center}
\caption{\textit{The projection from the unit $2$-sphere to the $x$,$y$-plane is called a Stereographic projection. The point $P$ on the shpere is mapped to the point $Q$ on the plane by translation along the line joining the points $N$ and $P$}}
\label{Stereographic_Projecttion_Fig}
\includegraphics[scale=0.6]{figs/4_1.jpg}
\end{center}
\end{figure}

\noindent By coordinate geometry

\begin{eqnarray*}  
x = \frac{lX + mO}{l+m} = \frac{lX}{l+m}, \\
y = \frac{lY + mO}{l+m} = \frac{lY}{l+m}, \\
z = \frac{l.0+ m.1}{l+m} = \frac{m}{l+m}.
\end{eqnarray*}

\noindent Where O is the position of the origin. This implies that

\begin{eqnarray*}
1-z = \frac{l}{l+m}, \\
x = (1-z)X, \\
y = (1-z)Y.
\end{eqnarray*}

\noindent It is also known that $x^2+ y^2 +z^2 = 1$, so using this relation it is clear that

\begin{eqnarray*}
x^2 + y^2 = 1-z^2 = (1-z) (1+z), \\
(1-z)^2(X^2 +Y^2) = (1-z) (1+z).
\end{eqnarray*}

\noindent If the point $N$ is excluded, i.e. $z \neq 1$ then divide by $(1-z)^2$ to obtain

\begin{equation*} 
X^2 + Y^2 = \frac{1+z}{1-z},
\end{equation*}

\noindent and Rearrange to find

\begin{equation}\label{Ext_Complex_z_in_term_XY} 
z = \frac{X^2 + y^2 - 1}{X^2 + y^2 + 1}.
\end{equation}

Define $\zeta = X+ iY$ and rewrite Eqn.(\ref{Ext_Complex_z_in_term_XY}) to see that

\begin{equation*}
z = \frac{\zeta\bar{\zeta} - 1}{\zeta\bar{\zeta} + 1},
\end{equation*}

\noindent Which implies

\begin{equation*}
1- z = \frac{2}{\zeta\bar{\zeta} + 1}.
\end{equation*}

\noindent So relations for $(x,y,z) \in \mathbb{S}^2 \backslash \{N\}$ have been obtained in terms of $\zeta$, such that

\begin{align}\label{Ext_Complex_xy_interms_zeta}
x + iy & = \zeta (1-z) = \frac{2\zeta}{\zeta\bar{\zeta} + 1}, \\\label{Ext_Complex_z_interms_zeta}
z  & = \frac{\zeta\bar{\zeta} - 1}{\zeta\bar{\zeta} + 1}.
\end{align}

\noindent Hence the points on $\mathbb{S}^2 \backslash \{N\}$ are labelled by complex numbers $\zeta \in \mathbb{C}$. If a point $\zeta = \infty$, called the point at infinity of $\mathbb{C}$, is allowed then the following limits hold

\begin{eqnarray*}
x+ iy = \frac{2/\bar{\zeta}}{1 + 1/\zeta\bar{\zeta}} \rightarrow 0 \text{, as  } \zeta \rightarrow \infty, \\
z = \frac{1- 1/\zeta\bar{\zeta}}{1+ 1/\zeta\bar{\zeta}} \rightarrow 1 \text{, as  } \zeta \rightarrow \infty. 
\end{eqnarray*}

\noindent Then $N = (0,0,1)$ corresponds to $\zeta = \infty$. Thus in this way there is a one to one correspondence between the points of $\mathbb{S}^2$ and the points of the \textit{extended complex plane} $\hat{\mathbb{C}} = \mathbb{C} \cup {\infty}$, which is the usual complex plane with the point at infinity added. Since $\mathbb{S}^2$ has finite surface area, and is therefore called a \textit{compact manifold}, the identification of the points of $\hat{\mathbb{C}}$ with the points of $\mathbb{S}^2$ is called the \textit{compactification} of $\hat{\mathbb{C}}$.

The pair $(\zeta, \bar{\zeta})$ are called the \textit{stereographic coordinates} on $\mathbb{S}^2 \backslash \{N\}$. How are they related to the polar angles $\theta$ and $\phi$? To investigate this write the usual spherical polar coordinates in terms of $\zeta$. First it is known that

\begin{align*}
x & = \sin{\theta}\cos{\phi}, \\
y & = \sin{\theta}\sin{\phi}, \\
z & = \cos{\theta}.
\end{align*}

\noindent So by Eqn.(\ref{Ext_Complex_z_interms_zeta}) it is clear that

\begin{gather*}
z = \cos{\theta} = \frac{\zeta\bar{\zeta} - 1}{\zeta\bar{\zeta} + 1} \\
\zeta\bar{\zeta}\cos{\theta} + \cos{\theta} = \zeta\bar{\zeta} - 1\\
\zeta\bar{\zeta} = \frac{1 + \cos{\theta}}{1 - \cos{\theta}} = \frac{2\cos^2{\left(\theta/2\right)}}{2\sin^2{\left(\theta/2\right)}} \\
\zeta\bar{\zeta} = \cot^2{\left(\theta/2\right)}.
\end{gather*}

\noindent Now use Eqn.(\ref{Ext_Complex_xy_interms_zeta}) to obtain

\begin{gather*}
\sin{\theta}(\cos{\phi} +i \sin{\phi}) = \frac{2\zeta}{\cot^2{\left(\theta/2\right)} + 1 }, \\
2\sin{\left(\theta/2\right)}\cos{\left(\theta/2\right)}e^{i\phi} = 2\zeta \sin^2{\left(\theta/2\right)}, \\
\zeta = e^{i\phi}\cot{\left(\theta/2\right)}. 
\end{gather*}

\noindent This makes sense as if $\zeta = \infty$ then $\theta = 0$ as one would expect. In summary the following coordinate transformations have been constructed

\begin{align}
\vec{n} = (x, y, z) & = (\sin{\theta}\cos{\phi}, \sin{\theta}\sin{\phi}, \cos{\theta}), \\ \label{Ext_Complex_Vec_n}
&
\begin{Huge}
                     = \left( \frac{\bar{\zeta} + \zeta}{\bar{\zeta}\zeta + 1}  ,i\frac{\bar{\zeta} - \zeta}{\bar{\zeta}\zeta + 1}, \frac{\bar{\zeta}\zeta - 1}{\bar{\zeta}\zeta + 1}  \right),
\end{Huge}
\end{align}

\noindent where here $\vec{n}$ is a unit vector in $\mathbb{R}^3$ such that $\vec{n} \cdot \vec{n} = 1$.

These results are now extended to Minkowskian space-time. Let $\vec{x} = (x,y,z,t)$ be a point on the future null cone with vector $(0,0,0,0)$. Denote the future null cone as $N^{+}$, so that 

\begin{equation*}
N^+ : x^2 + y^2 + z^2 - t^2 = 0 \text{,  for  } t>0,
\end{equation*}

\noindent as all the vectors in the null cone have a Lorentz quadratic form equal to zero by definition. (SEE FIG PG 4:5). The intersection of the space-like hypersurface $t = \text{const}>0$ is a 2-sphere denoted by 

\begin{equation*}
\mathbb{S}^2 (t) : x^2 + y^2 + z^2 - t^2 = \text{const}.
\end{equation*}

\noindent There is a generator of $N^+$ passing through each point of $\mathbb{S}^2 (t)$. These generators are the null geodesics tangent to $N^+$ and passing through the point $(0,0,0,0)$. Hence the points of $\mathbb{S}^2 (t)$, denoted by $(\theta,\phi)$ or $\zeta$, label the \textit{generators} of $N^{+}$.

For any $t>0$ it is clear that

\begin{equation*}
\left(\frac{x}{t}\right)^2 +\left(\frac{y}{t}\right)^2 +\left(\frac{z}{t}\right)^2 = 1.
\end{equation*}

\noindent Hence we can write

\begin{eqnarray}\nonumber
\vec{x} & = t(\sin{\theta}\cos{\phi}, \sin{\theta}\sin{\phi}, \cos{\theta}, 1), \\\label{Ext_Complex_vec_x_relations}
        & = t\left( \frac{\bar{\zeta} + \zeta}{\bar{\zeta}\zeta + 1}  ,i\frac{\bar{\zeta} - \zeta}{\bar{\zeta}\zeta + 1}, \frac{\bar{\zeta}\zeta - 1}{\bar{\zeta}\zeta + 1},1  \right).
\end{eqnarray}

\noindent It is shown explicitly that the direction of $\vec{x}$ is determined by $(\theta,\phi)$ or $\zeta$ by comparing this to Eqn.(\ref{Ext_Complex_Vec_n}). All possible directions of $\vec{x}$ on $N^{+}$ are covered if $\zeta \in \hat{\mathbb{C}}$. Now the Lorentz transformation $\vec{x} \rightarrow \vec{x'}$ is investigated. This transformation takes the form  

\begin{equation*}
\vec{x} \rightarrow \vec{x'} = t'\left( \frac{\bar{\zeta'} + \zeta'}{\bar{\zeta'}\zeta' + 1}  ,i\frac{\bar{\zeta'} - \zeta'}{\bar{\zeta'}\zeta' + 1}, \frac{\bar{\zeta'}\zeta' - 1}{\bar{\zeta'}\zeta' + 1},1  \right).
\end{equation*}

\noindent We say that the null direction $\zeta$ is transformed to the null direction $\zeta'$. The relation between $\zeta$ and $\zeta'$ must also be determined. Construct the matrix $A(\vec{x})$ as in Eqn.(\ref{Special_Matrices_A_first}). 

\begin{equation*}
A(\vec{x}) = 
\left(
\begin{array}{cc}
\frac{2t}{\zeta\bar{\zeta}+1} & \frac{2t\zeta}{\zeta\bar{\zeta}+1} \\
\frac{2t\bar{\zeta}}{\zeta\bar{\zeta}+1} & \frac{2t\zeta\bar{\zeta}}{\zeta\bar{\zeta}+1} \\
\end{array}
\right)
=
c_0\left(
\begin{array}{cc}
1           & \zeta \\ 
\bar{\zeta} & \bar{\zeta}\zeta \\ 
\end{array}
\right).
\end{equation*}

\noindent Where $c_0 = 2t/\zeta\bar{\zeta}+1 \in \mathbb{R}^2$. Note that as $\vec{x}$ is a null vector $\det{(A(\vec{x}))} = 0$. Thus the transformed matrix $A(\vec{x'})$ is given similarly as

\begin{equation*}
A(\vec{x'}) = 
{c_0}'\left(
\begin{array}{cc}
1           & \zeta' \\ 
\bar{\zeta'} & \bar{\zeta'}\zeta' \\ 
\end{array}
\right).
\end{equation*}

\noindent Now, as in the examples in section (\ref{Special_Linear_Matrices_of_Lorentz}) we determine the special linear matrix $U$ such that

\begin{equation}\label{Ext_Complex_UAU}
A(\vec{x'}) = U A(\vec{x}) U^{\dagger}
\end{equation}

\noindent As before, this matrix equation is written component wise as

\begin{equation*}
{c_0}'\left(
\begin{array}{cc}
1           & \zeta' \\ 
\bar{\zeta'} & \bar{\zeta'}\zeta' \\ 
\end{array}
\right)
=
\left(
\begin{array}{cc}
\alpha & \beta \\
\gamma & \delta \\
\end{array}
\right)
{c_0}\left(
\begin{array}{cc}
1           & \zeta \\ 
\bar{\zeta} & \bar{\zeta}\zeta \\ 
\end{array}
\right)
\left(
\begin{array}{cc}
\bar{\alpha} & \bar{\beta} \\
\bar{\gamma} & \bar{\delta} \\
\end{array}
\right).
\end{equation*}

\noindent Then three separate relations between $\zeta$ and $\zeta'$ are obtained

\begin{eqnarray}\label{Ext_Complex_zeta_trans_1} 
{c_0}' = c_0 (\alpha \bar{\alpha} + \alpha \bar{\beta} \zeta + \bar{\alpha} \beta \bar{\zeta} + \beta \bar{\beta} \zeta \bar{\zeta}), \\\label{Ext_Complex_zeta_trans_2} 
{c_0}'\zeta' = c_0 (\alpha \bar{\gamma} + \alpha \bar{\delta} \zeta + \bar{\gamma} \beta \bar{\zeta} + \beta \bar{\delta} \zeta \bar{\zeta}), \\\label{Ext_Complex_zeta_trans_3} 
{c_0}'\zeta'\bar{\zeta'} = c_0 (\gamma \bar{\gamma} + \gamma \bar{\delta} \zeta + \bar{\gamma} \delta \bar{\zeta} + \delta \bar{\delta} \zeta \bar{\zeta}). \\
\end{eqnarray} 

\noindent Using Eqns.(\ref{Ext_Complex_zeta_trans_1}) and (\ref{Ext_Complex_zeta_trans_2}) to obtain

\begin{equation*}
\zeta' = \frac{{c_0}'\zeta'}{{c_0}'} = \frac{\alpha(\bar{\gamma} + \bar{\delta}\zeta) + \beta \bar{\zeta}(\bar{\gamma} + \bar{\delta}\zeta)}{\alpha(\bar{\alpha} + \bar{\beta}\zeta) + \beta \bar{\zeta}(\bar{\alpha} + \bar{\beta}\zeta)} \\
\end{equation*}

\noindent Thus 

\begin{equation}\label{Extended_Complex_Fractional_Linear_Transformation}
\zeta' = \frac{(\bar{\gamma} + \bar{\delta}\zeta)}{(\bar{\alpha} + \bar{\beta}\zeta)},
\end{equation}

\noindent with $\alpha\delta - \beta\gamma = 1$ as before. This is a fractional linear transformation of the extended complex plane $\hat{\mathbb{C}}$. There is a one to one correspondence here between proper, orthochronous Lorentz transformations and fractional linear transformations of the extended complex plane. This is because the matrices $\pm U$ both satisfy Eqn.(\ref{Ext_Complex_UAU}) as in the previous section., but now both matrices give the same transformation as the signs will cancel in the fractional transformation.

\subsection{Singular and Non-Singular Lorentz Transformations}

A given Lorentz transformation is equivalent to known $\alpha$, $\beta$ , $\gamma$ and $\delta$ parameters module a sign and therefore gives an explicit fractional linear transformation. For a given Lorentz transformation a \textit{fixed point} of the corresponding fractional linear transformation corresponds to an invariant null direction. The fixed points $\zeta$ satisfy the relation $\zeta' = \zeta$. Thus

\begin{eqnarray}\nonumber
\frac{(\bar{\gamma} + \bar{\delta}\zeta)}{(\bar{\alpha} + \bar{\beta}\zeta)} = \zeta, \\\label{Ext_Complex_fixed_point}
\Rightarrow \bar{\beta}\zeta^2 + (\bar{\alpha}- \bar{\delta})\zeta - \bar{\gamma} = 0. 
\end{eqnarray}

\noindent Clearly this is a quadratic equation over the field $\mathbb{C}$, thus it has two roots in general. Hence a Lorentz transformation does indeed leave two null directions invariant in general. The non-singular case is when these roots do not coincide. If Eqn.(\ref{Ext_Complex_fixed_point}) has only one root then the corresponding Lorentz transformation leaves one null direction invariant, this is the singular case.

Consider Eqn.(\ref{Ext_Complex_fixed_point}) again. Divide by $\zeta^2$ to obtain

\begin{equation*}
\bar{\beta} + (\bar{\alpha}- \bar{\delta})\zeta^{-1} - \bar{\gamma}\zeta^{-2} = 0.
\end{equation*}

\noindent Hence $\zeta = \infty$ is a solution of this equation if $\beta = 0$. If $zeta = \infty$ then $\vec{x}$ is given by $\vec{x} = t(0,0,1,1)$ by Eqn.(\ref{Ext_Complex_vec_x_relations}). Thus it is clear that this corresponds to the null direction $z=t$. Compare this to example 3,section (\ref{Special_Linear_Matrices_Example_3}). Here $\beta$ is zero AND the null direction is $z=t$ as expected. If $\zeta = 0$ is a solution to Eqn.(\ref{Ext_Complex_fixed_point}) it is required that $\gamma = 0$, thus $\vec{x} = (0,0,-1,1)$. So it is predicted that a Lorentz transformation with a special linear matrix of the form

\begin{equation*}   
U = 
\left(
\begin{array}{cc}
\alpha & \beta \\
0 & \delta \\
\end{array}
\right),
\end{equation*}   

\noindent will leave the $Z=-t$ null direction invariant.

\subsection{Example: Standard Lorentz Transformation}

Continuing on from Example 2, section (\ref{Special_Linear_Matrices_Example_3}), where $\alpha$, $\beta$, $\gamma$ and $\delta$ were determined. $\zeta'$ can now be expressed as

\begin{equation*} 
\zeta' = \frac{-\sqrt{\gamma_0 - 1} + \sqrt{\gamma_0 + 1}\zeta}{\sqrt{\gamma_0 + 1} - \sqrt{\gamma_0 - 1}\zeta}.
\end{equation*}

If the condition $\zeta' = \zeta$ is imposed then

\begin{eqnarray*}
\sqrt{\gamma_0 - 1}(\zeta^2 - 1) = 0 \\
\Rightarrow \zeta = \pm 1
\end{eqnarray*}

In the $\zeta = +1$ case $\vec{x} = t(1,0,0,1)$ and the invariant direction is $x=t$. Similarly in the $\zeta = -1$ case $\vec{x} = t(-1,0,0,1)$ and the invariant direction is $x = - t$. 

(ADD SECTION 4b IN NOTES, MAYBE IN THE APPENDIX?)





   














\section{Infinitesimal Lorentz Transformations}

There are Lorentz transformations that are small perturbations of the identity transformation and so $U \in SL(2,\mathbb{C})$ has the form

\begin{equation*}
U = \pm
\left(
\begin{array}{cc}
1 + \epsilon a & \epsilon b \\
\epsilon c & 1 + \epsilon f \\
\end{array}
\right),
\end{equation*}

\noindent where $a,b,c,f \in \mathbb{C}$ and $\epsilon$ is a small real parameter. Here terms of order $\epsilon^2$ will be neglected. As $U \in SL(2,\mathbb{C})$ its determinant is calculated as

\begin{equation*}
\det{(U)} = 1 + O(\epsilon^2).
\end{equation*}

\noindent Using this it is possible to obtain a relation between $f$ and $a$

\begin{eqnarray*}
(1 + \epsilon a)(1 + \epsilon f) - \epsilon^2 b c = 1 + O(\epsilon^2), \\
1 + \epsilon (a +f) = 1 + O(\epsilon^2), \\
\Rightarrow f = -a + O(\epsilon).
\end{eqnarray*}

\noindent Hence 

\begin{equation*}
U =
\left(
\begin{array}{cc}
1 + \epsilon a & \epsilon b \\
\epsilon c & 1 - \epsilon a \\
\end{array}
\right),
\end{equation*}

Now the explicit infinitesimal Lorentz transformations are calculated as in section \ref{Special_Linear_Matrices_of_Lorentz}, by substituting $U$ into

\begin{equation*}
A(\vec{x}') = U A(\vec{x}) U^{\dagger}.
\end{equation*}

\noindent Now writing this out in component form to obtain

\begin{eqnarray*}
\left(
\begin{array}{cc}
t'-z' & x' + iy' \\
x' + iy' & t'+z' \\
\end{array}
\right)
=
\left(
\begin{array}{cc}
1 + \epsilon a & \epsilon b \\
\epsilon c & 1 - \epsilon a \\
\end{array}
\right)
\left(
\begin{array}{cc}
t-z & x + iy \\
x - iy & t+z \\
\end{array}
\right)
\left(
\begin{array}{cc}
1 + \epsilon  \bar{a} & \epsilon \bar{c} \\
\epsilon \bar{b} & 1 - \epsilon \bar{a} \\
\end{array}
\right), \\
=
\left(
\begin{array}{cc}
1 + \epsilon a & \epsilon b \\
\epsilon c & 1 - \epsilon a \\
\end{array}
\right)
\left(
\begin{array}{cc}
(t-z)(1 + \epsilon \bar{a})+\epsilon \bar{b}(x + iy) & (t-z)\epsilon \bar{c}+ (1 - \epsilon \bar{a})(x + iy) \\
(x - iy)(1 + \epsilon \bar{a}) +\epsilon \bar{b} (t+z) & (x - iy)\epsilon \bar{c}+(1 - \epsilon \bar{a})(t+z) \\
\end{array}
\right).
\end{eqnarray*}

\noindent This then implies the three relations

\begin{eqnarray}\label{Infinitesimal_Lorentz_Transform_1}
t'-z' = t-z + \epsilon(a + \bar{a})(t-z) + \epsilon(b + \bar{b})x + i\epsilon(\bar{b} - b)y + O(\epsilon^2), \\\label{Infinitesimal_Lorentz_Transform_2}
t'+z' = t+ z - \epsilon(a + \bar{a})(t+z) + \epsilon (c + \bar{c})x + i \epsilon(c-\bar{c})y + O(\epsilon^2), \\\label{Infinitesimal_Lorentz_Transform_3}
x'+iy' = x+iy + \epsilon(a-\bar{a})(x+iy) + \epsilon(b + \bar{c})t + \epsilon (b-\bar{c})z + O(\epsilon^2).
\end{eqnarray}

\noindent As $a,b,c \in \mathbb{C}$, set 

\begin{equation*}
a = a_1 +ia_2 \text{,  } b = b_1 +ib_2 \text{,  } c = c_1 +ic_2 \text{.} 
\end{equation*}

\noindent Then subbing these into the above equations, eliminating $t$ and $z$ respectively from Eqn.(\ref{Infinitesimal_Lorentz_Transform_1}) and (\ref{Infinitesimal_Lorentz_Transform_2}) and taking real and imaginary parts of Eqn.(\ref{Infinitesimal_Lorentz_Transform_3}) to obtain

\begin{equation}\label{infinitesimal_Matrix_component_wise}
\left(
\begin{array}{c}
x' \\
y'\\
z'\\
t'\\
\end{array}
\right)
=
\left(
\begin{array}{c}
x \\
y\\
z\\
t\\
\end{array}
\right)
+ \epsilon
\left(
\begin{array}{cccc}
0            & -2a_2        & (b_1 - c_1) & (b_1+c_1)\\
2a_2         & 0            & (b_2+c_2)   & (b_2 - c_2) \\
-(b_1 - c_1) & -(b_2 - c_2) & 0           & -2a_1 \\
(b_1 + c_1)  & (b_2-c_2)    & -2a_1       & 0 \\
\end{array}
\right)
\left(
\begin{array}{c}
x \\
y\\
z\\
t\\
\end{array}
\right)
+ O(\epsilon^2).
\end{equation}

\noindent The above $4 \times 4$ matrix will be denoted as $\tensor{L}{^i_j}$, so that Eqn.(\ref{infinitesimal_Matrix_component_wise}) can be written simply as 

\begin{equation}\label{Infinitesimal_Infinitesimal_Lorentz_Transformation}
\bar{x}^i = x^i + \epsilon \tensor{L}{^{i}_{j}} x^j + O(\epsilon^2).
\end{equation}

\noindent Where $\bar{x}^i = (x',y',z',t')$. It is also necessary to check that the Lorentz invariance of the quadratic form still holds. 

\begin{eqnarray*}
{x'}^2 + {y'}^2 + {z'}^2 - {t'}^2 & = x^2 + y^2 + z^2 - t^2 - 4\epsilon a_2 x y + 2 \epsilon(b_1 + c_1)xt \\
                                  & + 2\epsilon (b_1 - c_1)xz + 4 \epsilon a_2 yx + 2\epsilon (b_2 - c_2)yt \\
                                  & + 2 \epsilon (b_2 + c_2)yz - 4\epsilon a_1 zt + 2 \epsilon (c_1 - b_1)zx \\
                                  & -2 \epsilon (c_2 + b_2)zy + 4 \epsilon a_1 tz - 2 \epsilon (c_1 + b_1)tx \\
                                  & -2\epsilon (b_2 - c_2) ty + O(\epsilon^2) \\
                                  & = x^2 + y^2 + z^2 - t^2 + O(\epsilon^2)
\end{eqnarray*}

\noindent Hence this transforamtion is still a Lorentz Transformation if we neglect terms of order $\epsilon^2$.

Consider the time-like world line (SEE FIG pg 5:3)of a particle in Minkowkian space-time $x^i = x^i(s)$. If $s$ is arc length or proper time then $v^i(s) = \frac{dx^i}{ds}$ is the unit tangent(NOT SURE WHY???) vector field. It is clear that $v^i(s)$ must be time-like as $x^i(s)$ is time-like, thus

\begin{equation*}
\eta_{ij} v^i v^j = -1. 
\end{equation*}

\noindent Where $\eta_{ij} - \text{diag}(1,1,1,-1)$ is the metric of minkowskiam space-time. This implies that 

\begin{equation*}
(v^1)^2  + (v^2)^2 + (v^3)^2  - (v^4)^2 = -1.
\end{equation*}

Now consider taking a step along the world line of the particle. Define $\bar{s} = s + \alpha$, where $\alpha$ is some real parameter, so that $v^i (s+\alpha) \vcentcolon = \bar{v}^i(\bar{s})$. Hence we also have

\begin{equation*}
({\bar{v}^1})^2  + ({\bar{v}^2})^2 + ({\bar{v}^3})^2  - ({\bar{v}^4})^2 = -1,
\end{equation*}

\noindent and so $v^i(s)$ and $\bar{v}^i(\bar{s})$ are related by a Lorentz transformation. In particlular $v^i(s+\epsilon)$ and $v^i(s)$ are related by an infinitesimal Lorentz Transformation given by Eqn.(\ref{Infinitesimal_Infinitesimal_Lorentz_Transformation}),

\begin{equation}
v^i(s+\epsilon) = v^i(s) = \epsilon \tensor{L}{^{i}_{j}}(s)v^j (s) + O(\epsilon^2).
\end{equation}

\noindent Rearranging to obtain

\begin{equation}
\frac{v^i(s+\epsilon) - v^i(s)}{\epsilon} = \tensor{L}{^{i}_{j}}(s)v^j (s) + O(\epsilon).
\end{equation}

\noindent Now taking the limit as the infinitesimal step, $\epsilon$ goes to zero to obtain a continuous differentiable equation,

\begin{equation}\label{Infinitesimal_DE_interms_v}
\frac{dv^i}{ds} = \tensor{L}{^{i}_{j}}(s) v^j (s).
\end{equation}

\noindent This equation determines the trajectory of the particle through Minkowskian space-time. In terms of $x$ this is equivalent to

\begin{equation*}
\frac{d^2 x^i}{ds^2} = \tensor{L}{^{i}_{j}}(s) \frac{d x^j}{ds}.
\end{equation*}

It is interesting to write these equations in terms of the particles $3$-velocity given by

\begin{equation*}
\vec{u} = \left( \frac{dx}{dt}, \frac{dy}{dt}, \frac{dz}{dt} \right).
\end{equation*}

\noindent Start by using the chain rule on $v^i$,

\begin{equation*}\label{Infinitesimal_Chain_Rule}
v^i = \frac{dx^i}{ds} = \frac{dx^i}{dt} \frac{dt}{ds} = \left(\frac{dx}{dt},\frac{dy}{dt},\frac{dz}{dt},1\right) \frac{dt}{ds}.
\end{equation*}

\noindent Now determine the first integral of $v^i$, which is equal to $-1$ as $v^i$ is time-like,

\begin{equation*} 
-1 = \eta_{ij} v^i v^j =  \left\{ \left( \frac{d}{dt} \right)^2  + \left( \frac{d}{dt} \right)^2  + \left( \frac{d}{dt} \right)^2 - 1  \right\} \left( \frac{dt}{ds} \right)^2,
\end{equation*} 

\noindent as this is just the scalar product in Minkowskian space-time. Therefore (NOT SURE WHERE THIS COMES FROM)

\begin{equation*}
\frac{dt}{ds} = \gamma (s) \vcentcolon = (1-u^2)^{-1/2},
\end{equation*}

\noindent where $u = |\vec{u}| = \sqrt{\vec{u} \cdot \vec{u}}$. Thus from Eqn.(\ref{Infinitesimal_Chain_Rule})

\begin{equation}\label{infinitesimal_v_interms_gamma}
v^i = \gamma(u) (\vec{u}, 1)
\end{equation}

\noindent It is now conveinient to display Eqn.(\ref{Infinitesimal_DE_interms_v}) as two equations denoting the spacial part and the temproal part, in terms of $\gamma$ and $u$. Again using the chain rule to obtain

\begin{equation*} 
\frac{dt}{ds} \frac{dv^i}{dt} = \tensor{L}{^{i}_{j}}v^j. 
\end{equation*} 

\noindent This then implies that 

\begin{eqnarray}\label{Infinitesimal_gamma_u_1}
\gamma(u) \frac{d}{dt} (\gamma(u) u^{\alpha}) = \tensor{L}{^{\alpha}_{j}} v^j, \\\nonumber
\gamma(u) \frac{d}{dt} \gamma(u) = \tensor{L}{^{4}_{j}} v^j,
\end{eqnarray}

\noindent as $v^i = \gamma(u)(\vec{u},1)$. Here we have used the usual convention that greek indices denote the sum over the spacial indices only, thus $\alpha = 1,2,3$. Now Eqn.(\ref{infinitesimal_v_interms_gamma}) can be used to rewrite the $\tensor{L}{^{i}_{j}}$ coefficients to get

\begin{eqnarray}\label{Infinitesimal_gamma_u_2}
\tensor{L}{^{\alpha}_{j}} v^j = \gamma(u) (\tensor{L}{^{\alpha}_{\beta}} u^{\beta} + \tensor{L}{^{\alpha}_4}) \\ \nonumber
\tensor{L}{^{4}_{j}} v^j = \gamma(u) (\tensor{L}{^{4}_{\alpha}} u^{\alpha0})
\end{eqnarray} 

\noindent where $\tensor{L}{^{4}_{4}} = 0$ from Eqn.(\ref{infinitesimal_Matrix_component_wise}). Putting together Eqns.(\ref{Infinitesimal_gamma_u_1}) and (\ref{Infinitesimal_gamma_u_2}) to obtain differential equations for the spacial and temporal coordinates in terms of the particles $3$-velocity,

\begin{eqnarray*} 
\frac{d}{dt} (\gamma(u) u^{\alpha}) = \tensor{L}{^{\alpha}_{\beta}} u^{\beta} + \tensor{L}{^{\alpha}_{4}} \\
\frac{d\gamma(u)}{dt} = \tensor{L}{^{4}_{\alpha}} u^{\alpha}
\end{eqnarray*} 

(ON TOP OF pg 5:6)




 










\section{Conclusion}

In this project it has been shown that in general, proper orthochronous Lorentz transformations (POLTs) have two invariant null directions in Minkowskian space-time. The special case where both of these coincide, known as the singular case, has been investigated. Starting from the Schwarzschild solution of the vacuum field equations a strange line element was derived and shown to be the Minkowskian line element with $m = 0$, as in Eqn.(\ref{Kasner_after_limit_no_k}). A Lorentz quadratic form preserving transformation on this line element was found to have two parameters and to be a singular Lorentz transformation. Thus it was shown that all singular Lorentz transformations are two parameter abelian subgroups of the Lorentz group. A connection between $2 \times 2$ Hermitian matrices and points in Minkowskian space-time was then established, from which it was shown that there is a two to one correspondence between $SL(2,\mathbb{C})$ matrices and POLTs. Then a coordinate transformation between the sphere $\mathbb{S}^2$ and the extended complex plane $\hat{\mathbb{C}}$ was used to relate Minkowskian space time to $\hat{\mathbb{C}}$. A POLT in these coordinates was then shown to derive the fractional linear transformation, or Mobius transformation of Eqn.(\ref{Extended_Complex_Fractional_Linear_Transformation}), thus there is a one to one correspondence between POLTs and fractional linear transformations. Also this can be understood as an isomorphism between the proper orthochronous Lorentz group and the Mobius group. It was then shown that the fractional linear form of the infinitesimal Lorentz transformations is equivalent to the Lorentz force and thus a connection was made between $L_{ij}$ and the electromagnetic tensor, usually denoted by $F_{ij}$. When the fractional linear form of the infinitesimal Lorentz transformations was constructed it was clear that there were a number of cases, each with either one or two fixed null directions. Of interest to this project was the singular case, thus the condition $a^2 + bc = 0$ was established. By writing this condition in terms of the electric and magnetic fields in the electromagnetic tensor the conditions for a pure radiation field, Eqn(\ref{Infinitesimal_Pure_Rad_Cond_1}) and (\ref{Infinitesimal_Pure_Rad_Cond_2}) emerged. By then writing the condition for a singular Lorentz transformation in the coordinates of Minkowskian space-time it was shown that the singular null direction, $k^i$ was in fact the direction of propagation of the pure radiation field. Finally, it was proven that the electric and magnetic fields and the singular null direction form an orthonormal triad.     


\newpage

\begin{appendix}

\section{Singular Lorentz Transformation with Special Significance Given to $x$}\label{Appendix_Special_Significance_x}

The choice of components of the matrix $A$ in Eqn.(\ref{Special_Matrices_A_first}) gives an arbitrary but special significance to the corrdinate $z$. In this calculation the matrix $U$ of Eqn.(\ref{SL_trans}) is determined for a singular Lorentz transformation, in which special significance has been given to the coordinate $x$. This transformation is given by

\begin{align*}
t'-x' & = t-x, \\
z'+iy' & = z + iy + w(t-x), \\
t'+x' & = t+x + w(z-iy) + \bar{w} (z + iy) + w \bar{w} (t-x).
\end{align*}

\noindent Notice that this is the same tranformation as in Example 3, Section (\ref{Special_Linear_Matrices_Example_3}), with $x$ and $z$ swapped. This is equivalent to swapping these two coordinates in the matrix $A$ only. Note that if $x$ and $z$ were exchanged in both $A$ and the transformation then the $U$ obtained would be exactly the same as the example done previously.

First, the components of the matrix $A$ must be construced. Thus the transformation must be written out explicitly as 

\begin{align*}
x' & = x + \frac{1}{2} (w + \bar{w})z + \frac{i}{2}(\bar{w}- w)y + \frac{1}{2}w\bar{w}(t-x), \\
y' & = y + \frac{i}{2} (\bar{w} - w)(t-x), \\
z' & = z + \frac{1}{2} (w + \bar{w})(t-x), \\
t' & = t + \frac{1}{2} (w + \bar{w})z + \frac{i}{2} (\bar{w}-w)y + \frac{1}{2} w\bar{w}(t-x).
\end{align*}

\noindent Now construct the components of $A$

\begin{align*}
t' - z' & = t - z + \frac{1}{2}(w + \bar{w})z + \frac{i}{2} (\bar{w}-w)y + \frac{1}{2}w\bar{w}(t-x) - \frac{1}{2}(\bar{w} + w)(t-x),\\
t' + z' & = t + z + \frac{1}{2}(w + \bar{w})z + \frac{i}{2} (\bar{w}-w)y + \frac{1}{2}w\bar{w}(t-x) + -\frac{1}{2}(\bar{w} + w)(t-x),\\
x' + iy' & = x +  \frac{1}{2}(w + \bar{w})z + \frac{i}{2} (\bar{w}-w)y + \frac{1}{2}w\bar{w}(t-x) - \frac{1}{2}(\bar{w} + w)(t-x) + iy.
\end{align*}

\noindent Equating coefficients of $x$, $y$, $z$, $t$ on both sides of Eqn.(\ref{general_coeff_equate_a}) to obtain

\begin{subequations}
\begin{gather}
\label{Ex_Ap_equate_coeffs_first_a}
\alpha \bar{\beta} + \bar{\alpha} \beta = \frac{1}{2}(\bar{w}+w) - \frac{1}{2}w\bar{w}, 
\\\label{Ex_Ap_equate_coeffs_first_b}
\alpha \bar{\beta} - \bar{\alpha} \beta = \frac{1}{2} (\bar{w} - w), 
\\\label{Ex_Ap_equate_coeffs_first_c}
-\alpha \bar{\alpha} + \beta \bar{\beta} = - 1 + \frac{1}{2}(\bar{w} + w), 
\\\label{Ex_Ap_equate_coeffs_first_d}
\alpha \bar{\alpha} + \beta \bar{\beta} = 1 + \frac{1}{2}w\bar{w} - \frac{1}{2}(\bar{w} + w). 
\end{gather}
\end{subequations}

\noindent Then Eqn.(\ref{Ex_Ap_equate_coeffs_first_a}) and (\ref{Ex_Ap_equate_coeffs_first_b}) imply $\alpha \bar{\beta} = \frac{1}{2}\bar{w} - \frac{1}{4}w\bar{w}$. Also Eqn.(\ref{Ex_Ap_equate_coeffs_first_c}) and (\ref{Ex_Ap_equate_coeffs_first_d}) imply $\beta\bar{\beta} = \frac{1}{4}w\bar{w}$ and $\alpha \bar{\alpha} = 1 + \frac{1}{4}w\bar{w} - \frac{1}{2}\bar{w} - \frac{1}{2}w$. Hence $\alpha$ can be written in terms of $\beta$

\begin{gather*}
\alpha \bar{\beta} = (\frac{1}{2}\bar{w} - \frac{1}{4}w\bar{w}), \\
\frac{1}{4} w \bar{w} \alpha = \frac{1}{2}\bar{w}(1-\frac{1}{2}w)\beta, \\
\alpha = \frac{(2-w)}{w}\beta.
\end{gather*}

Equating coefficients of $x$, $y$, $z$, $t$ on both sides of Eqn.(\ref{general_coeff_equate_b}) to obtain

\begin{subequations}
\begin{gather}\label{Ex_Ap_equate_coeffs_second_a}
\alpha \bar{\delta} + \beta\bar{\gamma} = 1 - \frac{1}{2}w\bar{w} + \frac{1}{2}(\bar{w}-w), \\\label{Ex_Ap_equate_coeffs_second_b}
\alpha \bar{\delta} - \beta\bar{\gamma} = \frac{1}{2}(\bar{w}-w) + 1,\\\label{Ex_Ap_equate_coeffs_second_c}
-\alpha\bar{\gamma} + \beta \bar{\delta} = \frac{1}{2}(\bar{w} + w) ,\\\label{Ex_Ap_equate_coeffs_second_d}
\alpha\bar{\gamma} + \beta \bar{\delta} = \frac{1}{2}w\bar{w} - \frac{1}{2}(\bar{w}-w). 
\end{gather}
\end{subequations}

\noindent Now Eqn.(\ref{Ex_Ap_equate_coeffs_second_a}) and (\ref{Ex_Ap_equate_coeffs_second_b}) imply $\beta \bar{\gamma} = -\frac{1}{4}w\bar{w}$. So using $\bar{\beta} \beta= \frac{1}{4}w\bar{w}$ again to obtain

\begin{gather*}
\frac{1}{4}w\bar{w} \bar{\gamma} = \bar{\beta}\beta\bar{\gamma},\\
\frac{1}{4}w\bar{w} \bar{\gamma} = -\frac{1}{4}w\bar{w} \bar{\beta},\\
\gamma = -\beta.
\end{gather*}

\noindent Also Eqn.(\ref{Ex_Ap_equate_coeffs_second_c}) and (\ref{Ex_Ap_equate_coeffs_second_d}) imply $\beta \bar{\delta} = \frac{1}{4}w\bar{w} + \frac{1}{2}w$. Thus

\begin{gather*}
\frac{1}{4}w\bar{w}\bar{\delta} = \left(\frac{1}{4}w\bar{w}+ \frac{1}{2}w\right)\bar{\beta},\\
\delta = \left(\frac{w\bar{w} + 2\bar{w}}{w\bar{w}}\right)\beta.
\end{gather*}

\noindent Equating coefficients of $x$, $y$, $z$, $t$ on both sides of Eqn.(\ref{general_coeff_equate_c}) to obtain

\begin{subequations}
\begin{gather}\label{Ex_Ap_equate_coeffs_third_a}
\gamma \bar{\delta} + \delta \bar{\gamma} = -\frac{1}{2}w\bar{w} - \frac{1}{2}(w + \bar{w}), \\\label{Ex_Ap_equate_coeffs_third_b}
\gamma \bar{\delta} - \delta \bar{\gamma} =  \frac{1}{2}(\bar{w}-w),\\\label{Ex_Ap_equate_coeffs_third_c}
-\gamma \bar{\gamma} + \delta \bar{\delta} = 1 + \frac{1}{2}(w + \bar{w}),\\\label{Ex_Ap_equate_coeffs_third_d}
\gamma \bar{\gamma} + \delta \bar{\delta} = 1 + \frac{1}{2}w\bar{w} + \frac{1}{2}(\bar{w}+w). 
\end{gather}
\end{subequations}

\noindent Here Eqn(\ref{Ex_Ap_equate_coeffs_third_a}) and (\ref{Ex_Ap_equate_coeffs_third_b}) imply $\gamma \bar{\delta} = -\frac{w}{4}(2+\bar{w})$. Also, Eqn(\ref{Ex_Ap_equate_coeffs_third_c}) and Eqn(\ref{Ex_Ap_equate_coeffs_third_d}) imply $\gamma \bar{\gamma} = \frac{1}{4}w\bar{w}$. So using these relations $\delta$ can be written in terms of $\gamma$ 

\begin{gather*}
\gamma \bar{\gamma} \bar{\delta} = \frac{w}{4}(2+\bar{w})\bar{\gamma},\\
\frac{1}{4}w\bar{w} \bar{\delta} = \frac{w}{4}(2+\bar{w})\bar{\gamma},\\
\delta = -\frac{(2 + w)}{w}\gamma.
\end{gather*}

\noindent At this point $\beta$ and $\delta$ have been written in terms of $\gamma$ and $\alpha$ is written in terms of $\beta$. Write $\alpha$ in terms of $\gamma$

\begin{equation*}
\alpha = \frac{(2-w)}{w}\beta = \frac{(w-2)}{w}\gamma.
\end{equation*}

\noindent Now use the condition that $\det{(U)} = 1$ and replace everything in favour of $\gamma$

\begin{gather*}
\alpha \delta - \beta \gamma = 1,\\
-\frac{(w^2 - 4)}{w^2}\gamma^2 + \gamma^2 = 1,\\
\frac{4}{w^2}\gamma^2 = 1,\\
\gamma = \pm \frac{w}{2}.
\end{gather*}

\noindent Replace $\gamma$ and write all the components of $U$ as functions of $w$ only

\begin{align*}
\alpha & = \pm \frac{1}{2}(w-2),\\
\beta & = \mp \frac{w}{2},\\
\gamma & = \pm \frac{w}{2},\\
\delta & = \mp \frac{1}{2}(w+2).
\end{align*}

\noindent Finally it is found that 

\begin{equation*}
U = \pm
\left(
\begin{array}{ccc}
\frac{w-2}{2} & & -\frac{w}{2}     \\
 & & \\
\frac{w}{2}   & & -\frac{(w+2)}{2} \\
\end{array}
\right).
\end{equation*}

Now the fractional linear trasnformation associated with $U$ must be determined. Using Eqn.(\ref{Extended_Complex_Fractional_Linear_Transformation}) to find that

\begin{equation*}
\zeta' = \frac{\bar{w} - (\bar{w} +2)\zeta}{\bar{w} -2 - \bar{w}\zeta}. 
\end{equation*}

\noindent The fixed points of the system and thus the null directions are found by setting $\zeta' = \zeta$ to obtain

\begin{gather*}
\bar{w} \zeta^2 - 2 \bar{w} \zeta + \bar{w} = 0,\\
\bar{w}(\zeta-1)^2 = 0.
\end{gather*}

\noindent Thus there is a fixed point at $\zeta = 1$. The corresponding null direction is then given by Eqn.(\ref{Ext_Complex_vec_x_relations}) such that $\vec{x}= t(1,0,0,1)$. Hence finally the null direction is the generator of $N^+$, $x=t$. So in this case where $x$ was given special significance instead of $z$ with the given singular Lorentz transformation it is found that the null direction is $x=t$ instead of $z=t$. This is as expected as by exchanging $z$ with $x$ the coordinate system has been rotated. 

\section{Standard Lorentz Transformation of the Electromagnetic Field Vectors}\label{Appendix_Standard_Transform_EM_Vectors}

In this appendix the transformation laws for $\vec{E} = (E^1, E^2, E^3)$ and $\vec{B} = (B^1, B^2, B^3)$ under the standard Lorentz transformation of Eqn.(\ref{Special_Matrices_Standard_Lorentz}) are derived. If $p^i$ is any vector transported along $x^i = x^i (s)$ then, in Eqn.(\ref{Infinitesimal_DE_interms_v}), $v^i$ can be replaced with $p^i$, such that

\begin{equation*} 
\frac{\mathrm{d}p^i}{\mathrm{d}s} = \tensor{L}{^i_j} p^j.
\end{equation*} 

\noindent Here $s =$ arc length so the usual line element

\begin{equation*} 
\epsilon \mathrm{d}s^2 = \mathrm{d}x^2 + \mathrm{d}y^2 + \mathrm{d}z^2 - \mathrm{d}t^2,
\end{equation*} 

\noindent is formed. $p^i$ is a $4$-vector so it transforms like $x^i = (x,y,z,t)$ under the standard Lorentz transformation, such that

\begin{align*}
\bar{p}^1 & = \gamma (p^1 - v p^4),\\
\bar{p}^2 & = p^2, \\
\bar{p}^3 & = p^3,\\
\bar{p}^4 & = \gamma (p^4 - v p^1).
\end{align*}

\noindent Which implies that 

\begin{equation*}
\mathrm{d}p^i = (\mathrm{d}p^1,\mathrm{d}p^2,\mathrm{d}p^3,\mathrm{d}p^4),
\end{equation*}

\noindent is a $4$-vector and ds is invariant, thus $\frac{\mathrm{d}p^i}{\mathrm{d}s}$ is also a $4$-vector. Hence $\tensor{L}{^i_j} p^j$ is a $4$-vector for any $4$-vector $p^i$, where $\tensor{L}{^i_j}$ is given by

\begin{equation*}
\tensor{L}{^i_j} = 
\left(
\begin{array}{cccc}
0    & B^3  & -B^2 & E^1 \\
-B^3 & 0    & B^1  & E^2 \\
B^2  & -B^1 & 0    & E^3 \\
E^1  & E^2  & E^3  & 0   \\
\end{array}
\right)
\end{equation*}

\noindent and a factor of $q/m$ has been left out as it will play no role. Write out $\tensor{L}{^i_j} p^j$ explicitly

\begin{align*}
\tensor{L}{^1_j}p^j & = B^3 p^2 - B^2 p^3 + E^1 p^4, \\
\tensor{L}{^2_j}p^j & = -B^3 p^1 + B^1 p^3 + E^2 p^4, \\
\tensor{L}{^3_j}p^j & = B^2 p^1 - B^1 p^2 + E^3 p^4, \\
\tensor{L}{^4_j}p^j & = E^1 p^1 + E^2p^2 + E^3 p^3.
\end{align*}

\noindent Since $\tensor{L}{^i_j} p^j$ is a $4$-vector it transforms under the standard Lorentz transformation, such that

\begin{align}
\label{Appendix_Lp_Standard_Transform_a}
\tensor{\bar{L}}{^1_j}\bar{p}^j & = \gamma (\tensor{L}{^1_j}p^j - v \tensor{L}{^4_j}p^j),
\\\label{Appendix_Lp_Standard_Transform_b}
\tensor{\bar{L}}{^2_j}\bar{p}^j & = \tensor{L}{^2_j}p^j,
\\\label{Appendix_Lp_Standard_Transform_c}
\tensor{\bar{L}}{^3_j}\bar{p}^j & = \tensor{L}{^3_j}p^j,
\\\label{Appendix_Lp_Standard_Transform_d}
\tensor{\bar{L}}{^4_j}\bar{p}^j & = \gamma(\tensor{L}{^4_j}p^j - v \tensor{L}{^1_j}p^j).
\end{align}

\noindent Write out Eqn.(\ref{Appendix_Lp_Standard_Transform_a}) explicitly

\begin{gather*}
\bar{B}^3 \bar{p}^2 - \bar{B}^2 \bar{p}^3 + \bar{E}^1 \bar{p}^4 = \gamma(B^3p^2 - B^2 p^3 + E^1p^4 - v(E^1p^1 + E^2 p^2 + E^3p^3)),\\
\bar{B}^3 \bar{p}^2 - \bar{B}^2 \bar{p}^3 + \bar{E}^1 \gamma(p^4 - vp^1) = \gamma(B^3p^2 - B^2 p^3 + E^1p^4 - vE^1p^1 -v E^2 p^2 -v E^3p^3).
\end{gather*}

\noindent Now equate the coefficients of $p^1$, $p^2$, $p^3$ and $p^4$, which gives

\begin{subequations}
\begin{align}
\label{Appendix_EB_Standard_Transform_1_p_coffs_a}
\bar{E}^1 & = E^1,
\\\label{Appendix_EB_Standard_Transform_1_p_coffs_b}
\bar{B}^3 & = \gamma(B^3 - vE^2),
\\\label{Appendix_EB_Standard_Transform_1_p_coffs_b}
\bar{B}^2 & = \gamma(B^2 + vE^3).
\end{align}
\end{subequations}

\noindent Where generally one of the relations given is trivial. Continue by writting Eqn.(\ref{Appendix_Lp_Standard_Transform_b}) explicitly and again equate coefficients to get

%\begin{subequations}
\begin{align}
\label{Appendix_EB_Standard_Transform_2_p_coffs_a}
B^3 & = \gamma (\bar{B}^3 + v \bar{E}^2),
\\\label{Appendix_EB_Standard_Transform_2_p_coffs_b}
\bar{B}^1 & = {B}^1,
\\\label{Appendix_EB_Standard_Transform_2_p_coffs_c}
E^2 & = \gamma(v\bar{B}^3 + \bar{E}^2).
\end{align}
%\end{subequations}

\noindent Eqn(\ref{Appendix_EB_Standard_Transform_2_p_coffs_a}) and (\ref{Appendix_EB_Standard_Transform_2_p_coffs_c}) now need to be inverted using $\gamma (1-v^2) = \gamma^{-1}$, to obtain

\begin{subequations}
\begin{align}
\label{Appendix_EB_Standard_Transform_2_p_coeffs_Inverted_a}
\bar{B}^3 & = \gamma (B^3 - vE^2),
\\\label{Appendix_EB_Standard_Transform_2_p_coeffs_Inverted_b}
\bar{E}^2 & = \gamma (E^2 - vB^3)
\end{align}
\end{subequations}

\noindent It is also necessary to write out Eqn.(\ref{Appendix_Lp_Standard_Transform_c}) explicitly and equate the coefficients of $p$ to get

\begin{align}
\label{Appendix_EB_Standard_Transform_3_p_coffs_a}
B^2 & = \gamma (\bar{B}^2 - v\bar{E}^3),
\\\label{Appendix_EB_Standard_Transform_3_p_coffs_b}
\bar{B}^1 & = {B}^1,
\\\label{Appendix_EB_Standard_Transform_3_p_coffs_c}
E^3 & = \gamma(\bar{E}^3 - v\bar{B}^2).
\end{align}

\noindent Again, Eqn(\ref{Appendix_EB_Standard_Transform_3_p_coffs_a}) and (\ref{Appendix_EB_Standard_Transform_3_p_coffs_c}) need to be inverted to give

\begin{subequations}
\begin{align}
\label{Appendix_EB_Standard_Transform_3_p_coeffs_Inverted_a}
\bar{E}^3 & = \gamma (E^3 + vB^2),
\\\label{Appendix_EB_Standard_Transform_3_p_coeffs_Inverted_b}
\bar{B}^2 & = \gamma (B^2 + vE^3)
\end{align}
\end{subequations}

\noindent Following a similar procedure write Eqn.(\ref{Appendix_Lp_Standard_Transform_d}) out explicitly, equate coefficients to get

\begin{align*}
\bar{E}^1 = E^1,\\
\bar{E}^2 = \gamma (E^2 - vB^3),\\
\bar{E}^3 = \gamma (E^3 + vB^2),
\end{align*}

\noindent as obtained already. Thus in summary the final form of the standard Lorentz transformation of the electromagnetic vectors $E$ and $B$ is given by

\begin{align*}
\bar{E}^1 = E^1, \qquad & \bar{E}^2 = \gamma (E^2 - vB^3), \qquad & \bar{E}^3 = \gamma(E^3 + vB^2),\\
\bar{B}^1 = B^1, \qquad & \bar{B}^2 = \gamma (B^2 + vE^3), \qquad & \bar{B}^3 = \gamma (B^3 - vE^2). 
\end{align*}

\noindent Using these transformations it is possible to verify that the quantities in Eqn.(\ref{Infinitesimal_Pure_Rad_Cond_1}) and (\ref{Infinitesimal_Pure_Rad_Cond_2} are invariant, by direct computation. Start with

\begin{align*} 
{|\bar{E}|}^2 - {|\bar{B}|}^2 & = {(\bar{E}^1)}^2 + {(\bar{E}^2)}^2 + {(\bar{E}^3)}^2 - {(\bar{B}^1)}^2 - {(\bar{B}^2)}^2 - {(\bar{B}^3)}^2,\\
                              & = (E^1)^2 + \gamma^2 ((E^2)^2 - 2vE^2B^3 + v^2 (B^3)^2) \\
                              & + \gamma^2 ((E^3)^2 + 2vE^3 B^2 + v^2 (B^2)^2) - (B^1)^2 \\
                              & - \gamma^2 ((B^2)^2 + 2vB^2E^3 + v^2 (B^3)^2) \\
                              & - \gamma^2 ((B^3)^2 - 2vB^3E^2 + v^2 (E^2)^2),
\end{align*}

\noindent which gives after some cancellations

\begin{align*}
{|\bar{E}|}^2 - {|\bar{B}|}^2 & = (E^1)^2 + \gamma^2 (1-v^2)(E^2)^2 + \gamma^2 (1-v^2)(E^3)^2, \\
                              & - (B^1)^2 - \gamma^2 (1-v^2)(B^2)^2 - \gamma^2 (1-v^2)(B^3)^2, \\
                              & = {({E}^1)}^2 + {({E}^2)}^2 + {({E}^3)}^2 - {({B}^1)}^2 - {({B}^2)}^2 - {({B}^3)}^2,
\end{align*}

\noindent as $\gamma^2 (1-v^2) = 1$. Finaly the next equation is verified as follows

\begin{align*}
\bar{E} \cdot \bar{B} & = \bar{E}^1\bar{B}^1 + \bar{E}^2\bar{B}^2 + \bar{E}^3\bar{B}^3, \\
                      & = E^1B^1 \\
                      & + \gamma^2 (E^2B^2 + vE^2E^3 - vB^2B^3 - v^2E^3B^3) \\
                      & + \gamma^2 (E^3B^3 - vE^2E^3 + vB^2B^3 - v^2 E^2B^2), \\
                      & = E^1B^1 + \gamma^2 (1-v^2)E^2B^2 \gamma^2 (1-v^2)E^3B^3,\\ 
                      & = {E}^1{B}^1 + {E}^2{B}^2 + {E}^3{B}^3. \\
\end{align*}

\section{The Triad of Electromagnetic Radiation}\label{Appendix_Orthonormal_Triad}

In this appendix it is shown that the vectors $(\vec{n}, \vec{B}, \vec{E})$ form a right-handed triad. 

From Eqn.(\ref{Infinitesimal_K_Unit_Vector_Formula}), $k^i$ can be rewritten as

\begin{equation*}
k^i = (\zeta \bar{\zeta} + 1)(\vec{n}, 1),
\end{equation*}

\noindent where $\vec{n} = (n^1, n^2, n^3)$ is a unit vector such that $\vec{n} \cdot \vec{n} = 1$. Expanding the relation $\mathcal{L}_{ij} k^i = 0$ in terms of $\vec{n}$ gives

\begin{subequations}
\begin{align}
\label{Appendix_2_Expand_L_K_product_a}
(B^3 - i E^3)n^2 - (B^2 - i E^2)n^3 + E^1 + iB^1 & = 0, 
\\\label{Appendix_2_Expand_L_K_product_b}
-(B^3 - i E^3)n^1 + (B^1 - i E^1)n^3 + E^2 + iB^2 & = 0, 
\\\label{Appendix_2_Expand_L_K_product_c}
(B^2 - i E^2)n^1 - (B^1 - i E^1)n^2 + E^3 + iB^3 & = 0, 
\\\label{Appendix_2_Expand_L_K_product_d}
-(E^1 + i B^1)n^1 - (E^2 + iB^2)n^2 - (E^3 + iB^3) & = 0.
\end{align}
\end{subequations}

\noindent Then by equating real and imaginary parts of Eqn.(\ref{Appendix_2_Expand_L_K_product_d}) it is clear that  $\vec{E} \cdot \vec{n} = \vec{B} \cdot \vec{n}$. Also, from Eqns.(\ref{Appendix_2_Expand_L_K_product_a}) - (\ref{Appendix_2_Expand_L_K_product_c}) the following series of equations are seen, first Eqn.(\ref{Appendix_2_Expand_L_K_product_a}) implies

\begin{align*}
E^1 & = B^2 n^3 - B^3 n^2, \\
B^1 & = E^3 n^2 - E^2 n^3, 
\end{align*}

\noindent while Eqn.(\ref{Appendix_2_Expand_L_K_product_b}) gives

\begin{align*} 
E^2 & = B^3 n^1 - B^1n^3, \\
B^2 & = E^1 n^3 - E^3 n^1.
\end{align*} 

\noindent Then Eqn.(\ref{Appendix_2_Expand_L_K_product_c}) implies

\begin{align*}
E^3 & = B^1 n^2 - B^2 n^1,\\
B^3 & = E^2 n^1 - E^1 n^2.
\end{align*}

\noindent Thus is it clear that these are the components of the curl $\vec{E} = \vec{B} \times \vec{n}$ and $\vec{B} = \vec{n} \times \vec{E}$. This curl is proof that the vectors $(\vec{n}, \vec{E}, \vec{B})$ form a right-handed triad. 

\section{Derivation of Familiar Electromagnetism Equations from Pure Radiation Conditions}\label{Appendix_Familair_EM_from_Pure_Cond}

In this appendix the familiar equations \cite[p. 42]{Schild_Lectures}

\begin{equation*}
L_{ij}L^{ij} = 0 = L_{ij} ^{*}L_{ij},
\end{equation*}

\noindent are shown to be equivalent to the pure radiation conditions of Eqn.(\ref{Infinitesimal_Pure_Rad_Cond_1}) and (\ref{Infinitesimal_Pure_Rad_Cond_2}). First consider the scalar product

\begin{align*}
\mathcal{L}_{ij} \mathcal{L}^{ij} & = 2( \mathcal{L}_{12}\mathcal{L}_{12}+\mathcal{L}_{13} \mathcal{L}_{13}+\mathcal{L}_{23}\mathcal{L}_{23} \\
                                  & - \mathcal{L}_{14} \mathcal{L}_{14} - \mathcal{L}_{24}\mathcal{L}_{24} - \mathcal{L}_{34}\mathcal{L}_{34})
\end{align*}

\noindent where $\mathcal{L}_{ij} = L_{ij} + i ^{*}L^{ij}$ as before. Writting this in terms of the components of $L$ gives

\begin{align*}
\frac{1}{2} \mathcal{L}_{ij} \mathcal{L}^{ij} & = \frac{q^2}{m^2} ( -4a^2 + b^2 - 2bc + c^2 - b^2 - 2bc - c^2 \\
                                              & - b^2 - 2bc - c^2 + b^2 - 2bc + c^2 - 4a^2) \\
                                              & = -\frac{8q^2}{m^2} (a^2 + bc) = 0,  
\end{align*}

\noindent as $a^2 + bc = 0$ for a singular Lorentz transformation and this is equivalent to the pure radiation field conditions as shown in Section (\ref{Infinitesimal_Section_Fractional_Linear}). This scalar product can also to written in terms of $L_{ij}$ such that

\begin{equation*}
\mathcal{L}_{ij} \mathcal{L}^{ij} = L_{ij}L^{ij} - ^{*}L_{ij}^{*}L^{ij} +2i L_{ij}^{*}L^{ij}.
\end{equation*}

\noindent Rewritting the middle terms component-wise gives the result

\begin{align*}
^{*}L_{ij} ^{*}L^{ij} & = 2 ( ^{*}L_{12} ^{*}L_{12} + ^{*}L_{13} ^{*}L_{13} + ^{*}L_{23} ^{*}L_{23} \\
                     & - ^{*}L_{14} ^{*}L_{14} - ^{*}L_{24} ^{*}L_{24} - ^{*}L_{34} ^{*}L_{34}) \\
                     & 2( L_{34}L_{34} + L_{24}L_{24} + L_{14}L_{14} \\
                     & - L_{23}L_{23} - L_{13}L_{13} - L_{12}L_{12})
                     & = - L_{ij}L^{ij},
\end{align*}

\noindent where $^{*}L_{ij} = \frac{1/2}\epsilon_{ijkl}L^{kl}$ has been used. So the scalar product becomes

\begin{equation*}
\mathcal{L}_{ij} \mathcal{L}^{ij} = 2 L_{ij}L^{ij} + 2i L_{ij}^{*}L^{ij}.
\end{equation*}

\noindent So the condition that the Lorentz transformation is singular gives $a^2 + bc = 0$ which implies first that the pure radiation conditions of Eqn.(\ref{Infinitesimal_Pure_Rad_Cond_1}) and (\ref{Infinitesimal_Pure_Rad_Cond_2}) hold and second that the scalar product $\mathcal{L}_{ij} \mathcal{L}^{ij}$ vanishes, which is equivalent to 
   
\begin{equation*}
L_{ij}L^{ij} = 0 = L_{ij} ^{*}L_{ij},
\end{equation*}

\end{appendix}


\begin{thebibliography}{10}
\bibitem{Relativity_Synge}
J.L. Synge - ``Relativity: The Special Theory'' - North Holland Publishing Company (1965)
\bibitem{Finkelstein_Paper}
D. Finkelstien - ``Past-Future Asymmetry of the Gravitational Field of a Point Particle'' - PHYSICAL REVIEW VOLUME 110, NUM B ER 4 MAY 15, 1958 - \url{http://journals.aps.org/pr/pdf/10.1103/PhysRev.110.965}
\bibitem{I_Robinson_Paper}
I. Robinson ``Spherical Gravitational Waves'' - Phys.Rev.Lett. 4 (1960) 431-432 - \url{http://journals.aps.org/prl/pdf/10.1103/PhysRevLett.4.431}
\bibitem{Hypersurfaces_Hogan_Barrabes}
P.A Hogan, C. Barrabes - ``Singular Null Hypersurfaces'' - World Scientific Pub Co Inc (April 2004)
\bibitem{Spinors_I_Penrose}
R. Penrose, W. Rindler - ``Spinors and Space-Time: Volume 1, Two-Spinor Calculus and Relativistic Fields'' - Cambridge University Press, (Feb 1987)
\bibitem{Needham_Mobius}
Tristan Needham - ``Visual Complex Analysis'' - Clarendon Press, Oxford (1997)
\bibitem{Schild_Lectures}
Various Authors - ``Space-Time and Geometry: The Alfred Schild Lectures'' - University of Texas Press (March 21, 2012)
\end{thebibliography}

\end{document}
