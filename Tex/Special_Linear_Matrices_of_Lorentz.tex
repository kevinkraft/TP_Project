\section{Special Linear Matrices of the Lorentz Transformation}\label{Special_Linear_Matrices_of_Lorentz}

In this section it is shown that there is a $2$ to $1$ correspondence between the matrices $SL(2, \mathbb{C})$ and the proper orthochronous Lorentz transformations. First it is demonstrated that there is a one to one correspondence between $2 \times 2$ Hermitian matrices and Minkowskian space-time, this is then applied to Lorentz transformations. Let $\vec{x} = (x,y,z,t)$ be the position vector of a point in Minkowskian space-time. Knowing $\vec{x}$ we can construct the following $2 \times 2$ Hermitian matrix

\begin{equation}\label{Special_Matrices_A_first}
A = 
\left( 
\begin{array}{cc}
t-z    & x + i y \\
x - iy & t+z \\
\end{array} 
\right),
\end{equation}

\noindent with $A^{\dagger}(\vec{x}) = A(\vec{x})$. This is useful as its determinant is the same as the Lorentz invariant quadratic form, up to an arbitrary sign.

\begin{equation*}
\det(A(\vec{x})) = t^2 - x^2 - y^2 - z^2.
\end{equation*}

Consider any $2 \times 2$ Hermitian matrix $H$, thus

\begin{equation*}
H = \left( \begin{array}{cc}
p & q \\
r & s \\
\end{array} \right) \text{ ,     }
H^{\dagger} = \left( \begin{array}{cc}
\bar{p} & \bar{r} \\
\bar{q} & \bar{s} \\
\end{array} \right)
\end{equation*}

\noindent It is know that $H^{\dagger}(\vec{x}) = H(\vec{x})$ so it is clear that $p = \bar{p}$ and $s = \bar{s}$ and thus $p$ and $s$ are real numbers. Also $q = \bar{r}$ and then of course $\bar{q} = r$. Hence knowing $p,q,r$ and $s$ is equivalent to knowing $4$ real numbers, one from $p$, one from $s$ and two from $q$. From these parameters the coordinates $(x,y,z,t)$ of a point in Minkowskian space-time can be constructed as

\begin{align*}
x + iy & = q = \bar{r}, \\
t - z & = p, \\
t+z  & = s.
\end{align*}

\noindent by comparing with matrix $A$ above. Hence it is true that there is a one to one correspondence between points in Minkowskian space-time and $2 \times 2$ Hermitian matrices.

Construct the following matrix

\begin{equation*} 
U = \left( 
\begin{array}{cc}
\alpha & \beta \\
\gamma & \delta \\
\end{array}
\right),
\end{equation*}

\noindent with $\alpha$, $\beta$, $\gamma$, $\delta \in \mathbb{C}$, and the condition that $\det(U) = 1$. Such matrices $U$ form a group called the special linear group, which is denoted by $SL(2, \mathbb{C})$. Given $A(\vec{x})$ consider $U A(\vec{x}) U^{\dagger}$. This is a $2 \times 2$ Hermitian matrix since

\begin{align*}
(U A(\vec{x}) U^{\dagger})^{\dagger} & =  (U^{\dagger})^{\dagger} A^{\dagger}(\vec{x}) U^{\dagger} \\
                                     & = U A(\vec{x}) U^{\dagger},
\end{align*} 

\noindent as $(U^{\dagger})^{\dagger} = U$ and $A^{\dagger} = A$. Hence there exists a point $\vec{x}' = (x', y', z', t')$ in Minkowskian space-time for which

\begin{equation}\label{SL_trans}
A(\vec{x}') = U A(\vec{x}) U^{\dagger}.
\end{equation}

Any $U$ involves 6 real parameters, 2 each from the four complex components, with the condition $\det(U) = 1$ supplying two constraints, one on the real parts and one on the imaginary parts of the components. Now calculate the determinant of the matrix in the primed frame

\begin{align*}  
\det(A(\vec{x}')) & = \det(U A(\vec{x}) U^{\dagger}), \\
                  & = (\det(U))(\det(A(\vec{x}))(\det(U^{\dagger})), \\
                  & = (\det(U))(\det(A(\vec{x}))\bar{(\det(U))}, \\
                  & = \det(A(\vec{x})).
\end{align*}

\noindent Thus we have the relation

\begin{equation*}  
{t'}^2 - {x'}^2 - {y'}^2 - {z'}^2 = {t}^2 - {x}^2 - {y}^2 - {z}^2.
\end{equation*}

\noindent Hence the transformation $\vec{x} \rightarrow \vec{x}'$ implicit in Eqn.(\ref{SL_trans}) is a Lorentz transformation. This equation describes the most general proper, orthochronous Lorentz transformation.

It is useful to calculate the matrix $U$ for some examples of Lorentz transformations. First, write Eqn.(\ref{SL_trans}) in terms of its components

\begin{align*} 
\left(
\begin{array}{cc}
t' - z' & x' + i y' \\
x' - i y' & t' + z' \\
\end{array}
\right)
& =
\left(
\begin{array}{cc}
\alpha & \beta \\
\gamma & \delta \\
\end{array}
\right)
\left(
\begin{array}{cc}
t-z & x + i y \\
x - i y & t + z   \\
\end{array}
\right)
\left(
\begin{array}{cc}
\bar{\alpha} & \bar{\gamma} \\
\bar{\beta} & \bar{\delta} \\
\end{array}
\right), \\
& = \left(
\begin{array}{cc}
\alpha & \beta \\
\gamma & \delta \\
\end{array}
\right)
\left(
\begin{array}{cc}
(t-z)\bar{\alpha} + (x + iy)\bar{\beta} & (t-z)\bar{\gamma} + (x + iy)\bar{\delta} \\
(x - iy)\bar{\alpha} + (t+z)\bar{\beta} & (x-iy)\bar{\gamma} + (t+z)\bar{\delta} \\
\end{array}
\right).
\end{align*}

\noindent Thus the relations

\begin{subequations}
\begin{gather}\label{general_coeff_equate_a}
t' - z'  = (t-z)\alpha\bar{\alpha} + (x + iy)\alpha\bar{\beta} + (x - iy)\beta\bar{alpha} + (t+z)\beta\bar{\beta},
\\\label{general_coeff_equate_b}
x' + iy'  = (t-z)\alpha\bar{\gamma} + (x + iy)\alpha\bar{\delta} + (x-iy)\beta\bar{\gamma} + (t+z)\beta\bar{\delta},
\\\label{general_coeff_equate_c}
t' + z'  = (t-z)\gamma\bar{\gamma} + (x + iy)\gamma\bar{\delta} + (x-iy)\delta\bar{\gamma} + (t+z)\delta\bar{\delta}.
\end{gather}
\end{subequations}

\noindent are obtained. Now these equations are used on some specific cases.

\subsection{Example 1: Rotational Transformation}

\noindent Find $U$ corresponding to the one parameter Lorentz transformation,

\begin{align*} 
x' & = x\cos{\theta} + y\sin{\theta}, \\
y' & = -x\sin{\theta} + y\cos{\theta}, \\
z' & = z, \\
t' & = t.
\end{align*} 

\noindent This implies that

\begin{align*}
t'-z' & = t-z \\
x'+iy' & = (x+iy) e^{-i \theta} \\
t'+z' & = t+z
\end{align*}

\noindent Equating coefficients of $x$, $y$, $z$, $t$ on both sides of Eqn.(\ref{general_coeff_equate_a}) to obtain

\begin{subequations}
\begin{gather}\label{Ex1_equate_coeffs_first_a}
\alpha \bar{\beta} + \bar{\alpha} \beta = 0, \\\label{Ex1_equate_coeffs_first_b}
i (\alpha \bar{\beta} - \bar{\alpha} \beta) = 0, \\\label{Ex1_equate_coeffs_first_c}
-\alpha \bar{\alpha} + \beta \bar{\beta} = -1, \\\label{Ex1_equate_coeffs_first_d}
\alpha \bar{\alpha} + \beta \bar{\beta} = 1. 
\end{gather}
\end{subequations}

\noindent Then Eqn.(\ref{Ex1_equate_coeffs_first_a}) and (\ref{Ex1_equate_coeffs_first_b}) imply $\alpha \bar{\beta} = 0$ so $\alpha = 0$ or $\beta = 0$. Also Eqn.(\ref{Ex1_equate_coeffs_first_c}) and (\ref{Ex1_equate_coeffs_first_d}) imply $2\beta\bar{\beta} = 0$ so $\beta = 0$ and $\alpha \bar{\alpha} = 1$ so $\alpha \neq 0$. Equating coefficients of $x$, $y$, $z$, $t$ on both sides of Eqn.(\ref{general_coeff_equate_b}) to obtain

\begin{subequations}
\begin{gather}\label{Ex1_equate_coeffs_second_a}
e^{-i\theta} = \alpha \bar{\delta} + \beta\bar{\gamma}, \\\label{Ex1_equate_coeffs_second_b}
e^{-i\theta} = \alpha \bar{\delta} - \beta\bar{\gamma},\\\label{Ex1_equate_coeffs_second_c}
0 = -\alpha\bar{\gamma} + \beta \bar{\delta},\\\label{Ex1_equate_coeffs_second_d}
0 = \alpha\bar{\gamma} + \beta \bar{\delta}. 
\end{gather}
\end{subequations}

\noindent With $\beta = 0$, Eqn.(\ref{Ex1_equate_coeffs_second_a}) and (\ref{Ex1_equate_coeffs_second_b}) imply $\alpha \bar{\delta} = e^{-i\theta}$. Also Eqn.(\ref{Ex1_equate_coeffs_second_c}) and (\ref{Ex1_equate_coeffs_second_d}) imply $\alpha \bar{\gamma} = 0$ so $\gamma = 0$ since $\alpha \neq 0$. Then using $\alpha \bar{\alpha} =1$

\begin{align*}
\alpha \bar{\delta} & = e^{-i\theta}, \\
\bar{\alpha} \alpha \bar{\delta} & = \bar{\alpha} e^{-i\theta},\\
\bar{\delta}  & =  \bar{\alpha} e^{-i\theta}.
\end{align*}

\noindent Equating coefficients of $x$, $y$, $z$, $t$ on both sides of Eqn.(\ref{general_coeff_equate_c}) to obtain

\begin{subequations}
\begin{gather}\label{Ex1_equate_coeffs_third_a}
\gamma \bar{\delta} + \delta \bar{\gamma} = 0, \\\label{Ex1_equate_coeffs_third_b}
\gamma \bar{\delta} - \delta \bar{\gamma} = 0 ,\\\label{Ex1_equate_coeffs_third_c}
-\gamma \bar{\gamma} + \delta \bar{\delta} = 1 ,\\\label{Ex1_equate_coeffs_third_d}
\gamma \bar{\gamma} + \delta \bar{\delta} = 1. 
\end{gather}
\end{subequations}

\noindent Eqn(\ref{Ex1_equate_coeffs_third_a}) and (\ref{Ex1_equate_coeffs_third_b}) are satisfied since $\gamma = 0$, this also implies that $\delta \bar{\delta} = 1$ from Eqn(\ref{Ex1_equate_coeffs_third_c}). Now use the fact that $\det{(U)} = 1$, which implies

\begin{equation*}
\alpha \delta - \beta \gamma = 1,
\end{equation*}

\noindent thus $\alpha \delta = 1$ as $\beta = 0$. Then using $\alpha \bar{\alpha} = 1$ again and $\delta \bar{\delta} = 1$

\begin{align*}
\alpha^2 e^{-\theta} & = 1, \\
\alpha^2 & = e^{-i\theta}, \\
\alpha & = \pm e^{{-i\theta}/2}, 
\end{align*}

\noindent which finally implies that $\delta = \pm e^{i\theta/2}$. Hence there are 2 matrices $U$ corresponding to the spacial rotation, namely

\begin{equation*}
U = \pm
\left(
\begin{array}{cc}
e^{-i\theta/2} & 0            \\
0              & e^{i\theta/2} \\
\end{array}
\right)
\end{equation*}

\noindent The next two examples are then very similar.

\subsection{Example 2: Standard Lorentz Transformation}\label{Special_Linear_Matrices_Example_2}

\noindent Find $U$ corresponding to the one parameter Lorentz transformation,

\begin{align}
\nonumber 
x' & = \gamma_0 (x-vt), 
\\\nonumber
t' & = \gamma_0 (t-vx), 
\\\nonumber
y' & = y, 
\\\label{Special_Matrices_Standard_Lorentz}
z' & = z, 
\end{align} 

\noindent Where $\gamma_0 = (1-v^2)^{-1/2}$.  This implies that

\begin{align*}
t'-z' & = -\gamma_0 v x + \gamma_0 t - z \\
x'+iy' & = \gamma_0 x - v \gamma_0 t + iy  \\
t'+z' & = -v \gamma_0 x + \gamma_0 t +z
\end{align*}

\noindent Equating coefficients of $x$, $y$, $z$, $t$ on both sides of Eqn.(\ref{general_coeff_equate_a}) to obtain

\begin{subequations}
\begin{gather}\label{Ex2_equate_coeffs_first_a}
\alpha \bar{\beta} + \bar{\alpha} \beta = -\gamma_0 v, \\\label{Ex2_equate_coeffs_first_b}
i (\alpha \bar{\beta} - \bar{\alpha} \beta) = 0, \\\label{Ex2_equate_coeffs_first_c}
-\alpha \bar{\alpha} + \beta \bar{\beta} = -1, \\\label{Ex2_equate_coeffs_first_d}
\alpha \bar{\alpha} + \beta \bar{\beta} = \gamma_0. 
\end{gather}
\end{subequations}

\noindent Then Eqn.(\ref{Ex2_equate_coeffs_first_a}) and (\ref{Ex2_equate_coeffs_first_b}) imply that $\alpha \bar{\beta} = \bar{\alpha} \beta $. Also Eqn.(\ref{Ex2_equate_coeffs_first_c}) and (\ref{Ex2_equate_coeffs_first_d}) imply that

\begin{align}
\label{Ex2_refer_to_beta_beta}
\beta \bar{\beta} & = \frac{\gamma_0 - 1}{2},
\\\label{Ex2_refer_to_alpha_alpha}
\alpha \bar{\alpha} & = \frac{\gamma_0 + 1}{2}. 
\end{align}

\noindent Thus $\beta$ can be written in terms of $\alpha$ using Eqn.(\ref{Ex2_equate_coeffs_first_a})

\begin{gather*}
\alpha \bar{\beta} = -\frac{\gamma_0 v}{2}, \\
\alpha \bar{\alpha} \bar{\beta} = - \bar{\alpha} \frac{\gamma_0 v}{2}, \\
\beta = -\alpha \frac{\gamma_0 v}{\gamma_0 + 1}.
\end{gather*}

\noindent Equating coefficients of $x$, $y$, $z$, $t$ on both sides of Eqn.(\ref{general_coeff_equate_b}) to obtain

\begin{subequations}
\begin{gather}\label{Ex2_equate_coeffs_second_a}
\gamma_0 = \alpha \bar{\delta} + \beta\bar{\gamma}, \\\label{Ex2_equate_coeffs_second_b}
1 = \alpha \bar{\delta} - \beta\bar{\gamma},\\\label{Ex2_equate_coeffs_second_c}
0 = -\alpha\bar{\gamma} + \beta \bar{\delta},\\\label{Ex2_equate_coeffs_second_d}
-v\gamma_0 = \alpha\bar{\gamma} + \beta \bar{\delta}. 
\end{gather}
\end{subequations}

\noindent Eqn.(\ref{Ex2_equate_coeffs_second_a}) and (\ref{Ex2_equate_coeffs_second_b}) imply that

\begin{align*}
\beta \bar{\gamma} = \frac{\gamma_0 - 1}{2}, \\
\alpha \bar{\delta} = \frac{\gamma_0 + 1}{2}.
\end{align*}

Thus $\delta$ can be written in terms of $\alpha$ by using Eqn.(\ref{Ex2_refer_to_alpha_alpha})

\begin{gather*}
\alpha \bar{\alpha} \bar{\delta} = \bar{\alpha} \frac{\gamma_0 + 1}{2}, \\
\bar{\delta} = \bar{\alpha}, \\
\delta = \alpha. 
\end{gather*}

Also, using Eqn.(\ref{Ex2_refer_to_beta_beta}) $\gamma$ can be written in terms of $\beta$

\begin{gather*}
\beta \bar{\gamma} = \frac{\gamma_0 - 1}{2},\\
\bar{\beta} \beta \bar{\gamma} = \bar{\beta} \frac{\gamma_0 - 1}{2},\\
\bar{\gamma} = \bar{\beta}, \\
\gamma = \beta.
\end{gather*}

\noindent  Now use the fact that $\det{(U)} = 1$, which implies

\begin{equation*}
\alpha \delta - \beta \gamma = 1,
\end{equation*}

\noindent thus $\alpha^2 - \beta^2 = 1$ as $\delta = \alpha$ and $\gamma = \beta$. Replace $\beta$

\begin{gather*}
\alpha^2\left( \frac{(\gamma_0 + 1)^2 - (\gamma_0 v)^2}{(\gamma_0 + 1)^2}   \right) = 1\ \\
\alpha = \pm \frac{\gamma_0 + 1}{\sqrt{(\gamma_0 + 1)^2 - (\gamma_0 v)^2}}.
\end{gather*}

\noindent Rewrite the denominator of $\alpha$ using $\gamma_0 v = \sqrt{{\gamma_0}^{2} - 1}$

\begin{align*}
(\gamma_0 + 1)^2 - (\gamma_0 v)^2 & = {\gamma_0}^{2} + 1 + 2\gamma_0 - {\gamma_0}^{2} + 1, \\
                                  & = 2(\gamma_0 + 1).
\end{align*}

\noindent So finally

\begin{equation*}
\alpha = \pm \frac{\sqrt{\gamma_0 + 1}}{2}.
\end{equation*}

\noindent Hence there are 2 matrices $U$ corresponding to this Lorentz transformation given by

\begin{equation*}
U = \pm
\left(
\begin{array}{ccc}
\frac{\sqrt{\gamma_0 + 1}}{2}   & & -\frac{\sqrt{\gamma_0 - 1}}{2} \\
 & & \\
- \frac{\sqrt{\gamma_0 - 1}}{2} & &  \frac{\sqrt{\gamma_0 + 1}}{2}  \\
\end{array}
\right)
\end{equation*}

\subsection{Example 3: Singular Lorentz Transformation}\label{Special_Linear_Matrices_Example_3}

\noindent Find $U$ corresponding to the two parameter Lorentz transformation,

\begin{align*}
t'-z' & = t-z, \\
x'+iy' & = x + iy + w(t-z), \\
t'+z' & = t+z + w(x-iy) + \bar{w} (x + iy) + w \bar{w} (t-z).
\end{align*}

\noindent Equating coefficients of $x$, $y$, $z$, $t$ on both sides of Eqn.(\ref{general_coeff_equate_a}) to obtain

\begin{subequations}
\begin{gather}\label{Ex3_equate_coeffs_first_a}
\alpha \bar{\beta} + \bar{\alpha} \beta = 0, \\\label{Ex3_equate_coeffs_first_b}
i (\alpha \bar{\beta} - \bar{\alpha} \beta) = 0, \\\label{Ex3_equate_coeffs_first_c}
-\alpha \bar{\alpha} + \beta \bar{\beta} = -1, \\\label{Ex3_equate_coeffs_first_d}
\alpha \bar{\alpha} + \beta \bar{\beta} = 1. 
\end{gather}
\end{subequations}

\noindent Then Eqn.(\ref{Ex3_equate_coeffs_first_a}) and (\ref{Ex3_equate_coeffs_first_b}) imply $\alpha \bar{\beta} = 0$ so $\alpha = 0$ or $\beta = 0$. Also Eqn.(\ref{Ex3_equate_coeffs_first_c}) and (\ref{Ex3_equate_coeffs_first_d}) imply $2\beta\bar{\beta} = 0$ so $\beta = 0$ and $\alpha \bar{\alpha} = 1$ so $\alpha \neq 0$. Equating coefficients of $x$, $y$, $z$, $t$ on both sides of Eqn.(\ref{general_coeff_equate_b}) to obtain

\begin{subequations}
\begin{gather}\label{Ex3_equate_coeffs_second_a}
1 = \alpha \bar{\delta} + \beta\bar{\gamma}, \\\label{Ex3_equate_coeffs_second_b}
1 = \alpha \bar{\delta} - \beta\bar{\gamma},\\\label{Ex3_equate_coeffs_second_c}
-w = -\alpha\bar{\gamma} + \beta \bar{\delta},\\\label{Ex3_equate_coeffs_second_d}
w= \alpha\bar{\gamma} + \beta \bar{\delta}. 
\end{gather}
\end{subequations}

\noindent With $\beta = 0$, Eqn.(\ref{Ex3_equate_coeffs_second_a}) implies $\alpha \bar{\delta} = 1$, so multiply by $\bar{\alpha}$ and use $\alpha \bar{\alpha} = 1$ to obtain $\delta = \alpha$. Also Eqn.(\ref{Ex3_equate_coeffs_second_c}) and (\ref{Ex3_equate_coeffs_second_d}) imply $\alpha \bar{\gamma} = w$ so $\gamma = \bar{w} \alpha$ since $\alpha \bar{\alpha} = 1$. Now use the fact that $\det{(U)} = 1$, which implies

\begin{equation*}
\alpha \delta - \beta \gamma = 1,
\end{equation*}

\noindent thus $\alpha^2 = 1$ as $\beta = 0$ and $\gamma = \alpha$. So $\alpha = \pm 1$, thus $\gamma = \bar{w}$. Hence there are 2 matrices $U$ corresponding to this two parameter transformation, namely

\begin{equation*}
U = \pm
\left(
\begin{array}{cc}
1       & 0 \\
\bar{w} & 1 \\
\end{array}
\right)
\end{equation*}

It is clear that there will always be two matrices, $\pm U$ corresponding to every Lorentz transformation, since if $U$ satisfies $A(\vec{x}') = U A(\vec{x}) U^{\dagger}$ then so does $-U$. Hence there is a $2$ to $1$ correspondence between the elements of $SL(2,\mathbb{C})$ and the proper orthochronous Lorentz transformation, with $A$ and $A'$ corresponding to differenent coordinates of Minkowskian space-time and $\pm U$ corresponding to proper orthochronous Lorentz transformations.  
