\section{Infinitesimal Lorentz Transformations}

There are Lorentz transformations that are small perturbations of the identity transformation and so $U \in SL(2,\mathbb{C})$ has the form

\begin{equation}\label{Infinitesimal_Infinitesimal_Lorentz_Transform_Matrix_U}
U = \pm
\left(
\begin{array}{cc}
1 + \epsilon a & \epsilon b \\
\epsilon c & 1 + \epsilon f \\
\end{array}
\right),
\end{equation}

\noindent where $a,b,c,f \in \mathbb{C}$ and $\epsilon$ is a small real parameter. Here terms of order $\epsilon^2$ will be neglected. As $U \in SL(2,\mathbb{C})$ its determinant is calculated as

\begin{equation*}
\det{(U)} = 1 + O(\epsilon^2).
\end{equation*}

\noindent Using this it is possible to obtain a relation between $f$ and $a$

\begin{eqnarray*}
(1 + \epsilon a)(1 + \epsilon f) - \epsilon^2 b c = 1 + O(\epsilon^2), \\
1 + \epsilon (a +f) = 1 + O(\epsilon^2), \\
\Rightarrow f = -a + O(\epsilon).
\end{eqnarray*}

\noindent Hence 

\begin{equation*}
U =
\left(
\begin{array}{cc}
1 + \epsilon a & \epsilon b \\
\epsilon c & 1 - \epsilon a \\
\end{array}
\right),
\end{equation*}

Now the explicit infinitesimal Lorentz transformations are calculated as in section \ref{Special_Linear_Matrices_of_Lorentz}, by substituting $U$ into

\begin{equation*}
A(\vec{x}') = U A(\vec{x}) U^{\dagger}.
\end{equation*}

\noindent Now writing this out in component form to obtain

\begin{eqnarray*}
\left(
\begin{array}{cc}
t'-z' & x' + iy' \\
x' + iy' & t'+z' \\
\end{array}
\right)
=
\left(
\begin{array}{cc}
1 + \epsilon a & \epsilon b \\
\epsilon c & 1 - \epsilon a \\
\end{array}
\right)
\left(
\begin{array}{cc}
t-z & x + iy \\
x - iy & t+z \\
\end{array}
\right)
\left(
\begin{array}{cc}
1 + \epsilon  \bar{a} & \epsilon \bar{c} \\
\epsilon \bar{b} & 1 - \epsilon \bar{a} \\
\end{array}
\right), \\
=
\left(
\begin{array}{cc}
1 + \epsilon a & \epsilon b \\
\epsilon c & 1 - \epsilon a \\
\end{array}
\right)
\left(
\begin{array}{cc}
(t-z)(1 + \epsilon \bar{a})+\epsilon \bar{b}(x + iy) & (t-z)\epsilon \bar{c}+ (1 - \epsilon \bar{a})(x + iy) \\
(x - iy)(1 + \epsilon \bar{a}) +\epsilon \bar{b} (t+z) & (x - iy)\epsilon \bar{c}+(1 - \epsilon \bar{a})(t+z) \\
\end{array}
\right).
\end{eqnarray*}

\noindent This then implies the three relations

\begin{eqnarray}\label{Infinitesimal_Lorentz_Transform_1}
t'-z' = t-z + \epsilon(a + \bar{a})(t-z) + \epsilon(b + \bar{b})x + i\epsilon(\bar{b} - b)y + O(\epsilon^2), \\\label{Infinitesimal_Lorentz_Transform_2}
t'+z' = t+ z - \epsilon(a + \bar{a})(t+z) + \epsilon (c + \bar{c})x + i \epsilon(c-\bar{c})y + O(\epsilon^2), \\\label{Infinitesimal_Lorentz_Transform_3}
x'+iy' = x+iy + \epsilon(a-\bar{a})(x+iy) + \epsilon(b + \bar{c})t + \epsilon (b-\bar{c})z + O(\epsilon^2).
\end{eqnarray}

\noindent As $a,b,c \in \mathbb{C}$, set 

\begin{equation*}
a = a_1 +ia_2 \text{,  } b = b_1 +ib_2 \text{,  } c = c_1 +ic_2 \text{.} 
\end{equation*}

\noindent Then subbing these into the above equations, eliminating $t$ and $z$ respectively from Eqn.(\ref{Infinitesimal_Lorentz_Transform_1}) and (\ref{Infinitesimal_Lorentz_Transform_2}) and taking real and imaginary parts of Eqn.(\ref{Infinitesimal_Lorentz_Transform_3}) to obtain

\begin{equation}\label{infinitesimal_Matrix_component_wise}
\left(
\begin{array}{c}
x' \\
y'\\
z'\\
t'\\
\end{array}
\right)
=
\left(
\begin{array}{c}
x \\
y\\
z\\
t\\
\end{array}
\right)
+ \epsilon
\left(
\begin{array}{cccc}
0            & -2a_2        & (b_1 - c_1) & (b_1+c_1)\\
2a_2         & 0            & (b_2+c_2)   & (b_2 - c_2) \\
-(b_1 - c_1) & -(b_2 - c_2) & 0           & -2a_1 \\
(b_1 + c_1)  & (b_2-c_2)    & -2a_1       & 0 \\
\end{array}
\right)
\left(
\begin{array}{c}
x \\
y\\
z\\
t\\
\end{array}
\right)
+ O(\epsilon^2).
\end{equation}

\noindent The above $4 \times 4$ matrix will be denoted as $\tensor{L}{^i_j}$, so that Eqn.(\ref{infinitesimal_Matrix_component_wise}) can be written simply as 

\begin{equation}\label{Infinitesimal_Infinitesimal_Lorentz_Transformation}
\bar{x}^i = x^i + \epsilon \tensor{L}{^{i}_{j}} x^j + O(\epsilon^2).
\end{equation}

\noindent Where $\bar{x}^i = (x',y',z',t')$. It is also necessary to check that the Lorentz invariance of the quadratic form still holds. 

\begin{eqnarray*}
{x'}^2 + {y'}^2 + {z'}^2 - {t'}^2 & = x^2 + y^2 + z^2 - t^2 - 4\epsilon a_2 x y + 2 \epsilon(b_1 + c_1)xt \\
                                  & + 2\epsilon (b_1 - c_1)xz + 4 \epsilon a_2 yx + 2\epsilon (b_2 - c_2)yt \\
                                  & + 2 \epsilon (b_2 + c_2)yz - 4\epsilon a_1 zt + 2 \epsilon (c_1 - b_1)zx \\
                                  & -2 \epsilon (c_2 + b_2)zy + 4 \epsilon a_1 tz - 2 \epsilon (c_1 + b_1)tx \\
                                  & -2\epsilon (b_2 - c_2) ty + O(\epsilon^2) \\
                                  & = x^2 + y^2 + z^2 - t^2 + O(\epsilon^2)
\end{eqnarray*}

\noindent Hence this transforamtion is still a Lorentz Transformation if we neglect terms of order $\epsilon^2$.

Consider the time-like world line (SEE FIG pg 5:3)of a particle in Minkowkian space-time $x^i = x^i(s)$. If $s$ is arc length or proper time then $v^i(s) = \frac{dx^i}{ds}$ is the unit tangent(NOT SURE WHY???) vector field. It is clear that $v^i(s)$ must be time-like as $x^i(s)$ is time-like, thus

\begin{equation*}
\eta_{ij} v^i v^j = -1. 
\end{equation*}

\noindent Where $\eta_{ij} - \text{diag}(1,1,1,-1)$ is the metric of minkowskiam space-time. This implies that 

\begin{equation*}
(v^1)^2  + (v^2)^2 + (v^3)^2  - (v^4)^2 = -1.
\end{equation*}

Now consider taking a step along the world line of the particle. Define $\bar{s} = s + \alpha$, where $\alpha$ is some real parameter, so that $v^i (s+\alpha) \vcentcolon = \bar{v}^i(\bar{s})$. Hence we also have

\begin{equation*}
({\bar{v}^1})^2  + ({\bar{v}^2})^2 + ({\bar{v}^3})^2  - ({\bar{v}^4})^2 = -1,
\end{equation*}

\noindent and so $v^i(s)$ and $\bar{v}^i(\bar{s})$ are related by a Lorentz transformation. In particlular $v^i(s+\epsilon)$ and $v^i(s)$ are related by an infinitesimal Lorentz Transformation given by Eqn.(\ref{Infinitesimal_Infinitesimal_Lorentz_Transformation}),

\begin{equation}
v^i(s+\epsilon) = v^i(s) = \epsilon \tensor{L}{^{i}_{j}}(s)v^j (s) + O(\epsilon^2).
\end{equation}

\noindent Rearranging to obtain

\begin{equation}
\frac{v^i(s+\epsilon) - v^i(s)}{\epsilon} = \tensor{L}{^{i}_{j}}(s)v^j (s) + O(\epsilon).
\end{equation}

\noindent Now taking the limit as the infinitesimal step, $\epsilon$ goes to zero to obtain a continuous differentiable equation,

\begin{equation}\label{Infinitesimal_DE_interms_v}
\frac{dv^i}{ds} = \tensor{L}{^{i}_{j}}(s) v^j (s).
\end{equation}

\noindent This equation determines the trajectory of the particle through Minkowskian space-time. In terms of $x$ this is equivalent to

\begin{equation*}
\frac{d^2 x^i}{ds^2} = \tensor{L}{^{i}_{j}}(s) \frac{d x^j}{ds}.
\end{equation*}

It is interesting to write these equations in terms of the particles $3$-velocity given by

\begin{equation*}
\vec{u} = \left( \frac{dx}{dt}, \frac{dy}{dt}, \frac{dz}{dt} \right).
\end{equation*}

\noindent Start by using the chain rule on $v^i$,

\begin{equation*}\label{Infinitesimal_Chain_Rule}
v^i = \frac{dx^i}{ds} = \frac{dx^i}{dt} \frac{dt}{ds} = \left(\frac{dx}{dt},\frac{dy}{dt},\frac{dz}{dt},1\right) \frac{dt}{ds}.
\end{equation*}

\noindent Now determine the first integral of $v^i$, which is equal to $-1$ as $v^i$ is time-like,

\begin{equation*} 
-1 = \eta_{ij} v^i v^j =  \left\{ \left( \frac{d}{dt} \right)^2  + \left( \frac{d}{dt} \right)^2  + \left( \frac{d}{dt} \right)^2 - 1  \right\} \left( \frac{dt}{ds} \right)^2,
\end{equation*} 

\noindent as this is just the scalar product in Minkowskian space-time. Therefore (NOT SURE WHERE THIS COMES FROM)

\begin{equation*}
\frac{dt}{ds} = \gamma (s) \vcentcolon = (1-u^2)^{-1/2},
\end{equation*}

\noindent where $u = |\vec{u}| = \sqrt{\vec{u} \cdot \vec{u}}$. Thus from Eqn.(\ref{Infinitesimal_Chain_Rule})

\begin{equation}\label{infinitesimal_v_interms_gamma}
v^i = \gamma(u) (\vec{u}, 1)
\end{equation}

\noindent It is now conveinient to display Eqn.(\ref{Infinitesimal_DE_interms_v}) as two equations denoting the spacial part and the temproal part, in terms of $\gamma$ and $u$. Again using the chain rule to obtain

\begin{equation*} 
\frac{dt}{ds} \frac{dv^i}{dt} = \tensor{L}{^{i}_{j}}v^j. 
\end{equation*} 

\noindent This then implies that 

\begin{eqnarray}\label{Infinitesimal_gamma_u_1}
\gamma(u) \frac{d}{dt} (\gamma(u) u^{\alpha}) = \tensor{L}{^{\alpha}_{j}} v^j, \\\nonumber
\gamma(u) \frac{d}{dt} \gamma(u) = \tensor{L}{^{4}_{j}} v^j,
\end{eqnarray}

\noindent as $v^i = \gamma(u)(\vec{u},1)$. Here we have used the usual convention that greek indices denote the sum over the spacial indices only, thus $\alpha = 1,2,3$. Now Eqn.(\ref{infinitesimal_v_interms_gamma}) can be used to rewrite the $\tensor{L}{^{i}_{j}}$ coefficients to get

\begin{eqnarray}\label{Infinitesimal_gamma_u_2}
\tensor{L}{^{\alpha}_{j}} v^j = \gamma(u) (\tensor{L}{^{\alpha}_{\beta}} u^{\beta} + \tensor{L}{^{\alpha}_4}) \\ \nonumber
\tensor{L}{^{4}_{j}} v^j = \gamma(u) (\tensor{L}{^{4}_{\alpha}} u^{\alpha0})
\end{eqnarray} 

\noindent where $\tensor{L}{^{4}_{4}} = 0$ from Eqn.(\ref{infinitesimal_Matrix_component_wise}). Putting together Eqns.(\ref{Infinitesimal_gamma_u_1}) and (\ref{Infinitesimal_gamma_u_2}) to obtain differential equations for the spacial and temporal coordinates in terms of the particles $3$-velocity,

\begin{eqnarray*} 
\frac{d}{dt} (\gamma(u) u^{\alpha}) = \tensor{L}{^{\alpha}_{\beta}} u^{\beta} + \tensor{L}{^{\alpha}_{4}}, \\
\frac{d\gamma(u)}{dt} = \tensor{L}{^{4}_{\alpha}} u^{\alpha}.
\end{eqnarray*} 

\noindent These can be written explicitely as four equations

\begin{eqnarray}\label{Infinitesimal_gamma_u_explicit_1}
\frac{d}{dt} (\gamma(u) u^{(1)}) = -2a_2u^{(2)} + (b_1 - c_1)u^{(3)} + b_1 + c_1, \\ \label{Infinitesimal_gamma_u_explicit_2}
\frac{d}{dt} (\gamma(u) u^{(2)}) = 2a_2 u^{(1)} + (b_2 + c_2) u^{(3)} + b_2 - c_2,\\ \label{Infinitesimal_gamma_u_explicit_3}
\frac{d}{dt} (\gamma(u) u^{(3)}) = -(b_1 - c_1) u^{(1)} - (b_2 + c_2 )u^{(2)} - 2a_1,\\ \label{Infinitesimal_gamma_u_explicit_4}
\frac{d\gamma(u)}{dt} = (b_1 + c_1)u^{(1)} + (b_2 - c_2) u^{(2)} - 2a_1 u^{(3)}.
\end{eqnarray}

Now define the $3$-vectors $\vec{P}$ and $\vec{Q}$ such that

\begin{eqnarray*}
\vec{P} = (b_1+c_1,b_2-c_2,-2a_1),
\vec{Q} = (b_2 + c_2, -(b_1 - c_1),-2a_2).
\end{eqnarray*}

It is clear that Eqns.(\ref{Infinitesimal_gamma_u_explicit_1})-(\ref{Infinitesimal_gamma_u_explicit_4}) can be written in terms of $\vec{P}$ and $\vec{Q}$ as follows,

\begin{eqnarray}\label{Infinitesimal_like_Lorentz_force_1}
\frac{d}{dt} (\gamma(u)\vec{u}) = \vec{P} + \vec{u} \times \vec{Q}, \\ \label{Infinitesimal_like_Lorentz_force_2}
\frac{d\gamma}{dt} = \vec{u} \cdot \vec{P}.
\end{eqnarray}

\noindent Note that these expressions look remarkably like the Lorentz force in electromagnetism. It is easily shown that Eqn.(\ref{Infinitesimal_like_Lorentz_force_1}) implies Eqn.(\ref{Infinitesimal_like_Lorentz_force_2}). To see this, first take the scalar product of Eqn.(\ref{Infinitesimal_like_Lorentz_force_1}) with $\vec{U}$.

\begin{eqnarray}\label{Infinitesimal_1_imples_2_calc_1}
\vec{u} \cdot \frac{d}{dt} (\gamma(u)\vec{u}) = \vec{u} \cdot \vec{P} + \vec{u} \cdot (\vec{u} \times \vec{Q}), \\ \label{Infinitesimal_1_imples_2_calc_2}
\gamma \vec{u} \frac{d\vec{u}}{dt} + \vec{u} \cdot \vec{u} \frac{d\gamma}{dt} = \vec{u} \cdot \vec{P},
\end{eqnarray}

\noindent by using the product rule and as the scalar product of the cross product with a repeated vector is zero in the third term. The quantity $\gamma$ is known in terms of $u$, so it is possible to write the derivative in the first term as a derivative of $\gamma$ as follows,

\begin{eqnarray*} 
\gamma^{-2} = 1 - u^{2} = 1- \vec{u} \cdot \vec{u}, \\
\Rightarrow -2 \gamma^{-3} \frac{d\gamma}{dt} = - 2 \vec{u} \cdot \frac{d \vec{u}}{dt}.
\end{eqnarray*}

\noindent Subbing this result back into Eqn.(\ref{Infinitesimal_1_imples_2_calc_2}) to obtain

\begin{eqnarray*} 
\gamma \gamma^{-3} \frac{d\gamma}{dt} + u^2 \frac{d\gamma}{dt} = \vec{u} \cdot \vec{P}, \\
(\gamma^{-2} + u^2) \frac{d\gamma}{dt} = \vec{u} \cdot \vec{P}.
\end{eqnarray*} 

Therefore

\begin{equation*}
\frac{d\gamma}{dt} =  \vec{u} \cdot \vec{P},
\end{equation*}

and so it is shown that Eqn.(\ref{Infinitesimal_like_Lorentz_force_2}) is a generalization of Eqn.(\ref{Infinitesimal_like_Lorentz_force_1}) and contains no new information. 

The dependance of the $3$-force acting on a particle as shown by Eqn.(\ref{Infinitesimal_like_Lorentz_force_1}), depends in general on the particles $3$-velocity $\vec{u}$ in a special way, in order to be compatible with Special Relativity. Thus in particular \textit{the Lorentz $3$-force acting on a particle of rest mass $m$ and charge $q$ must depend upon $\vec{u}$ as in Eqn.(\ref{Infinitesimal_like_Lorentz_force_1}) to be compatible with special Relativity.} So the Lorentz force of electromagnetism is a special case of the a charged particle moving through Minkowskian space-time along a world-line of infiniteimal Lorentz transformations. In this case, make the identifications

\begin{equation}\label{Infinitesimal_P_Q_interms_E_B} 
\vec{P} = \frac{q}{m} \vec{E} \text{, and  } \vec{Q} = \frac{q}{m}\vec{B},
\end{equation} 

\noindent where $\vec{E}$ is the external electric field and $\vec{B}$ is the external magnetic field in which the particle is moving. Then Eqn.(\ref{Infinitesimal_like_Lorentz_force_1}) takes the familiar form

\begin{equation*}
m \frac{d}{dt} (\gamma(u) \vec{u}) = q(\vec{E} + \vec{u} \times \vec{B}).
\end{equation*}

\noindent Or in the case of a slow moving particle $\gamma \approx 1$ and 

\begin{equation*}  
m \vec{a} = q (\vec{E} + \vec{u} \times \vec{B}).
\end{equation*}  

\subsection{Fractional Linear Transformations of the Infinitesimal Linear Transformation}

Recall that the fractional linear transformation constructed in section (\ref{Section_Stereographic_Extended_Complex}) had a one to one correspondence with proper orthochronous Lorentz transforamtions, and the fixed points of the fractional transformation corresponded to null directions of the Lorentz transformation. As in Eqn.(\ref{Extended_Complex_Fractional_Linear_Transformation}), section (\ref{Section_Stereographic_Extended_Complex}) construct the fractional linear transformation of the special linear $(SL(2, \mathbb(C)))$ matrix $U$ for the infinitesimal Lorentz transformation given in Eqn.(\ref{Infinitesimal_Infinitesimal_Lorentz_Transform_Matrix_U}). It is found to be

\begin{equation*}   
\zeta' = \frac{\zeta + \epsilon(\bar{c} - \bar{a}\zeta) + O(\epsilon^2)}{1 + \epsilon(\bar{a} + \bar{b} \zeta) + O(\epsilon^2)}.
\end{equation*}

\noindent Then the fixed points are given when $\zeta' = \zeta$, which implies,

\begin{eqnarray}\nonumber
\epsilon \bar{b} \zeta^2 + (\epsilon \bar{a} + \epsilon \bar{a})\zeta - \epsilon \bar{c} = O(\epsilon^2), \\ \label{Infinitesimal_fixed_point_quadratic}
\Rightarrow \bar{b} \zeta^2 + 2 \bar{a} \zeta - \bar{c} = O(\epsilon).
\end{eqnarray}

\noindent Of interest here are the singular Lorentx transforamtions, so it is required that the roots of this quadratic are the same. Thus the usual discriminant is set to zero,

\begin{equation*} 
4 {\bar{a}}^2 + 4 \bar{b} \bar{c} = 0.
\end{equation*} 

\noindent Therefore,

\begin{equation*}
{\bar{a}}^2 +  \bar{b} \bar{c} \Leftrightarrow a^2 + bc = 0.
\end{equation*}

\noindent Write these equations out explicly and equate real and imaginary coefficients to obtain

\begin{eqnarray}\label{Infinitesimal_discriminant_relation_1}
a_1^2 = a_2^2 + b_1 c_1 - b_2 c_2 = 0, \\\label{Infinitesimal_discriminant_relation_2}
2a_1 a_2 + b_2 c_1 + b_1 c_2 = 0.
\end{eqnarray}

It is interesting to write these equations in terms of the electric and magnetic vectors, namely $\vec{E} = (E^1, E^2, E^3)$ and $\vec{B} = (B^1,B^2,B^3)$. The relation between the $a$,$b$ and $c$ and the $\vec{B}$ and $\vec{E}$ coefficients comes from Eqn.(\ref{Infinitesimal_P_Q_interms_E_B}), where the factor $q/m$ has been suppressed for conveinience. 

\begin{eqnarray}\label{Infinitesimal_abc_interms_EB_1}
a_1 = -\frac{1}{2} E^3 \text{,  } b_2 = \frac{1}{2}(E^2 + B^1) \text{,  } c_1 = \frac{1}{2} (E^1 + B^2) \text{,  } \\\label{Infinitesimal_abc_interms_EB_2}
a_2 = -\frac{1}{2} B^3 \text{,  } b_1 = \frac{1}{2} (E^1 - B^2) \text{,  } c_2  = \frac{1}{2} (B^1 - E^2). 
\end{eqnarray}

\noindent So Eqn.(\ref{Infinitesimal_discriminant_relation_1}) implies

\begin{eqnarray*}
\frac{1}{4} {(E^3)}^2 - \frac{1}{4} {(B^3)}^2 + \frac{1}{4} ({(E^1)}^2 - {(B^2)}^2) + \frac{1}{4} ({(E^2)}^2 - {(B^1)}^2) = 0,
\Rightarrow {(E^1)}^2 + {(E^2)}^2 + {(E^3)}^2 = {(B^1)}^2 + {(B^2)}^2 + {(B^3)}^2.
\end{eqnarray*}

\noindent Thus it is clear that

\begin{equation}\label{Infinitesimal_Pure_Rad_Cond_1}
{|\vec{E}|}^2 = {|\vec{B}|}^2
\end{equation}

\noindent Similarly, Eqn.(\ref{Infinitesimal_discriminant_relation_2}) implies

\begin{eqnarray*}
\frac{1}{2} E^3 B^3 + \frac{1}{4} (E^1E^2 + E^1B^1 + E^2B^2 + B^1B^2) \\
 +  \frac{1}{4} (-E^1E^2 + E^1B^1 + E^2B^2 - B^1B^2) = 0, \\
\Rightarrow E^3B^3 + E^1B^1 + E^2B^2 = 0.
\end{eqnarray*}

\noindent So it is shown that

\begin{equation}\label{Infinitesimal_Pure_Rad_Cond_2}
\vec{E} \cdot \vec{B} = 0.
\end{equation}

\noindent The above Eqns.(\ref{Infinitesimal_Pure_Rad_Cond_1}) and (\ref{Infinitesimal_Pure_Rad_Cond_2}) are the (Lorentz invariant) conditions that the electromagnetic field in which the charged particle is moving is a pure radiation field. Thus in conclusion, \textit{if the world line of the charged particle is generated by infinitesimal singular Lorentz transformations then the particles moving in a pure radiation electromagnetic field.} (CASE WHERE ITS A PLAVE WAVE WORTH DOING???)

\subsection{Pure Radiation Field Conditions in Minkowskian Space-Time}

Eqns.(\ref{Infinitesimal_Pure_Rad_Cond_1}) and (\ref{Infinitesimal_Pure_Rad_Cond_2}) are the pure radiation field conditions in physical space, ${\mathbb{R}}^2$. It is also interesting to see what form these equations take in Minkowskian space-time. To do this, solve the quadratic equation in Eqn.(\ref{Infinitesimal_fixed_point_quadratic}) for the case where the roots coincide, to find the single fixed point of the system. It is clear that 

\begin{equation*}
\zeta = - \frac{\bar{a}}{\bar{b}},
\end{equation*}

\noindent is the fixed point. Now determine the corresponding null direction $k^i$. (HOW DID WE DO THIS???)

\begin{equation*}
k^i = (\bar{\zeta} + \zeta,i(\bar{\zeta} - \zeta),\bar{\zeta}\zeta - 1,\bar{\zeta}\zeta + 1).
\end{equation*}

\noindent Now relate $k^i$ to $\tensor{L}{_i_j} = \eta_{ij} \tensor{L}{^k_j}$. From the relations in (\ref{Infinitesimal_abc_interms_EB_1}) and (\ref{Infinitesimal_abc_interms_EB_2}) it is clear that

\begin{equation*}  
L_{ij} = 
\frac{q}{m}
\left(
\begin{array}{cccc}
0    & B^3  & -B^2 & E^1 \\
-B^3 & 0    & B^1  & E^2 \\
B^2  & -B^1 & 0    & E^3 \\
-E^1 & -E^2 & -E^3 & 0   \\
\end{array}
\right)
=
-(L_{ij}).
\end{equation*}

\noindent The dual of this quantity is defined by

\begin{equation*}
^*L_{ij} = \frac{1}{2} \epsilon_{ijkl} L^{kl},
\end{equation*}

\noindent where $\epsilon_{ijkl}$ is the Levi-Civita permutation symbol in 4 dimensions. 
  










