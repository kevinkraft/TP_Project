\section{Infinitesimal Lorentz Transformations}

There are Lorentz transformations that are small perturbations of the identity transformation and so $U \in SL(2,\mathbb{C})$ has the form

\begin{equation*}
U = \pm
\left(
\begin{array}{cc}
1 + \epsilon a & \epsilon b \\
\epsilon c & 1 + \epsilon f \\
\end{array}
\right),
\end{equation*}

\noindent where $a,b,c,f \in \mathbb{C}$ and $\epsilon$ is a small real parameter. Here terms of order $\epsilon^2$ will be neglected. As $U \in SL(2,\mathbb{C})$ its determinant is calculated as

\begin{equation*}
\det{(U)} = 1 + O(\epsilon^2).
\end{equation*}

\noindent Using this it is possible to obtain a relation between $f$ and $a$

\begin{eqnarray*}
(1 + \epsilon a)(1 + \epsilon f) - \epsilon^2 b c = 1 + O(\epsilon^2), \\
1 + \epsilon (a +f) = 1 + O(\epsilon^2), \\
\Rightarrow f = -a + O(\epsilon).
\end{eqnarray*}

\noindent Hence 

\begin{equation*}
U =
\left(
\begin{array}{cc}
1 + \epsilon a & \epsilon b \\
\epsilon c & 1 - \epsilon a \\
\end{array}
\right),
\end{equation*}

Now the explicit infinitesimal Lorentz transformations are calculated as in section \ref{Special_Linear_Matrices_of_Lorentz}, by substituting $U$ into

\begin{equation*}
A(\vec{x}') = U A(\vec{x}) U^{\dagger}.
\end{equation*}

\noindent Now writing this out in component form to obtain

\begin{eqnarray*}
\left(
\begin{array}{cc}
t'-z' & x' + iy' \\
x' + iy' & t'+z' \\
\end{array}
\right)
=
\left(
\begin{array}{cc}
1 + \epsilon a & \epsilon b \\
\epsilon c & 1 - \epsilon a \\
\end{array}
\right)
\left(
\begin{array}{cc}
t-z & x + iy \\
x - iy & t+z \\
\end{array}
\right)
\left(
\begin{array}{cc}
1 + \epsilon  \bar{a} & \epsilon \bar{c} \\
\epsilon \bar{b} & 1 - \epsilon \bar{a} \\
\end{array}
\right), \\
=
\left(
\begin{array}{cc}
1 + \epsilon a & \epsilon b \\
\epsilon c & 1 - \epsilon a \\
\end{array}
\right)
\left(
\begin{array}{cc}
(t-z)(1 + \epsilon \bar{a})+\epsilon \bar{b}(x + iy) & (t-z)\epsilon \bar{c}+ (1 - \epsilon \bar{a})(x + iy) \\
(x - iy)(1 + \epsilon \bar{a}) +\epsilon \bar{b} (t+z) & (x - iy)\epsilon \bar{c}+(1 - \epsilon \bar{a})(t+z) \\
\end{array}
\right).
\end{eqnarray*}

\noindent This then implies the three relations

\begin{eqnarray}\label{Infinitesimal_Lorentz_Transform_1}
t'-z' = t-z + \epsilon(a + \bar{a})(t-z) + \epsilon(b + \bar{b})x + i\epsilon(\bar{b} - b)y + O(\epsilon^2), \\\label{Infinitesimal_Lorentz_Transform_2}
t'+z' = t+ z - \epsilon(a + \bar{a})(t+z) + \epsilon (c + \bar{c})x + i \epsilon(c-\bar{c})y + O(\epsilon^2), \\\label{Infinitesimal_Lorentz_Transform_3}
x'+iy' = x+iy + \epsilon(a-\bar{a})(x+iy) + \epsilon(b + \bar{c})t + \epsilon (b-\bar{c})z + O(\epsilon^2).
\end{eqnarray}

\noindent As $a,b,c \in \mathhbb{C}$, set 

\begin{equation*}
a = a_1 +ia_2 \text{,  } b = b_1 +ib_2 \text{,  } c = c_1 +ic_2 \text{.} 
\end{equation*}

\noindent Then subbing these into the above equations, eliminating $t$ and $z$ respectively from Eqn.(\ref{Infinitesimal_Lorentz_Transform_1}) and (\ref{Infinitesimal_Lorentz_Transform_2}) and taking real and imaginary parts of Eqn.(\ref{Infinitesimal_Lorentz_Transform_3}) to obtain

\begin{equation*}
\left(
\begin{array}{c}
x' \\
y'\\
z'\\
t'\\
\end{array}
\right)
=
\left(
\begin{array}{c}
x \\
y\\
z\\
t\\
\end{array}
\right)
+ \epsilon
\left(
\begin{array}{cccc}
0            & -2a_2        & (b_1 - c_1) & (b_1+c_1)\\
2a_2         & 0            & (b_2+c_2)   & (b_2 - c_2) \\
-(b_1 - c_1) & -(b_2 - c_2) & 0           & -2a_1 \\

\end{array}
\right)
\left(
\begin{array}{c}

\end{array}
\right)
+ O(\epsilon^2)
\end{equation*}

   

