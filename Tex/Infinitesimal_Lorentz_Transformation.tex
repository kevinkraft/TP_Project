\section{Infinitesimal Lorentz Transformations}

There are Lorentz transformations that are small perturbations of the identity transformation and so $U \in SL(2,\mathbb{C})$ has the form

\begin{equation*}
U = \pm
\left(
\begin{array}{cc}
1 + \epsilon a & \epsilon b \\
\epsilon c & 1 + \epsilon f \\
\end{array}
\right),
\end{equation*}

\noindent where $a,b,c,f \in \mathbb{C}$ and $\epsilon$ is a small real parameter. Here terms of order $\epsilon^2$ will be neglected. As $U \in SL(2,\mathbb{C})$ its determinant is calculated as

\begin{equation*}
\det{(U)} = 1 + O(\epsilon^2).
\end{equation*}

\noindent Using this it is possible to obtain a relation between $f$ and $a$

\begin{eqnarray*}
(1 + \epsilon a)(1 + \epsilon f) - \epsilon^2 b c = 1 + O(\epsilon^2), \\
1 + \epsilon (a +f) = 1 + O(\epsilon^2), \\
\Rightarrow f = -a + O(\epsilon).
\end{eqnarray*}

\noindent Hence 

\begin{equation*}
U =
\left(
\begin{array}{cc}
1 + \epsilon a & \epsilon b \\
\epsilon c & 1 - \epsilon a \\
\end{array}
\right),
\end{equation*}

Now the explicit infinitesimal Lorentz transformations are calculated as in section \ref{Special_Linear_Matrices_of_Lorentz}, by substituting $U$ into

\begin{equation*}
A(\vec{x}') = U A(\vec{x}) U^{\dagger}.
\end{equation*}

\noindent Now writing this out in component form to obtain

\begin{eqnarray*}
\left(
\begin{array}{cc}
t'-z' & x' + iy' \\
x' + iy' & t'+z' \\
\end{array}
\right)
=
\left(
\begin{array}{cc}
1 + \epsilon a & \epsilon b \\
\epsilon c & 1 - \epsilon a \\
\end{array}
\right)
\left(
\begin{array}{cc}
t-z & x + iy \\
x - iy & t+z \\
\end{array}
\right)
\left(
\begin{array}{cc}
1 + \epsilon  \bar{a} & \epsilon \bar{c} \\
\epsilon \bar{b} & 1 - \epsilon \bar{a} \\
\end{array}
\right), \\
=
\left(
\begin{array}{cc}
1 + \epsilon a & \epsilon b \\
\epsilon c & 1 - \epsilon a \\
\end{array}
\right)
\left(
\begin{array}{cc}
(t-z)(1 + \epsilon \bar{a})+\epsilon \bar{b}(x + iy) & (t-z)\epsilon \bar{c}+ (1 - \epsilon \bar{a})(x + iy) \\
(x - iy)(1 + \epsilon \bar{a}) +\epsilon \bar{b} (t+z) & (x - iy)\epsilon \bar{c}+(1 - \epsilon \bar{a})(t+z) \\
\end{array}
\right).
\end{eqnarray*}

\noindent This then implies the three relations

\begin{eqnarray}\label{Infinitesimal_Lorentz_Transform_1}
t'-z' = t-z + \epsilon(a + \bar{a})(t-z) + \epsilon(b + \bar{b})x + i\epsilon(\bar{b} - b)y + O(\epsilon^2), \\\label{Infinitesimal_Lorentz_Transform_2}
t'+z' = t+ z - \epsilon(a + \bar{a})(t+z) + \epsilon (c + \bar{c})x + i \epsilon(c-\bar{c})y + O(\epsilon^2), \\\label{Infinitesimal_Lorentz_Transform_3}
x'+iy' = x+iy + \epsilon(a-\bar{a})(x+iy) + \epsilon(b + \bar{c})t + \epsilon (b-\bar{c})z + O(\epsilon^2).
\end{eqnarray}

\noindent As $a,b,c \in \mathbb{C}$, set 

\begin{equation*}
a = a_1 +ia_2 \text{,  } b = b_1 +ib_2 \text{,  } c = c_1 +ic_2 \text{.} 
\end{equation*}

\noindent Then subbing these into the above equations, eliminating $t$ and $z$ respectively from Eqn.(\ref{Infinitesimal_Lorentz_Transform_1}) and (\ref{Infinitesimal_Lorentz_Transform_2}) and taking real and imaginary parts of Eqn.(\ref{Infinitesimal_Lorentz_Transform_3}) to obtain

\begin{equation}\label{infinitesimal_Matrix_component_wise}
\left(
\begin{array}{c}
x' \\
y'\\
z'\\
t'\\
\end{array}
\right)
=
\left(
\begin{array}{c}
x \\
y\\
z\\
t\\
\end{array}
\right)
+ \epsilon
\left(
\begin{array}{cccc}
0            & -2a_2        & (b_1 - c_1) & (b_1+c_1)\\
2a_2         & 0            & (b_2+c_2)   & (b_2 - c_2) \\
-(b_1 - c_1) & -(b_2 - c_2) & 0           & -2a_1 \\
(b_1 + c_1)  & (b_2-c_2)    & -2a_1       & 0 \\
\end{array}
\right)
\left(
\begin{array}{c}
x \\
y\\
z\\
t\\
\end{array}
\right)
+ O(\epsilon^2).
\end{equation}

\noindent The above $4 \times 4$ matrix will be denoted as $\tensor{L}{^i_j}$, so that Eqn.(\ref{infinitesimal_Matrix_component_wise}) can be written simply as 

\begin{equation}\label{Infinitesimal_Infinitesimal_Lorentz_Transformation}
\bar{x}^i = x^i + \epsilon \tensor{L}{^{i}_{j}} x^j + O(\epsilon^2).
\end{equation}

\noindent Where $\bar{x}^i = (x',y',z',t')$. It is also necessary to check that the Lorentz invariance of the quadratic form still holds. 

\begin{eqnarray*}
{x'}^2 + {y'}^2 + {z'}^2 - {t'}^2 & = x^2 + y^2 + z^2 - t^2 - 4\epsilon a_2 x y + 2 \epsilon(b_1 + c_1)xt \\
                                  & + 2\epsilon (b_1 - c_1)xz + 4 \epsilon a_2 yx + 2\epsilon (b_2 - c_2)yt \\
                                  & + 2 \epsilon (b_2 + c_2)yz - 4\epsilon a_1 zt + 2 \epsilon (c_1 - b_1)zx \\
                                  & -2 \epsilon (c_2 + b_2)zy + 4 \epsilon a_1 tz - 2 \epsilon (c_1 + b_1)tx \\
                                  & -2\epsilon (b_2 - c_2) ty + O(\epsilon^2) \\
                                  & = x^2 + y^2 + z^2 - t^2 + O(\epsilon^2)
\end{eqnarray*}

\noindent Hence this transforamtion is still a Lorentz Transformation if we neglect terms of order $\epsilon^2$.

Consider the time-like world line (SEE FIG pg 5:3)of a particle in Minkowkian space-time $x^i = x^i(s)$. If $s$ is arc length or proper time then $v^i(s) = \frac{dx^i}{ds}$ is the unit tangent(NOT SURE WHY???) vector field. It is clear that $v^i(s)$ must be time-like as $x^i(s)$ is time-like, thus

\begin{equation*}
\eta_{ij} v^i v^j = -1. 
\end{equation*}

\noindent Where $\eta_{ij} - \text{diag}(1,1,1,-1)$ is the metric of minkowskiam space-time. This implies that 

\begin{equation*}
(v^1)^2  + (v^2)^2 + (v^3)^2  - (v^4)^2 = -1.
\end{equation*}

Now consider taking a step along the world line of the particle. Define $\bar{s} = s + \alpha$, where $\alpha$ is some real parameter, so that $v^i (s+\alpha) \vcentcolon = \bar{v}^i(\bar{s})$. Hence we also have

\begin{equation*}
({\bar{v}^1})^2  + ({\bar{v}^2})^2 + ({\bar{v}^3})^2  - ({\bar{v}^4})^2 = -1,
\end{equation*}

\noindent and so $v^i(s)$ and $\bar{v}^i(\bar{s})$ are related by a Lorentz transformation. In particlular $v^i(s+\epsilon)$ and $v^i(s)$ are related by an infinitesimal Lorentz Transformation given by Eqn.(\ref{Infinitesimal_Infinitesimal_Lorentz_Transformation}),

\begin{equation}
v^i(s+\epsilon) = v^i(s) = \epsilon \tensor{L}{^{i}_{j}}(s)v^j (s) + O(\epsilon^2).
\end{equation}

\noindent Rearranging to obtain

\begin{equation}
\frac{v^i(s+\epsilon) - v^i(s)}{\epsilon} = \tensor{L}{^{i}_{j}}(s)v^j (s) + O(\epsilon).
\end{equation}

\noindent Now taking the limit as the infinitesimal step, $\epsilon$ goes to zero to obtain a continuous differentiable equation,

\begin{equation}\label{Infinitesimal_DE_interms_v}
\frac{dv^i}{ds} = \tensor{L}{^{i}_{j}}(s) v^j (s).
\end{equation}

\noindent This equation determines the trajectory of the particle through Minkowskian space-time. In terms of $x$ this is equivalent to

\begin{equation*}
\frac{d^2 x^i}{ds^2} = \tensor{L}{^{i}_{j}}(s) \frac{d x^j}{ds}.
\end{equation*}

It is interesting to write these equations in terms of the particles $3$-velocity given by

\begin{equation*}
\vec{u} = \left( \frac{dx}{dt}, \frac{dy}{dt}, \frac{dz}{dt} \right).
\end{equation*}

\noindent Start by using the chain rule on $v^i$,

\begin{equation*}\label{Infinitesimal_Chain_Rule}
v^i = \frac{dx^i}{ds} = \frac{dx^i}{dt} \frac{dt}{ds} = \left(\frac{dx}{dt},\frac{dy}{dt},\frac{dz}{dt},1\right) \frac{dt}{ds}.
\end{equation*}

\noindent Now determine the first integral of $v^i$, which is equal to $-1$ as $v^i$ is time-like,

\begin{equation*} 
-1 = \eta_{ij} v^i v^j =  \left\{ \left( \frac{d}{dt} \right)^2  + \left( \frac{d}{dt} \right)^2  + \left( \frac{d}{dt} \right)^2 - 1  \right\} \left( \frac{dt}{ds} \right)^2,
\end{equation*} 

\noindent as this is just the scalar product in Minkowskian space-time. Therefore (NOT SURE WHERE THIS COMES FROM)

\begin{equation*}
\frac{dt}{ds} = \gamma (s) \vcentcolon = (1-u^2)^{-1/2},
\end{equation*}

\noindent where $u = |\vec{u}| = \sqrt{\vec{u} \cdot \vec{u}}$. Thus from Eqn.(\ref{Infinitesimal_Chain_Rule})

\begin{equation}\label{infinitesimal_v_interms_gamma}
v^i = \gamma(u) (\vec{u}, 1)
\end{equation}

\noindent It is now conveinient to display Eqn.(\ref{Infinitesimal_DE_interms_v}) as two equations denoting the spacial part and the temproal part, in terms of $\gamma$ and $u$. Again using the chain rule to obtain

\begin{equation*} 
\frac{dt}{ds} \frac{dv^i}{dt} = \tensor{L}{^{i}_{j}}v^j. 
\end{equation*} 

\noindent This then implies that 

\begin{eqnarray}\label{Infinitesimal_gamma_u_1}
\gamma(u) \frac{d}{dt} (\gamma(u) u^{\alpha}) = \tensor{L}{^{\alpha}_{j}} v^j, \\\nonumber
\gamma(u) \frac{d}{dt} \gamma(u) = \tensor{L}{^{4}_{j}} v^j,
\end{eqnarray}

\noindent as $v^i = \gamma(u)(\vec{u},1)$. Here we have used the usual convention that greek indices denote the sum over the spacial indices only, thus $\alpha = 1,2,3$. Now Eqn.(\ref{infinitesimal_v_interms_gamma}) can be used to rewrite the $\tensor{L}{^{i}_{j}}$ coefficients to get

\begin{eqnarray}\label{Infinitesimal_gamma_u_2}
\tensor{L}{^{\alpha}_{j}} v^j = \gamma(u) (\tensor{L}{^{\alpha}_{\beta}} u^{\beta} + \tensor{L}{^{\alpha}_4}) \\ \nonumber
\tensor{L}{^{4}_{j}} v^j = \gamma(u) (\tensor{L}{^{4}_{\alpha}} u^{\alpha0})
\end{eqnarray} 

\noindent where $\tensor{L}{^{4}_{4}} = 0$ from Eqn.(\ref{infinitesimal_Matrix_component_wise}). Putting together Eqns.(\ref{Infinitesimal_gamma_u_1}) and (\ref{Infinitesimal_gamma_u_2}) to obtain differential equations for the spacial and temporal coordinates in terms of the particles $3$-velocity,

\begin{eqnarray*} 
\frac{d}{dt} (\gamma(u) u^{\alpha}) = \tensor{L}{^{\alpha}_{\beta}} u^{\beta} + \tensor{L}{^{\alpha}_{4}} \\
\frac{d\gamma(u)}{dt} = \tensor{L}{^{4}_{\alpha}} u^{\alpha}
\end{eqnarray*} 

(ON TOP OF pg 5:6)




 








