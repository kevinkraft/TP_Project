\section{Stereographic Projecttion and the Extended Complex Plane}

Stereographic projection is the mapping of points on a sphere to points on a plane. In $\mathbb{R}^3$ with rectangular cartesian coordinates, $x$, $y$, $z$, consider the unit sphere with centre $(0,0,0,0)$, defined by

\begin{equation*}
\mathbb{S}^2 \subset \mathbb{R}^3 \text{ : } x^2 + y^2 + x^2 = 1.
\end{equation*}

(SEE FIG PG 4:1) $Q = Q(X,Y,0)$

\noindent The projection $P \rightarrow Q$ is a stereographic projection. A relationship between $X$,$Y$ and $(x,y,z)$ is constructed as follows. $P$ is subdivided into the line segment $NQ$ in some ratio, $l:m$ say. By coordinate geometry

\begin{eqnarray*}  
x = \frac{lX + mO}{l+m} = \frac{lX}{l+m}, \\
y = \frac{lY + mO}{l+m} = \frac{lY}{l+m}, \\
z = \frac{l.0+ m.1}{l+m} = \frac{m}{l+m}.
\end{eqnarray*}

\noindent This implies that

\begin{eqnarray*}
1-z = \frac{l}{l+m}, \\
x = (1-z)X, \\
y = (1-z)Y.
\end{eqnarray*}

\noindent It is also known that $x^2+ y^2 +z^2 = 1$, so using this relation it is clear that

\begin{eqnarray*}
x^2 + y^2 = 1-z^2 = (1-z) (1+z) \\
\Rightarrow (1-z^2)(X^2 +Y^2) = (1-z) (1+z)
\end{eqnarray*}

\noindent If the point $N$ is excluded, i.e. $z \neq 1$ then dividing by $(1-z)^2$ to obtain

\begin{equation*} 
X^2 + Y^2 = \frac{1+z}{1-z}.
\end{equation*}

\noindent Rearranging to find that

\begin{equation}\label{Ext_Complex_z_in_term_XY} 
z = \frac{X^2 + y^2 - 1}{X^2 + y^2 + 1}
\end{equation}

Define $\zeta = X+ iY$ and rewrite Eqn.(\ref{Ext_Complex_z_in_term_XY}) to see that


\begin{equation*}
z = \frac{\zeta\bar{\zeta} - 1}{\zeta\bar{\zeta} + 1},
\end{equation*}

\noindent Which implies that

\begin{equation*}
1- z = \frac{2}{\zeta\bar{\zeta} + 1}.
\end{equation*}

\noindent So relations for $(x,y,z) \in \mathbb{S}^2 \backslash \{N\}$ have been obtained in terms of $\zeta$.

\begin{eqnarray}\label{Ext_Complex_xy_interms_zeta}
x + iy = \frac{2\zeta}{\zeta\bar{\zeta}}, \\\label{Ext_Complex_z_interms_zeta}
z z = \frac{\zeta\bar{\zeta} - 1}{\zeta\bar{\zeta} + 1}.
\end{eqnarray}

\noindent Hence the points on $\mathbb{S}^2 \backslash \{N\}$ are labelled by complex numbers $\zeta \in \mathbb{C}$. If a point $\zeta = \infty$, called the point at infinity of $\mathbb{C}$, is allowed then the following limits hold:

\begin{eqnarray*}
x+ iy = \frac{2/\bar{\zeta}}{1 + 1/\zeta\bar{\zeta}} \rightarrow 0 \text{, as  } \zeta \rightarrow \infty \\
z = \frac{1- 1/\zeta\bar{\zeta}}{1+ 1/\zeta\bar{\zeta}} \rightarrow 1 \text{, as  } \zeta \rightarrow \infty 
\end{eqnarray*}

\noindent Then $N = (0,0,1)$ corresponds to $\zeta = \infty$. Thus in this way there is a one to one correspondence between the points of $\mathbb{S}^2$ and the points of the \textit{extended complex plane} $\hat{\mathbb{C}} = \mathbb{C} \cup {\infty}$, which is the usual complex plane with the point at infinity added. Since $\mathbb{S}^2$ has finite surface area, and is therefore called a \textit{compact manifold}, the identification of the points of $\hat{\mathbb{C}}$ with the points of $\mathbb{S}^2$ is called the \textit{compactification} of $\hat{\mathbb{C}}$.

$(\zeta, \bar{\zeta})$ are called the \textit{stereographic coordinates} on $\mathbb{S}^2 \backslash \{N\}$. How are they related to the polar angles $\theta$ and $\phi$? To investigate this write the usual spherical polar coordinates in terms of $\zeta$. First it is known that

\begin{eqnarray*}
x = \sin{\theta}\cos{\phi}, \\
y = \sin{\theta}\sin{\phi}, \\
z = \cos{\theta}.
\end{eqnarray*}

\noindent So using Eqn.(\ref{Ext_Complex_xy_interms_zeta}) it is easy to show that

\begin{eqnarray*}
\cos{\theta} = \frac{\zeta\bar{\zeta} - 1}{\zeta\bar{\zeta} + 1} \\
\Rightarrow \zeta\bar{\zeta}\cos{\theta} + \cos{\theta} = \zeta\bar{\zeta} - 1\\
\Rightarrow \zeta\bar{\zeta} = \frac{1 + \cos{\theta}}{1 - \cos{\theta}} = \frac{2\cos^2{\left(\theta/2\right)}}{2\sin^2{\left(\theta/2\right)}} \\
\Rightarrow \zeta\bar{\zeta} = \cot^2{\left(\theta/2\right)}.
\end{eqnarray*}

\noindent Now using Eqn.(\ref{Ext_Complex_z_interms_zeta}) to obtain:

\begin{eqnarray*}
\sin{\theta}(\cos{\phi} +i \sin{\phi}) = \frac{2\zeta}{\cot^2{\left(\theta/2\right)}}, \\
2\sin{\left(\theta/2\right)}\cos{\left(\theta/2\right)}e^{i\phi} = 2\zeta \sin^2{\left(\theta/2\right)}, \\
\Rightarrow 
\end{eqnarray*}












