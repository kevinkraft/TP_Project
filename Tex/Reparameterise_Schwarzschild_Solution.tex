\section{Reparameterisation of the Schwarzschild Solution}

In this section the Kasner solution of the vacuum field equations is derived from the Schwarschild solution by taking the limit as the mass goes to infinity. It is then shown that the special case of the Kasner solution with no mass is equivalent to a novel form of Minkowskian space-time. (what do we want the Kasner vacuum solution for???). Starting with the Schwarschild Solution of the vacuum field equations:

\begin{equation}\label{Schwarz_Sol} 
\epsilon {ds}^2 = {\big(1 - \frac{2m}{r}\big)}^{-1} {dr}^{2} + r^2 ({d\theta}^2 + {{\sin}^2 \theta}{d \phi}^2) - \big(1 - \frac{2m}{r}\big) {dt}^2
\end{equation}

\noindent Now make the Eddington-Finkelstein coordinate transformation:

\begin{equation}\label{Ed-Fin_trans}
u = t - r - 2m \ln(r - 2m)
\end{equation}

\noindent Working out the differential:

\begin{eqnarray*}
du = dt - dr - \frac{2m dr}{r - 2m}
\end{eqnarray*}

SEE CALCULATIONS PG1

\noindent Then write Eqn.(\ref{Schwarz_Sol}) in terms of $u$ to obtain:

\begin{equation*} 
\epsilon {ds}^{2} = r^2 ({d\theta}^2 + {{\sin}^2 \theta}{d \phi}^2) - 2 du dr - {du}^{2} + \frac{2m}{r} {du}^{2} 
\end{equation*}

Note that if $m = 0$ the space-time becomes Minkowskian, as expected. Setting $r = 0$ in this Minkowskian space-time the line element becomes:

$$ \epsilon {ds}^2 = - {du}^{2} \Rightarrow t = -1$$

\noindent Which implies that $r = 0$ is a time-like world-line in Minkowskian space-time, with proper time u.(CHECK ITS A GEODESIC???)

We want to find the limit of the Schwarzschild solution as $m \rightarrow \infty$. In its current form, the limit cannot be calculated so first a suitable coordinate transformation must be made. 

\begin{eqnarray*} 
u = \mu u'        & \text{, where }  \mu = \text{ const}\\
r = {\mu}^{-1} r' & 
\end{eqnarray*} 

\noindent So that
  
\begin{eqnarray*} 
du = \mu du' \\
dr = {\mu}^{-1} dr' \\
\Rightarrow du dr = du' dr'
\end{eqnarray*} 

\noindent To obtain (SEE CALCS PG3):

\begin{equation*}
\epsilon {ds}^2 = {r'}^2 \sin^2 \theta \left\{ \frac{{d\theta}^2}{\mu^2 \sin^2 \theta} +  \mu^{-2} {d \phi}^2 \right\} - 2 {du'} {dr'} - \left( \mu^2 - \frac{2 m \mu^3}{r'} {du'} \right) 
\end{equation*}

\noindent Now set $m \mu^{3} = k = \text{ const} \Rightarrow m = k \mu^{-3}$ and make another transformation first done by Ivor Robinson(check this???) given by:

\begin{eqnarray} 
\sin{\theta} = \frac{1}{\cosh{(\mu \xi)}} & \text{   ,    } \mu^{-1} \phi = \eta  
\end{eqnarray} 

\noindent Which results in (SEE CALCS PG 3): 

\begin{equation*}
\epsilon {ds}^2 = \frac{r^2}{\cosh^{2}{\mu \xi}} ({d\xi}^2 + {d\eta}^2) - 2 {du}{dr} - \left( \mu^{2} - \frac{2k}{r} \right) {du}^2
\end{equation*}

\noindent Where the primes have been dropped for notational simplicity. This is now in an appropriate form to take the limit $m \rightarrow \infty$ which is equivalent to $\mu \rightarrow 0$. This limit gives:

\begin{equation}\label{Kasner_after_limit}
\epsilon {ds}^2 = r^2 ({d\xi}^2 + {d\eta}^2) - 2 {du}{dr} - \frac{2k}{r} {du}^2
\end{equation}

\noindent This is still a solution of the field equations, but it is no longer the Schwarzschild solution. It is found that this is the Kasner solution. To see this (SEE CALCS PG ???):

\begin{equation}\label{Our_Kasner} 
\epsilon {ds}^2 = T^{-2/3} {dX}^2 + T^{4/3} \left( dY^2 + dZ^2 \right) - dT^2
\end{equation}

\noindent By definition the Kasner solution is given by:

\begin{equation*} 
\epsilon {ds}^2 = T^{2p} {dX}^2 + T^{2q} dY^2 + T^{2r} dZ^2 - dT^2
\end{equation*}

\noindent With:

\begin{eqnarray*}
p + q + r = 1 = p^2 + q^2 + r^2
\end{eqnarray*}

\noindent So it is clear that Eqn.(\ref{Our_Kasner}) is the Kasner solution with $p = -1/3$ and $q = r = 2/3$.

\subsection{Line Element of Minkowskian Space-Time}

Minkowskian space-time reemerges again by setting $k = 0$ in Eqn.(\ref{Kasner_after_limit}), which is equivalent to $m = 0$. 

\begin{equation}\label{Kasner_after_limit_no_k}
\epsilon {ds}^2 = r^2 ({d\xi}^2 + {d\eta}^2) - 2 {du}{dr}
\end{equation}

Setting $r = 0$ it can be shown that $\epsilon {ds}^2 = 0$ in this case. Then $r = 0$ is a null geodesic with $u$ an affine parameter along it. To demonstrate these properties first let $x^i = (x,y,z,t)$ be rectangular Cartesian coordinates with time in Minkowskian space-time with the usual line element:

\begin{equation*} 
\epsilon {ds_0}^2 = {dx}^2 + {dy}^2 + {dz}^2 - {dt}^2
\end{equation*} 

\noindent We note that $C: x = 0, y = 0, z = t$ is a null geodesic as it will lie in the light cone of Minkowskian space-time. it we write it parametrically as $x^i = w^i (u)$ such that $w^i = (0,0,u,u)$ then $u$ is an affine parameter along $C$. The tangent to $C$ is then computed as:

\begin{equation*} 
v^i (u) = \frac{d w^i}{du} = (0,0,1,1)
\end{equation*} 
   
\noindent As $C$ is a null geodesic the first integral will be $v_i v^i = 0$ (IS THIS CALLED THE FIRST INTEGRAL?) and thus $v_i = (0,0,1,-1)$ where we have chosen the convention $(+,+,+,-)$. 

The position vector of a point in Minkowskian space time can be written in the form(DO THE PICTURE FROM 2:1):

\begin{eqnarray*}
x^i = w^i (u) = r k^i \\
\text{or } x^i = w^i(u) + r k^i 
\end{eqnarray*}

\noindent Thus $r$ is a new parameter which tell us the shortest distance between $C$ and some point $x^i$, and $k^i$ is the unit vector in that direction. As $k^i$ is a unit vector it satisfies the realtions:

\begin{eqnarray}
k^i k_i = 0 \label{k_rel_1}\\
k^i v_i = -1 \label{k_rel_2}
\end{eqnarray}

\noindent Thus $k^i$ is normalized so that $k^i$ and $v^i$ are both future pointing (HOW DOES THIS MAKE THEM BOTH FUTURE POINTING?). Making the parameterisation:

\begin{eqnarray*}
k^i = (\xi, \eta, A, B) \\
\Rightarrow k_i = (\xi, \eta, A, -B)
\end{eqnarray*}

\noindent We can choose any variable for the first two slots of $k^i$ so we choose $\xi$ and $\eta$ from before for conveinience. Using the relation (\ref{k_rel_1}) it is clear that:

\begin{equation*}
\xi^2 + \eta^2 + A^2 - B^2 = 0
\end{equation*}

\noindent and using the relation (\ref{k_rel_2}) it is found that:

\begin{eqnarray}
A - B = -1 \label{sim_rel_1}\\
\Rightarrow A^2 - B^2 = (A + B)(A - B) = - (A + B)
\end{eqnarray}


\noindent Which implies:

\begin{equation}\label{sim_rel_2}
\xi^2 + \eta^2 = A + B 
\end{equation}

\noindent So expressions for $A$ and $B$ are found using Eqn.(\ref{sim_rel_1}) and Eqn.(\ref{sim_rel_2}):

\begin{eqnarray*}
A = \frac{1}{2} (-1 + \xi^2 + \eta^2) \\
B = \frac{1}{2} (1 + \xi^2 + \eta^2) \\
\end{eqnarray*}

In summary so far we have:

\begin{eqnarray}
x^i = w^i (u) + r k^i \label{rel_for_trans_1}\\
w^i = (0,0, u,u) \label{rel_for_trans_2}\\
k^i = (\xi, \eta, \frac{1}{2} (-1 + \xi^2 + \eta^2), \frac{1}{2} (1 + \xi^2 + \eta^2)) \label{rel_for_trans_3}\\
x^i = (x, y, z, t) \label{rel_for_trans_4}  
\end{eqnarray}

Consider Eqn.(\ref{rel_for_trans_1}) as a coordinate transformation from $(x,y,z,t)$ to $(\xi,\eta, r, u )$ such that:

\begin{eqnarray}
x = r \xi \nonumber \\
y = r \eta \nonumber \\
z = u + \frac{r}{2} (-1 + \xi^2 + \eta^2) \nonumber \\
t = u + \frac{r}{2} (1 + \xi^2 + \eta^2)  \label{trans_x_to_xi} 
\end{eqnarray} 

\noindent as is clear from Eqn.(\ref{rel_for_trans_1}) - Eqn.(\ref{rel_for_trans_4}). Now this is applied to the Minkowskian line element Eqn.(\ref{Kasner_after_limit_no_k}). First, the $x$ and $y$ differentials are:

\begin{eqnarray*} 
dx = r d\xi + \xi dr \\
dy = r d\eta + \eta dr 
\end{eqnarray*} 

\noindent Which gives:

\begin{equation}\label{differentials_1}
{dx}^2 + {dy}^2 = r^2 ({d\xi}^2 + {d\eta}^2) + 2 r \xi {d\xi} {dr} + 2 r \eta {d\eta}{dr} + (\xi^2 + \eta^2) {dr}^2
\end{equation}

\noindent Next, the $z$ and $t$ differentials:

\begin{eqnarray*}
z + t = 2 u + r (\xi^2 + \eta^2) \\
z - t = - r \\
{dz} + {dt} = 2 du + (\xi^2 + \eta^2) dr + 2 r \xi {d\xi} + 2 r \eta {d\eta} \\
{dz} - {dt} = - dr 
\end{eqnarray*}

\noindent Using difference of two squares to obtain:

\begin{equation}\label{differentials_2}
{dz}^2 - {dt}^2 = -2 {du}{dr} - (\xi^2 + \eta^2) {dr}^2 - 2 r \xi {d\xi}{dr} - 2 r \eta {d\eta}{dr}
\end{equation}

\noindent Combining Eqn.(\ref{differentials_1}) and (\ref{differentials_2}) to get:

\begin{equation*}
{dx}^2 + {dy}^2 + {dz}^2 - {dt}^2 = r^2 ({d\xi}^2 + {d\eta}^2) - 2 {du}{dr}
\end{equation*}

\noindent and from this it is clear that Eqn.(\ref{Kasner_after_limit_no_k}) is the line element of Minkowskian space-time with $r = 0$ a null geodesic with affine parameter $u$ along it as before. 
