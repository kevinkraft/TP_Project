\newpage

\begin{appendix}

\section{Singular Lorentz Transformation with Special Significance Given to $x$}\label{Appendix_Special_Significance_x}

The choice of components of the matrix $A$ in Eqn.(\ref{Special_Matrices_A_first}) gives an arbitrary but special significance to the corrdinate $z$. In this calculation the matrix $U$ of Eqn.(\ref{SL_trans}) is determined for a singular Lorentz transformation, in which special significance has been given to the coordinate $x$. This transformation is given by

\begin{align*}
t'-x' & = t-x, \\
z'+iy' & = z + iy + w(t-x), \\
t'+x' & = t+x + w(z-iy) + \bar{w} (z + iy) + w \bar{w} (t-x).
\end{align*}

\noindent Notice that this is the same tranformation as in Example 3, Section (\ref{Special_Linear_Matrices_Example_3}), with $x$ and $z$ swapped. This is equivalent to swapping these two coordinates in the matrix $A$ only. Note that if $x$ and $z$ were exchanged in both $A$ and the transformation then the $U$ obtained would be exactly the same as the example done previously.

First, the components of the matrix $A$ must be construced. Thus the transformation must be written out explicitly as 

\begin{align*}
x' & = x + \frac{1}{2} (w + \bar{w})z + \frac{i}{2}(\bar{w}- w)y + \frac{1}{2}w\bar{w}(t-x), \\
y' & = y + \frac{i}{2} (\bar{w} - w)(t-x), \\
z' & = z + \frac{1}{2} (w + \bar{w})(t-x), \\
t' & = t + \frac{1}{2} (w + \bar{w})z + \frac{i}{2} (\bar{w}-w)y + \frac{1}{2} w\bar{w}(t-x).
\end{align*}

\noindent Now construct the components of $A$

\begin{align*}
t' - z' & = t - z + \frac{1}{2}(w + \bar{w})z + \frac{i}{2} (\bar{w}-w)y + \frac{1}{2}w\bar{w}(t-x) - \frac{1}{2}(\bar{w} + w)(t-x),\\
t' + z' & = t + z + \frac{1}{2}(w + \bar{w})z + \frac{i}{2} (\bar{w}-w)y + \frac{1}{2}w\bar{w}(t-x) + -\frac{1}{2}(\bar{w} + w)(t-x),\\
x' + iy' & = x +  \frac{1}{2}(w + \bar{w})z + \frac{i}{2} (\bar{w}-w)y + \frac{1}{2}w\bar{w}(t-x) - \frac{1}{2}(\bar{w} + w)(t-x) + iy.
\end{align*}

\noindent Equating coefficients of $x$, $y$, $z$, $t$ on both sides of Eqn.(\ref{general_coeff_equate_a}) to obtain

\begin{subequations}
\begin{gather}\label{Ex_Ap_equate_coeffs_first_a}
\alpha \bar{\beta} + \bar{\alpha} \beta = \frac{1}{2}(\bar{w}+w) - \frac{1}{2}w\bar{w}, \\\label{Ex_Ap_equate_coeffs_first_b}
\alpha \bar{\beta} - \bar{\alpha} \beta = \frac{1}{2} (\bar{w} - w), \\\label{Ex_Ap_equate_coeffs_first_c}
-\alpha \bar{\alpha} + \beta \bar{\beta} = - 1 + \frac{1}{2}(\bar{w} + w), \\\label{Ex_Ap_equate_coeffs_first_d}
\alpha \bar{\alpha} + \beta \bar{\beta} = 1 + \frac{1}{2}w\bar{w} - \frac{1}{2}(\bar{w} + w). 
\end{gather}
\end{subequations}

\noindent Then Eqn.(\ref{Ex_Ap_equate_coeffs_first_a}) and (\ref{Ex_Ap_equate_coeffs_first_b}) imply $\alpha \bar{\beta} = \frac{1}{2}\bar{w} - \frac{1}{4}w\bar{w}$. Also Eqn.(\ref{Ex_Ap_equate_coeffs_first_c}) and (\ref{Ex_Ap_equate_coeffs_first_d}) imply $\beta\bar{\beta} = \frac{1}{4}w\bar{w}$ and $\alpha \bar{\alpha} = 1 + \frac{1}{4}w\bar{w} - \frac{1}{2}\bar{w} - \frac{1}{2}w$. Hence $\alpha$ can be written in terms of $\beta$

\begin{gather*}
\alpha \bar{\beta} = (\frac{1}{2}\bar{w} - \frac{1}{4}w\bar{w}), \\
\frac{1}{4} w \bar{w} \alpha = \frac{1}{2}\bar{w}(1-\frac{1}{2}w)\beta, \\
\alpha = \frac{(2-w)}{w}\beta.
\end{gather*}

Equating coefficients of $x$, $y$, $z$, $t$ on both sides of Eqn.(\ref{general_coeff_equate_b}) to obtain

\begin{subequations}
\begin{gather}\label{Ex_Ap_equate_coeffs_second_a}
\alpha \bar{\delta} + \beta\bar{\gamma} = 1 - \frac{1}{2}w\bar{w} + \frac{1}{2}(\bar{w}-w), \\\label{Ex_Ap_equate_coeffs_second_b}
\alpha \bar{\delta} - \beta\bar{\gamma} = \frac{1}{2}(\bar{w}-w) + 1,\\\label{Ex_Ap_equate_coeffs_second_c}
-\alpha\bar{\gamma} + \beta \bar{\delta} = \frac{1}{2}(\bar{w} + w) ,\\\label{Ex_Ap_equate_coeffs_second_d}
\alpha\bar{\gamma} + \beta \bar{\delta} = \frac{1}{2}w\bar{w} - \frac{1}{2}(\bar{w}-w). 
\end{gather}
\end{subequations}

\noindent Now Eqn.(\ref{Ex_Ap_equate_coeffs_second_a}) and (\ref{Ex_Ap_equate_coeffs_second_b}) imply $\beta \bar{\gamma} = -\frac{1}{4}w\bar{w}$. So using $\bar{\beta} \beta= \frac{1}{4}w\bar{w}$ again to obtain

\begin{gather*}
\frac{1}{4}w\bar{w} \bar{\gamma} = \bar{\beta}\beta\bar{\gamma},\\
\frac{1}{4}w\bar{w} \bar{\gamma} = -\frac{1}{4}w\bar{w} \bar{\beta},\\
\gamma = -\beta.
\end{gather*}

\noindent Also Eqn.(\ref{Ex_Ap_equate_coeffs_second_c}) and (\ref{Ex_Ap_equate_coeffs_second_d}) imply $\beta \bar{\delta} = \frac{1}{4}w\bar{w} + \frac{1}{2}w$. Thus

\begin{gather*}
\frac{1}{4}w\bar{w}\bar{\delta} = \left(\frac{1}{4}w\bar{w}+ \frac{1}{2}w\right)\bar{\beta},\\
\delta = \left(\frac{w\bar{w} + 2\bar{w}}{w\bar{w}}\right)\beta.
\end{gather*}

\noindent Equating coefficients of $x$, $y$, $z$, $t$ on both sides of Eqn.(\ref{general_coeff_equate_c}) to obtain

\begin{subequations}
\begin{gather}\label{Ex_Ap_equate_coeffs_third_a}
\gamma \bar{\delta} + \delta \bar{\gamma} = -\frac{1}{2}w\bar{w} - \frac{1}{2}(w + \bar{w}), \\\label{Ex_Ap_equate_coeffs_third_b}
\gamma \bar{\delta} - \delta \bar{\gamma} =  \frac{1}{2}(\bar{w}-w),\\\label{Ex_Ap_equate_coeffs_third_c}
-\gamma \bar{\gamma} + \delta \bar{\delta} = 1 + \frac{1}{2}(w + \bar{w}),\\\label{Ex_Ap_equate_coeffs_third_d}
\gamma \bar{\gamma} + \delta \bar{\delta} = 1 + \frac{1}{2}w\bar{w} + \frac{1}{2}(\bar{w}+w). 
\end{gather}
\end{subequations}

\noindent Here Eqn(\ref{Ex_Ap_equate_coeffs_third_a}) and (\ref{Ex_Ap_equate_coeffs_third_b}) imply $\gamma \bar{\delta} = -\frac{w}{4}(2+\bar{w})$. Also, Eqn(\ref{Ex_Ap_equate_coeffs_third_c}) and Eqn(\ref{Ex_Ap_equate_coeffs_third_d}) imply $\gamma \bar{\gamma} = \frac{1}{4}w\bar{w}$. So using these relations $\delta$ can be written in terms of $\gamma$ 

\begin{gather*}
\gamma \bar{\gamma} \bar{\delta} = \frac{w}{4}(2+\bar{w})\bar{\gamma},\\
\frac{1}{4}w\bar{w} \bar{\delta} = \frac{w}{4}(2+\bar{w})\bar{\gamma},\\
\delta = -\frac{(2 + w)}{w}\gamma.
\end{gather*}

\noindent At this point $\beta$ and $\delta$ have been written in terms of $\gamma$ and $\alpha$ is written in terms of $\beta$. Write $\alpha$ in terms of $\gamma$

\begin{equation*}
\alpha = \frac{(2-w)}{w}\beta = \frac{(w-2)}{w}\gamma.
\end{equation*}

\noindent Now use the condition that $\det{(U)} = 1$ and replace everything in favour of $\gamma$

\begin{gather*}
\alpha \delta - \beta \gamma = 1,\\
-\frac{(w^2 - 4)}{w^2}\gamma^2 + \gamma^2 = 1,\\
\frac{4}{w^2}\gamma^2 = 1,\\
\gamma = \pm \frac{w}{2}.
\end{gather*}

\noindent Replace $\gamma$ and write all the components of $U$ as functions of $w$ only

\begin{align*}
\alpha & = \pm \frac{1}{2}(w-2),\\
\beta & = \mp \frac{w}{2},\\
\gamma & = \pm \frac{w}{2},\\
\delta & = \mp \frac{1}{2}(w+2).
\end{align*}

\noindent Finally it is found that 

\begin{equation*}
U = \pm
\left(
\begin{array}{ccc}
\frac{w-2}{2} & & -\frac{w}{2}     \\
 & & \\
\frac{w}{2}   & & -\frac{(w+2)}{2} \\
\end{array}
\right).
\end{equation*}

Now the fractional linear trasnformation associated with $U$ must be determined. Using Eqn.(\ref{Extended_Complex_Fractional_Linear_Transformation}) to find that

\begin{equation*}
\zeta' = \frac{\bar{w} - (\bar{w} +2)\zeta}{\bar{w} -2 - \bar{w}\zeta}. 
\end{equation*}

\noindent The fixed points of the system and thus the null directions are found by setting $\zeta' = \zeta$ to obtain

\begin{gather*}
\bar{w} \zeta^2 - 2 \bar{w} \zeta + \bar{w} = 0,\\
\bar{w}(\zeta-1)^2 = 0.
\end{gather*}

\noindent Thus there is a fixed point at $\zeta = 1$. The corresponding null direction is then given by Eqn.(\ref{Ext_Complex_vec_x_relations}) such that $\vec{x}= t(1,0,0,1)$. Hence finally the null direction is the generator of $N^+$, $x=t$. So in this case where $x$ was given special significance instead of $z$ with the given singular Lorentz transformation it is found that the null direction is $x=t$ instead of $z=t$. This is as expected as by exchanging $z$ with $x$ the coordinate system has been rotated. 

\section{Standard Lorentz Transformation of the Electromagnetic Field Vectors}\label{Appendix_Standard_Transform_EM_Vectors}

In this appendix the transformations laws for $\vec{E} = (E^1, E^2, E^3)$ and $\vec{B} = (B^1, B^2 B^3)$ under the standard Lorentz transformation of Eqn.(\ref{Special_Matrices_Standard_Lorentz}) are derived. If $p^i$ is any vector transported along $x^i = x^i (s)$ then, in Eqn.(\ref{Infinitesimal_DE_interms_v}), $v^i$ can be replaced with $p^i$, such that

\begin{equation*} 
\frac{\mathrm{d}p^i}{\mathrm{d}s} = \tensor{L}{^i_j} p^j.
\end{equation*} 

\noindent Here $s =$ arc length so the usual line element

\begin{equation*} 
\epsilon \mathrm{d}s^2 = \mathrm{d}x^2 + \mathrm{d}y^2 + \mathrm{d}z^2 - \mathrm{d}t^2,
\end{equation*} 

\noindent is formed. $p^i$ is a $4$-vector so it transforms like $x^i = (x,y,z,t)$ under the standard Lorentz transformation, such that

\begin{align*}
\bar{p}^1 = \gamma (p^1 - v p^4),\\
\bar{p}^2 = p^2, \\
\bar{p}^3 = p^3,\\
\bar{p}^4 = \gamma (p^4 - v p^1).
\end{align*}

\noindent Which implies that 

\begin{equation*}
\mathrm{d}p^i = (\mathrm{d}p^1,\mathrm{d}p^2,\mathrm{d}p^3,\mathrm{d}p^4),
\end{equation*}

\noindent is a $4$-vector and ds is invariant, thus $\frac{\mathrm{d}p^i}{\mathrm{d}s}$ is also a $4$-vector. Hence $\tensor{L}{^i_j} p^j$ is a $4$-vector for any $4$-vector $p^i$, where $\tensor{L}{^i_j}$ is given by

\begin{equation*}
\tensor{L}{^i_j} = 
\left(
\begin{array}
0    & B^3  & -B^2 & E^1 \\
-B^3 & 0    & B^1  & E^2 \\
B^2  & -B^1 & 0    & E^3 \\
E^1  & E^2  & E^3  & 0   \\
\end{array}
\right)
\end{equation*}

\noindent and a factor of $q/m$ has been left out as it will play no role. Write out $\tensor{L}{^i_j} p^j$ explicitly

\begin{align*}
\tensor{L}{^1_j}p^j & = B^3 p^2 - B^2 p^3 + E^1 p^4, \\
\tensor{L}{^2_j}p^j & = -B^3 p^1 + B^1 p^3 + E^2 p^4, \\
\tensor{L}{^3_j}p^j & = B^2 p^1 - B^1 p^2 + E^3 p^4, \\
\tensor{L}{^4_j}p^j & = E^1 p^1 + E^2p^2 + E^3 p^3.
\end{align*}

\noindent Since $\tensor{L}{^i_j} p^j$ is a $4$-vector it transforms under the standard Lorentz transformation, such that

\begin{align}
\label{Appendix_Lp_Standard_Transform_a}
\tensor{\bar{L}}{^1_j}\bar{p}^j = \gamma (\tensor{L}{^1_j}p^j - v \tensor{L}{^4_j}p^j),
\\\label{Appendix_Lp_Standard_Transform_b}
\tensor{\bar{L}}{^2_j}\bar{p}^j = \tensor{L}{^2_j}p^j,
\\\label{Appendix_Lp_Standard_Transform_c}
\tensor{\bar{L}}{^3_j}\bar{p}^j = \tensor{L}{^3_j}p^j,
\\\label{Appendix_Lp_Standard_Transform_d}
\tensor{\bar{L}}{^4_j}\bar{p}^j = \gamma(\tensor{L}{^4_j}p^j - v \tensor{L}{^1_j}p^j).
\end{align}

(ERROR. DONE TO HERE)


















\section{The Triad of Electromagnetic Radiation}\label{Appendix_Orthonormal_Triad}

In this appendix it is shown that the vectors $(\vec{n}, \vec{B}, \vec{E})$ form a right-handed triad. From Eqn.(\ref{Infinitesimal_K_Unit_Vector_Formula}), $k^i$ can be rewritten as

\begin{equation*}
k^i = (\zeta \bar{\zeta} + 1)(\vec{n}, 1),
\end{equation*}

\noindent where $\vec{n} = (n^1, n^2, n^3)$ is a unit vector such that $\vec{n} \cdot \vec{n} = 1$. Expanding the relation $\mathcal{L}_{ij} k^i = 0$ in terms of $\vec{n}$ gives

\begin{subequations}
\begin{align}
\label{Appendix_2_Expand_L_K_product_a}
(B^3 - i E^3)n^2 - (B^2 - i E^2)n^3 + E^1 + iB^1 & = 0, 
\\\label{Appendix_2_Expand_L_K_product_b}
-(B^3 - i E^3)n^1 + (B^1 - i E^1)n^3 + E^2 + iB^2 & = 0, 
\\\label{Appendix_2_Expand_L_K_product_c}
(B^2 - i E^2)n^1 - (B^1 - i E^1)n^2 + E^3 + iB^3 & = 0, 
\\\label{Appendix_2_Expand_L_K_product_d}
-(E^1 + i B^1)n^1 - (E^2 + iB^2)n^2 - (E^3 + iB^3) & = 0.
\end{align}
\end{subequations}

\noindent Then by equating real and imaginary parts of Eqn.(\ref{Appendix_2_Expand_L_K_product_d}) it is clear that  $\vec{E} \cdot \vec{n} = \vec{B} \cdot \vec{n}$. Also, from Eqns.(\ref{Appendix_2_Expand_L_K_product_a}) - (\ref{Appendix_2_Expand_L_K_product_c}) the following series of equations are seen, first Eqn.(\ref{Appendix_2_Expand_L_K_product_a}) implies

\begin{align*}
E^1 & = B^2 n^3 - B^3 n^2, \\
B^1 & = E^3 n^2 - E^2 n^3, 
\end{align*}

\noindent while Eqn.(\ref{Appendix_2_Expand_L_K_product_b}) gives

\begin{align*} 
E^2 & = B^3 n^1 - B^1n^3, \\
B^2 & = E^1 n^3 - E^3 n^1.
\end{align*} 

\noindent Then Eqn.(\ref{Appendix_2_Expand_L_K_product_c}) implies

\begin{align*}
E^3 & = B^1 n^2 - B^2 n^1,\\
B^3 & = E^2 n^1 - E^1 n^2.
\end{align*}

\noindent Thus is it clear that these are the components of the curl $\vec{E} = \vec{B} \times \vec{n}$ and $\vec{B} = \vec{n} \times \vec{E}$. This curl is proof that the vectors $(\vec{n}, \vec{E}, \vec{B})$ form a right-handed triad. 




\end{appendix}
