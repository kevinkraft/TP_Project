\begin{appendix}

\section{Singular Lorentz Transformation with Special Significance Given to $x$}\label{Appendix_Special_Significance_x}

The choice of components of the matrix $A$ in Eqn.(\ref{Special_Matrices_A_first}) gives an arbitrary but special significance to the corrdinate $z$. In this calculation the matrix $U$ of Eqn.(\ref{SL_trans}) is determined for a singular Lorentz transformation, in which special significance has been given to the coordinate $x$. This transformation is given by

\begin{align*}
t'-x' & = t-x, \\
z'+iy' & = z + iy + w(t-x), \\
t'+x' & = t+x + w(z-iy) + \bar{w} (z + iy) + w \bar{w} (t-x).
\end{align*}

\noindent Notice that this is the same tranformation as in Example 3, Section (\ref{Special_Linear_Matrices_Example_3}), with $x$ and $z$ swapped. This is equivalent to swapping these two coordinates in the matrix $A$ only. Note that if $x$ and $z$ were exchanged in both $A$ and the transformation then the $U$ obtained would be exactly the same as the example done previously.

First, the components of the matrix $A$ must be construced. Thus the transformation must be written out explicitly as 

\begin{align*}
x' & = x + \frac{1}{2} (w + \bar{w})z + \frac{i}{2}(\bar{w}- w)y + \frac{1}{2}w\bar{w}(t-x), \\
y' & = y + \frac{i}{2} (\bar{w} - w)(t-x), \\
z' & = z + \frac{1}{2} (w + \bar{w})(t-x), \\
t' & = t + \frac{1}{2} (w + \bar{w})z + \frac{i}{2} (\bar{w}-w)y + \frac{1}{2} w\bar{w}(t-x).
\end{align*}

\noindent Now construct the components of $A$

\begin{align*}
t' - z' & = t - z \frac{1}{2}(w + \bar{w})z + \frac{i}{2} (\bar{w}-w)y + \frac{1}{2}w\bar{w}(t-x) - \frac{1}{2}(\bar{w} + w)(t-x),\\
t' + z' & = t + z \frac{1}{2}(w + \bar{w})z + \frac{i}{2} (\bar{w}-w)y + \frac{1}{2}w\bar{w}(t-x) + -\frac{1}{2}(\bar{w} + w)(t-x),\\
x' + iy' & = x +  \frac{1}{2}(w + \bar{w})z + \frac{i}{2} (\bar{w}-w)y + \frac{1}{2}w\bar{w}(t-x) - \frac{1}{2}(\bar{w} + w)(t-x) + iy.
\end{align*}

\noindent Equating coefficients of $x$, $y$, $z$, $t$ on both sides of Eqn.(\ref{general_coeff_equate_a}) to obtain

\begin{subequations}
\begin{gather}\label{Ex_Ap_equate_coeffs_first_a}
\alpha \bar{\beta} + \bar{\alpha} \beta = \frac{1}{2}(\bar{w}+w) - \frac{1}{2}w\bar{w}, \\\label{Ex_Ap_equate_coeffs_first_b}
i (\alpha \bar{\beta} - \bar{\alpha} \beta) = \frac{1}{2} (\bar{w} - w), \\\label{Ex_Ap_equate_coeffs_first_c}
-\alpha \bar{\alpha} + \beta \bar{\beta} = - 1 + \frac{1}{2}(\bar{w} + w), \\\label{Ex_Ap_equate_coeffs_first_d}
\alpha \bar{\alpha} + \beta \bar{\beta} = 1. 
\end{gather}
\end{subequations}

\noindent Then Eqn.(\ref{Ex_Ap_equate_coeffs_first_a}) and (\ref{Ex_Ap_equate_coeffs_first_b}) imply $\alpha \bar{\beta} = \frac{1}{2}\bar{w} - \frac{1}{4}w\bar{w}$. Also Eqn.(\ref{Ex_Ap_equate_coeffs_first_c}) and (\ref{Ex_Ap_equate_coeffs_first_d}) imply $2\beta\bar{\beta} = \frac{1}{4}w\bar{w}$ and $\alpha \bar{\alpha} = 1 + \frac{1}{4}w\bar{w} - \frac{1}{2}\bar{w} - \frac{1}{2}w$. Hence $\alpha$ can be written in terms of $\beta$

\begin{gather*}
\alpha \bar{\beta} = (\frac{1}{2}\bar{w} - \frac{1}{4}w\bar{w})\beta, \\
\frac{1}{4} w \bar{w} \alpha = \frac{1}{2}\bar{w}(1-\frac{1}{2}w)\beta, \\
\alpha = \frac{(2-w)}{w}\beta.
\end{gather*}

Equating coefficients of $x$, $y$, $z$, $t$ on both sides of Eqn.(\ref{general_coeff_equate_b}) to obtain

\begin{subequations}
\begin{gather}\label{Ex_Ap_equate_coeffs_second_a}
\alpha \bar{\delta} + \beta\bar{\gamma} = 1 - \frac{1}{2}w\bar{w} + \frac{1}{2}(\bar{w}-w), \\\label{Ex_Ap_equate_coeffs_second_b}
\alpha \bar{\delta} - \beta\bar{\gamma} = \frac{1}{2}(\bar{w}-w) + 1,\\\label{Ex_Ap_equate_coeffs_second_c}
-\alpha\bar{\gamma} + \beta \bar{\delta} = \frac{1}{2}(\bar{w} + w) ,\\\label{Ex_Ap_equate_coeffs_second_d}
\alpha\bar{\gamma} + \beta \bar{\delta} = \frac{1}{2}w\bar{w} - \frac{1}{2}(\bar{w}-w). 
\end{gather}
\end{subequations}

\noindent Now Eqn.(\ref{Ex_Ap_equate_coeffs_second_a}) and (\ref{Ex_Ap_equate_coeffs_second_b}) imply $\beta \bar{\gamma} = -\frac{1}{4}w\bar{w}$. So using $\bar{\beta} = \frac{1}{4}w\bar{w}$ again to obtain

\begin{gather*}
\frac{1}{4}w\bar{w} \bar{\gamma} = \bar{\beta}\beta\bar{\gamma},\\
\frac{1}{4}w\bar{w} \bar{\gamma} = -\frac{1}{4}w\bar{w} \bar{\beta},\\
\gamma = -\beta.
\end{gather*}

\noindent Also Eqn.(\ref{Ex_Ap_equate_coeffs_second_c}) and (\ref{Ex_Ap_equate_coeffs_second_d}) imply $\beta \bar{\delta} = \frac{1}{4}w\bar{w} + \frac{1}{2}w$. Thus

\begin{gather*}
\frac{1}{4}w\bar{w}\bar{\delta} = \left(\frac{1}{4}w\bar{w}+ \frac{1}{2}w\right)\bar{\beta},\\
\delta = \left(\frac{w\bar{w} + 2\bar{w}}{w\bar{w}}\right)\beta.
\end{gather*}


(DONE TO HERE)


\noindent Equating coefficients of $x$, $y$, $z$, $t$ on both sides of Eqn.(\ref{general_coeff_equate_c}) to obtain

\begin{subequations}
\begin{gather}\label{Ex_Ap_equate_coeffs_third_a}
\gamma \bar{\delta} + \delta \bar{\gamma} = 0, \\\label{Ex_Ap_equate_coeffs_third_b}
\gamma \bar{\delta} - \delta \bar{\gamma} = 0 ,\\\label{Ex_Ap_equate_coeffs_third_c}
-\gamma \bar{\gamma} + \delta \bar{\delta} = 1 ,\\\label{Ex_Ap_equate_coeffs_third_d}
\gamma \bar{\gamma} + \delta \bar{\delta} = 1. 
\end{gather}
\end{subequations}

\noindent Eqn(\ref{Ex_Ap_equate_coeffs_third_a}) and (\ref{Ex_Ap_equate_coeffs_third_b}) are satisfied since $\gamma = 0$, this also implies that $\delta \bar{\delta} = 1$ from Eqn(\ref{Ex_Ap_equate_coeffs_third_c}). Now use the fact that $\det{(U)} = 1$, which implies

\begin{equation*}
\alpha \delta - \beta \gamma = 1,
\end{equation*}

\noindent thus $\alpha \delta = 1$ as $\beta = 0$. Then using $\alpha \bar{\alpha} = 1$ again and $\delta \bar{\delta} = 1$

\begin{align*}
\alpha^2 e^{-\theta} & = 1, \\
\alpha^2 & = e^{-i\theta}, \\
\alpha & = \pm e^{{-i\theta}/2}, 
\end{align*}






\noindent which finally implies that $\delta = \pm e^{i\theta/2}$. Hence there are 2 matrices $U$ corresponding to the spacial rotation, namely

\begin{equation*}
U = \pm
\left(
\begin{array}{cc}
e^{-i\theta/2} & 0            \\
0              & e^{i\theta/2} \\
\end{array}
\right)
\end{equation*}











\end{appendix}
