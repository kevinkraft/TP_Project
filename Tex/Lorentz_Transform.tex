\section{The Lorentz Transformation}

In this section the two null directions inherent in all Lorentz transformations, except the singular Lorentz transformations, are illustrated. The Lorentz Transform is defined by 

\begin{eqnarray*}
(x,y,z,t) \rightarrow (x',y',z',t') \text{ such that } \\
{x'}^2 + {y'}^2 + {z'}^2 - {t'}^2 = x^2 + y^2 + z^2 - t^2 
\end{eqnarray*}

\noindent If the transformation preserves the orientation of the spatial axes then is it called a proper Lorentz transformation. This is equivalent to saying the transformation does not change the handedness of the axes. Also If $t \geq 0 \Rightarrow t' \geq 0$ then it is called an orthochronous Lorentz transforamtion. This ensures that the time direction is preserved. In this project the ``Lorentz transforamation'' will refer to the proper, orthochronous Lorentz transformation.

Consider a photon moving in the $x$ direction at the speed of light, $c = 1$, and starting at $x = 0$. The space-time for such a photon can be illustrated as follows (FIGURE). It is clear that there are two null directions in this space-time, $x = \pm t$. To see this use the standard Lorentz transformation:

\begin{eqnarray*}
x'  = \gamma (x - vt) & \text{, where } \gamma = {(1 - v^2)}^{-1/2} \\
t'  = \gamma (t - vx) & 
\end{eqnarray*}

\noindent Rearranging:

\begin{eqnarray*}
x' - t' = \gamma (1 + v) (x - t) \\
x' + t' = \gamma (1 - v) (x + t)
\end{eqnarray*}

\noindent It is clear that:

$$x = \pm t \leftrightarrow x' = \pm t'$$

\noindent Thus there are two null directions (are null directions by definition invarient??? yes)in this space-time at $x = \pm t$. It can be shown that all Lorentz transformations have two invarient null directions except the singular Lorentz transformation which has only one fixed null direcion.  
